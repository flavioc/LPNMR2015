Linear logic has been used in the past as a basis for logic-based programming
languages~\cite{Miller85anoverview}, including forwards-chaining and
backwards-chaining programming languages.
LinLog~\cite{Andreoli92logicprogramming} is a backwards-chaining programming
language that originated from the idea of using focused proofs as a basis for a
programming language.  Lolli, a programming language presented
in~\cite{Hodas94logicprogramming}, is based on a fragment of intuitionistic
linear logic and proves goals by lazily managing the context of linear resources
during backwards-chaining proof search.
LolliMon~\cite{Lopez:2005:MCL:1069774.1069778} is a concurrent linear logic
programming language that integrates both forward-chaining and
backwards-chaining search, where forward-chaining computations are encapsulated
inside a monad and are concurrent, while backwards-chaining search is done
sequentially.

Proof-theoretic methods have been used extensively for specifying the
operational semantics~\cite{Cervesato96efficientresource} and
compilation~\cite{DBLP:journals/corr/abs-1210-1653} aspects of logic programming
languages. However, the available literature for linear logic programming only
deals with backwards-chaining and the resource management issues that arise in
such paradigm.
