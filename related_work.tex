Linear logic has been used in the past as a basis for logic-based programming
languages~\cite{Miller85anoverview}, including forwards-chaining and
backwards-chaining programming languages.
LinLog~\cite{Andreoli92logicprogramming} is a backwards-chaining programming
language that originated from the idea of using focused proofs as a basis for a
programming language. Focused proofs are tractable since most irrelevant
choices in during search are removed from the sequent calculus, making them
ideal for a programming language. The semantics presented for LM also use
focusing as a basis for developing the high level semantics and relating the
operational semantics with linear logic.

Lolli, a programming language presented in~\cite{Hodas94logicprogramming}, is
based on a fragment of intuitionistic linear logic and proves goals by lazily
managing the context of linear resources during backwards-chaining proof search.
LolliMon~\cite{Lopez:2005:MCL:1069774.1069778} is a concurrent linear logic
programming language that integrates both forward-chaining and
backwards-chaining search, where forward-chaining computations are encapsulated
inside a monad and are concurrent, while backwards-chaining search is done
sequentially. The programs starts by performing backwards-chaining search but
this can be suspended in order to perform forward-chaining search. This
concurrent forward-chaining search stops until a fix-point is achieved, after
which backwards-chaining search is resumed. LolliMon is derived from the a
concurrent logical framework called
CLF~\cite{Watkins:2004uq,Cervesato02aconcurrent,Watkins03aconcurrent}.

As a forward-chaining linear logic programming language, LM shares similarities
with Constraint Handling Rules
(CHR)~\cite{Betz:2005kx,Betz:2013:LBA:2422085.2422086}.  CHR is a concurrent
committed-choice constraint language used to write constraint solvers. A CHR
program is a set of rules and a set of constraints (which can be seen as facts).
Constraints can be consumed or generated during the application of rules.
Unlike LM, in CHR there is no concept of rule priorities, but there is an
extension to CHR that supports them~\cite{DeKoninck:2007:URP:1273920.1273924}.
Comprehension patterns have been researched recently by Lam et
al.~\cite{DBLP:journals/corr/LamC14}.  Finally, there is also a CHR extension
that adds persistent constraints and it has been proven to be sound and
complete~\cite{DBLP:journals/corr/abs-1007-3829}.
