The fragment of linear logic used by LM is presented in
Fig.~\ref{fig:sequent_calculus} in the form of the sequent calculus.
The sequent is written as $\Psi; \seqx{\Gamma}{\Delta}{C}$ and
can be read as "assuming persistent resources $\Gamma$ and linear resources
$\Delta$ then $C$ is true". More specifically, $\Psi$ is the typing context,
$\Gamma$ is a multi-set of persistent resources, $\Delta$ is a multi-set of
linear resources while $C$ is the proposition we want to prove.

\begin{figure}[ht]
{\stuffsize
\[
\infer[\otimes R]
{\Psi ; \seqx{\Gamma}{\Delta, \Delta'}{A \otimes B}}
{\Psi ; \seqx{\Gamma}{\Delta}{A} & \Psi ; \seqx{\Gamma}{\Delta}{B}}
\tab
\infer[\otimes L]
{\Psi ;\seqx{\Gamma}{\Delta, A \otimes B}{C}}
{\Psi ; \seqx{\Gamma}{\Delta, A, B}{C}}
\]


\[
   \infer[\with L_1]
   {\Psi; \seqx{\Gamma}{\Delta, A \with B}{C}}
   {\Psi; \seqx{\Gamma}{\Delta, A}{C}}
   \tab
   \infer[\with L_2]
   {\Psi; \seqx{\Gamma}{\Delta, A \with B}{C}}
   {\Psi; \seqx{\Gamma}{\Delta, B}{C}}
   \tab
   \infer[\with R]
   {\Psi; \seqx{\Gamma}{\Delta}{A \with B}}
   {\Psi; \seqx{\Gamma}{\Delta}{A} & \Psi; \seqx{\Gamma}{\Delta}{B}}
\]

\[
\infer[\lolli R]
{\Psi ; \seqx{\Gamma}{\Delta}{A \lolli B}}
{\Psi ; \seqx{\Gamma}{\Delta, A}{B}}
\tab
\infer[\lolli L]
{\seqx{\Gamma}{\Delta, \Delta', A \lolli B}{C}}
{\Psi ; \seqx{\Gamma}{\Delta}{A} &
   \Psi ; \seqx{\Gamma}{\Delta', B}{C}}
\]

\[
\infer[\bang R]
{\Psi ; \seqx{\Gamma}{\cdot}{\bang A}}
{\Psi ; \seqx{\Gamma}{\cdot}{A}}
\tab
\infer[\bang L]
{\Psi ; \seqx{\Gamma}{\Delta, \bang A}{C}}
{\Psi ; \seqx{\Gamma, A}{\Delta}{C}}
\tab
\infer[\m{copy}]
{\Psi ; \seqx{\Gamma, A}{\Delta}{C}}
{\Psi ; \seqx{\Gamma, A}{\Delta, A}{C}}
\]

\[
\infer[\one R]
{\Psi ; \seqx{\Gamma}{\cdot}{\one}}
{}
\tab
\infer[\one L]
{\Psi ; \seqx{\Gamma}{\Delta, \one}{C}}
{\Psi ; \seqx{\Gamma}{\Delta}{C}}
\tab
\infer[id_A]
{\Psi ; \seqx{\Gamma}{A}{A}}
{}
\]

\[
\infer[\forall R]
{\Psi ; \seqx{\Gamma}{\Delta}{\forall_{n:\tau}. A}}
{\Psi, m:\tau ; \seqx{\Gamma}{\Delta}{A\{m/n\}}}
\tab
\infer[\forall L]
{\Psi ; \seqx{\Gamma}{\Delta, \forall_{n:\tau}. A}{C}}
{\Psi \vdash M : \tau & \Psi ; \seqx{\Gamma}{\Delta, A\{M/n\}}{C}}
\]
\[
\infer[cut_A]
{\Psi ; \seqx{\Gamma}{\Delta, \Delta'}{C}}
{\Psi ; \seqx{\Gamma}{\Delta}{A} & \Psi ; \seqx{\Gamma}{\Delta', A}{C}}
\tab
\infer[cut\bang_A]
{\Psi ; \seqx{\Gamma}{\Delta}{C}}
{\Psi ; \seqx{\Gamma}{\cdot}{A} & \Psi ; \seqx{\Gamma, A}{\Delta}{C}}
\]
}
\label{fig:sequent_calculus}
\caption{Fragment of linear logic used to implement LM.}
\vspace{-\bigskipamount}
\end{figure}

\paragraph{From the Sequent Calculus to LM}

\newcommand{\agg}[2]{\mathcal{R}_{#1}^{(#2)}}

We translate the rule of the asynchronous PageRank~(Fig.~\ref{code:pagerank}) to
a proposition in the sequent calculus:

{\stuffsize
\begin{align}
   \forall_A. \forall_{OldRank}. \mathtt{pagerank}(A, OldRank) \lolli
   \defsone{agg}{A, 0}
\end{align}
}

The translation is fairly straightforward, except for the aggregate. Each
comprehension\footnote{Comprehensions are a special case of aggregates without
aggregate operators.} and aggregate of a LM program must be assigned to an
unique name and a corresponding persistent term. For the symbol $agg$ we have
the following persistent term:

{\stuffsize
\begin{multline}
   \bang \forall_A. \forall_V. \defsone{agg}{A, V} \lolli (\mathtt{pagerank}(A,
   damp/P + (1.0 - damp) \times V) \with \\ (\forall_V'. \forall_B.
   (\mathtt{edge}(A, B) \otimes \mathtt{pagerank}(B, V')) \lolli
   (\mathtt{edge}(A, B) \otimes \mathtt{pagerank}(B, V')) \otimes \defsone{agg}{A, V
   + V'}))
\end{multline}
}

Notice that the argument list of $\defsone{agg}{A, V}$ is being used as
$\forall$ variables that need to be recursively passed around. Any variable used
in the aggregate expression must be part of the argument list.  For aggregates,
have an extra variable (in the example, $V$) that accumulates the value of the
aggregate. As we are going to see next, by making the aggregate term persistent,
we are allowed to use it multiple times in order to compute the value of the
aggregate.
