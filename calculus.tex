\fragment is based on a fragment of linear plus an extension we call
\emph{iterative definitions}.  The sequent is written as $\Psi;
\seqx{\Gamma}{\Delta}{C}$ and can be read as "assuming persistent resources
$\Psi$ and linear resources $\Delta$ then $C$ is true".  More specifically,
$\Psi$ is the typing context, $\Gamma$ is a multi-set of persistent resources,
$\Delta$ is a multi-set of linear resources while $C$ is the proposition we want
to prove.

{\stuffsize
\[
\infer[\otimes R]
{\Psi ; \seqx{\Gamma}{\Delta, \Delta'}{A \otimes B}}
{\Psi ; \seqx{\Gamma}{\Delta}{A} & \Psi ; \seqx{\Gamma}{\Delta}{B}}
\tab
\infer[\otimes L]
{\Psi ;\seqx{\Gamma}{\Delta, A \otimes B}{C}}
{\Psi ; \seqx{\Gamma}{\Delta, A, B}{C}}
\]

\[
\infer[\lolli R]
{\Psi ; \seqx{\Gamma}{\Delta}{A \lolli B}}
{\Psi ; \seqx{\Gamma}{\Delta, A}{B}}
\tab
\infer[\lolli L]
{\seqx{\Gamma}{\Delta, \Delta', A \lolli B}{C}}
{\Psi ; \seqx{\Gamma}{\Delta}{A} &
   \Psi ; \seqx{\Gamma}{\Delta', B}{C}}
\]

\[
\infer[\one R]
{\Psi ; \seqx{\Gamma}{\cdot}{\one}}
{}
\tab
\infer[\one L]
{\Psi ; \seqx{\Gamma}{\Delta, \one}{C}}
{\Psi ; \seqx{\Gamma}{\Delta}{C}}
\tab
\infer[id_A]
{\Psi ; \seqx{\Gamma}{A}{A}}
{}
\]

\[
\infer[\forall R]
{\Psi ; \seqx{\Gamma}{\Delta}{\forall_{n:\tau}. A}}
{\Psi, m:\tau ; \seqx{\Gamma}{\Delta}{A\{m/n\}}}
\tab
\infer[\forall L]
{\Psi ; \seqx{\Gamma}{\Delta, \forall_{n:\tau}. A}{C}}
{\Psi \vdash M : \tau & \Psi ; \seqx{\Gamma}{\Delta, A\{M/n\}}{C}}
\]
\[
\infer[cut_A]
{\Psi ; \seqx{\Gamma}{\Delta, \Delta'}{C}}
{\Psi ; \seqx{\Gamma}{\Delta}{A} & \Psi ; \seqx{\Gamma}{\Delta', A}{C}}
\tab
\infer[cut\bang_A]
{\Psi ; \seqx{\Gamma}{\Delta}{C}}
{\Psi ; \seqx{\Gamma}{\cdot}{A} & \Psi ; \seqx{\Gamma, A}{\Delta}{C}}
\]
}

Iterative definitions are used to describe comprehensions and aggregates. This
connective is a definition that can be unfolded recursively for an arbitrary
number of times and is inspired in the Baelde's work on least and greatest fixed
points in linear logic~\cite{Baelde:2012:LGF:2071368.2071370}. Baelde's system
goes beyond simple recursive definitions and allows proofs using induction and
co-induction in linear logic.  Iterative definitions are written as
$\iters{name}{\widehat{V}}$, where $name$ is the identifier of the definition.

{\stuffsize
\[
\infer[\itersname^* R]
{\Psi ; \seqx{\Gamma}{\Delta}{\iters{name}{\widehat{V}}}}
{\Psi ; \seqx{\Gamma}{\Delta}{\iter{name}{N}{\widehat{V}}}}
\tab
\infer[\itersname^* L]
{\Psi ; \seqx{\Gamma}{\Delta, \iters{name}{\widehat{V}}}{C}}
{\Psi ; \seqx{\Gamma}{\Delta, \iter{name}{N}{\widehat{V}}}{C}}
\]
}

A definition $name$ has the following definition:

{\stuffsize
\begin{align}
\iter{name}{0}{\widehat{V}} & \defeq \iterunfoldz{C}{V} \\
\iter{name}{N}{\widehat{V}} & \defeq \iterunfold{name}{N-1}{x}{V}{A}
\end{align}
}

Where $\widehat{V}$ is a list of arguments and $\mathtt{op}$ is some function
that merges its arguments. Terms $A$ and $C$ are not allowed to have other
iterative definitions.  Each $\iter{name}{N}{\widehat{V}}$ has left and right
rules for the case when $N
> 0$.

{\stuffsize
\[
\infer[\itersname^N R]
{\Psi ; \seqx{\Gamma}{\Delta}{\iter{name}{N}{\widehat{V}}}}
{\Psi ; \seqx{\Gamma}{\Delta}{\forall_{\widehat{x}}. (A \widehat{x} \otimes \iter{name}{N-1}{(\iterop{x}{V})}})}
\tab
\infer[\itersname^N L]
{\Psi ; \seqx{\Gamma}{\Delta, \iter{name}{N}{\widehat{V}}}{C}}
{\Psi ; \seqx{\Gamma}{\Delta, \forall_{\widehat{x}}. (A \widehat{x} \otimes \iter{name}{N-1}{(\iterop{x}{V})})}{C}}
\]
}

Otherwise, if $N = 0$, then the iteration stops:

{\stuffsize
\[
\infer[\itersname^0 R]
{\Psi ; \seqx{\Gamma}{\Delta}{\iter{name}{0}{\widehat{V}}}}
{\Psi ; \seqx{\Gamma}{\Delta}{(\lambda_{\widehat{x}}. C)\widehat{V}}}
\tab
\infer[\itersname^0 L]
{\Psi ; \seqx{\Gamma}{\Delta, \iter{name}{0}{\widehat{V}}}{C}}
{\Psi ; \seqx{\Gamma}{\Delta, (\lambda_{\widehat{x}}. C)\widehat{V}}{C}}
\]
}

Finally, we complete the linear logic system with the \emph{cut rules} and the
\emph{identity rule}:

\paragraph{Cut Reduction} To use \fragment, we prove that it is
\emph{consistent} by defining a new sequent $\Psi; \seqnocut{\Gamma}{\Delta}{C}$
without the cut rules and then proving that it is possible to prove everything
without cuts.

\begin{theorem}[Cut elimination]
If $\Psi ; \seqx{\Gamma}{\Delta}{A}$ then $\Psi ; \seqnocut{\Gamma}{\Delta}{A}$.
\end{theorem}
\begin{proof}
Induction on the structure of the assumption. All cases are trivial except for
the cut rules, where we use the admissibility of cut.
\end{proof}

\begin{theorem}[Cut Admissibility]
If $\Psi ; \seqnocut{\Gamma}{\Delta}{A}$ and $\Psi ; \seqnocut{\Gamma}{\Delta',
   A}{C}$ then $\Psi ; \seqnocut{\Gamma}{\Delta, \Delta'}{C}$.
\end{theorem}
\begin{proof}
By a nested induction, first on the structure of $A$ and on the structures of
the first or the second assumption.
\end{proof}

\paragraph{From the \fragment sequent calculus to LM}

We translate the rule of the asynchronous PageRank~(Fig.~\ref{code:pagerank}) to
a proposition in \fragment:\footnote{Comprehensions are a special case of
aggregates where nothing is aggregated and nothing is derived at the end.}

{\stuffsize
\begin{align}
\forall_A. \forall_{OldRank}. \mathtt{pagerank}(A, OldRank) \lolli \itersz{agg}
\; A \; 0
\end{align}
}

The translation is fairly straightforward, except for the aggregate. Each
comprehension and aggregate of a LM program must be assigned to an unique name and its
corresponding terms. For the iterative definition $agg$, it is defined as
following:

{\stuffsize
\begin{align}
\iterz{agg}{0} A \; S & \defeq \mathtt{pagerank}(A, damp/P + (1.0 -
         damp) * S)\\
\iterz{agg}{N} A \; S & \defeq \forall_V. \forall_B. ((\mathtt{edge}(A, B) \otimes
         \mathtt{pagerank}(B, V)) \lolli \iterz{agg}{N-1} A \; (S + V)
\end{align}
}

Notice that the argument list of the iterative definition is being used
to pass around terms from outside the definition, in this case, the variable
$A$. The variable $S$ accumulates the aggregate value.
