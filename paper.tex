% This is LLNCS.DEM the demonstration file of
% the LaTeX macro package from Springer-Verlag
% for Lecture Notes in Computer Science,
% version 2.4 for LaTeX2e as of 16. April 2010
%
\documentclass{llncs}
%
\usepackage{makeidx}  % allows for indexgeneration
\usepackage{graphicx}
\usepackage{times}

\usepackage{enumitem}
\usepackage{amsmath}
\usepackage{url}
\usepackage{fancyvrb}
\usepackage{verbatim}
\usepackage{caption}
\usepackage{subfig}
\usepackage{multirow}
\usepackage{floatrow}
\usepackage{proof-dashed}
%\usepackage{amsthm}
\usepackage{nth}
\usepackage[numbers,sort]{natbib}
\usepackage[backref,pageanchor=true,plainpages=false, pdfpagelabels, bookmarks,bookmarksnumbered]{hyperref}
\input{fp-macros}
\input{macros}

%
\begin{document}
\newcommand{\cmu}{\ensuremath{^\dag}}
\newcommand{\fcup}{\ensuremath{^\ddag}}

\title{Proof-Theoretic Semantics For Forward-Chaining Linear Logic Programming with Comprehensions and Aggregates}
\author{Flavio Cruz\inst{1,2}, Frank Pfenning\inst{1}}

\institute{Carnegie Mellon University, Pittsburgh, PA 15213\\
       \url{{fmfernan, fp}@cs.cmu.edu} \\
       \and
       CRACS \& INESC TEC, Faculty of Sciences, University Of Porto\\
       Rua do Campo Alegre, 1021/1055, 4169-007 Porto, Portugal\\
       \url{flavioc@dcc.fc.up.pt}}

%\newtheorem{lemma}{Lemma}[section]
%\newtheorem{theorem}{Theorem}[section]
%\newtheorem{definition}{Definition}[section]

\maketitle
\begin{abstract}
We describe the LLAM, an abstract machine that specifies the operational
behaviour of a linear logic programming language that supports comprehensions
and aggregates. Comprehensions iterate over combinations of facts to generate
other facts while aggregates iterate over facts to aggregate values. We present
the linear logic fragment used in the language and a high level dynamic
semantics that closely mimics forward-chaining with focusing on the sequent
calculus.  We prove that the LLAM is sound in relation to the high level
semantics, meaning that any computation performed in the former can also be done
in the latter. The soundness proof further shows how focused proofs with
forward-chaining can be mapped to a low level representation that determines all
the steps of the computation process.

\end{abstract}
\pagestyle{headings}  % switches on printing of running heads

\section{Introduction}

Datalog~\cite{Ramakrishnan93asurvey} is a forward-chaining logic programming
language originally for deductive databases. In Datalog, the program is composed
of a database of facts and a set of rules. Datalog programs first populate the
database with axioms and then saturate the database using rule inference. In
Datalog logical facts are persistent, therefore once a fact is inferred, it
cannot be deleted. However, there has been a growing interest in integrating
linear logic~\cite{girard-87} into bottom-up logic
programming~\cite{Chang03ajudgmental,cruz-iclp14,Lopez:2005:MCL:1069774.1069778,simmons-lla},
allowing for both fact assertion and fact deletion.

Linear logic programming greatly increases the expressiveness of traditional
Datalog since it allows the programmer to manage state in a structured fashion.
However, such approach still lacks some facilities that are common in other
languages such as comprehensions and aggregates. Comprehensions have been
popular among functional programmers for years~\cite{npl1977} and aggregates
have found its way into
Datalog~\cite{Consens93lowcomplexity,Greco:1999:DPD:627321.627989} and also
functional programming in the form of
\emph{folds}~\cite{Hutton:1999:TUE:968578.968579}.  In this paper, we present
LM~\cite{cruz-iclp14}, a linear logic programming language that supports
comprehensions and aggregates.

LM is a concurrent programming language for programming over graph data
structures that supports both linear and persistent facts. LM is based on
\fragment, a logic that is a fragment of intuitionistic linear logic plus a
small extension we call \emph{iterative definitions} that were introduced to
logically justify comprehensions and aggregates. In LM, comprehensions allow the
programmer to iterate over combinations of facts to derive new facts, while
aggregates iterate over facts in order to aggregate values from such
combinations.

We present a high level dynamic semantics and then a low level dynamic
semantics of a fragment of LM.  Both semantics will be presented using
proof-theoretic methods. The high level semantics are closely related to the
sequent calculus of \fragment since it amounts to focusing with
forward-chaining~\cite{Andreoli92logicprogramming,laurent2004proof}, where atoms
have a positive polarity.  The low level semantics are much closer to a real
implementation since they remove most of non-determinism of the high level
semantics and also describe in detail how the focused proof search mechanism
works, including backtracking and resource management. We also relate both
semantics by proving that the low level dynamic semantics is sound in relation
to the high level dynamic semantics.

The contributions of this paper are 2-fold: (1) a low level dynamic semantics
for a forward-chaining linear logic programming language and the connection to its
logic based on proof-theoretic methods and (2) a new proof-theoretic approach
called iterative definitions to useful logic programming constructs such as
comprehension and aggregates. The paper is organized as follows. In Section 2 we
present a simple LM program. Next, in Section 3 we present the sequent calculus of
\fragment, the logic used for LM. In Sections 4 and 5 we present the high
level dynamic semantics and the low level dynamic semantics. The soundness proof
is presented in Section 6 and then we conclude with related work and
conclusions.


\section{The LM Language}
Linear Meld (LM) is a logic programming language that offers a declarative and
structured way to manage state using linear logic. A program consists of a
database and a set of inference rules. The database is satured by applying
inference rules to the facts in the database. LM distinguishes between
\emph{linear facts}, which are retracted after an inference, and
\emph{persistent facts} which cannot be retracted and are preceded by a
\code{\bang} symbol.

Figure~\ref{code:visit} presents a LM program that visits all the nodes of a
graph. Given a graph $G = (E, V)$, the program starts with a database that has a
fact \code{edge(a, b)} for every edge $(a, b) \in E$ and a fact
\code{unvisited(a)} for every node $a \in G$. The idea of the program is to
initially visit node 1 (line~\ref{line:visit_axiom}) by applying the first rule
(lines 3-5) which retracts fact \code{visit(1)} and \code{unvisited(1)} and
derives fact \code{visited(1)} and a \code{visit(b)} for all edges $(1, b) \in
E$ using a comprehension (line~\ref{line:visit_comprehension}). The
comprehension has two parts: \code{edge(A, B)} represents the body and
\code{visit(B)} represents the head. Semantically, the comprehension is a rule
of the form \code{edge(A, B) $\lolli$ visit(B)} that is inferred as many times
as the database allows. In this case, all the edges of the node are deleted to
infer \code{visit}. The second rule of the program
(lines~\ref{line:visit_rule21}-\ref{line:visit_rule22}) is inferred when a node
has been visited already. For a connected graph $G$, all the \code{unvisited}
facts will be deleted and a fact \code{visited(a)} will be derived for every
node $a \in G$.

\begin{figure}[h]
\begin{Verbatim}[numbers=left,commandchars=\*\#\&,fontsize=\stuffsize,xleftmargin=\stuffleftmargin]
visit(1). // start at node 1.*label#line:visit_axiom&

visit(A), unvisited(A) // change from unvisited to visited.
   -o visited(A),
      {B | edge(A, B) | visit(B)}.*label#line:visit_comprehension&
 
visit(A), visited(A) // node already visited.*label#line:visit_rule21&
   -o visited(A).*label#line:visit_rule22&
\end{Verbatim}
\caption{LM program to perform a visit of all nodes in a connected graph.}
  \label{code:visit}
\vspace{-5mm}
\end{figure}

Figure~\ref{code:pagerank} presents a fragment of an asynchronous PageRank
program that sums all the pagerank values of neighbor nodes. The rule deletes
the old \code{pagerank} fact and then derives an aggregate. The first argument
contains \code{sum} (the aggregate operator) and $V$ (the aggregate value). The
second argument \code{edge(A, B), pagerank(B, V)} is the body of the aggregate
that is iterated over to derive \code{edge(A, B), pagerank(B, V)}. The values
$V_i$ of each combination of \code{edge} and \code{pagerank} are summed up and
instantiated as $V = \sum_i V_i$ in \code{pagerank(A, damping/PAGES + (1.0 -
damping) * V)} to compute the new pagerank as the sum of the pageranks of the
neighborhood.

\begin{figure}[h]
\begin{Verbatim}[numbers=left,fontsize=\stuffsize,xleftmargin=\stuffleftmargin]
pagerank(A, OldRank)
      -o [sum => V | B | edge(A, B), pagerank(B, V) |
            edge(A, B), pagerank(B, V) |
            pagerank(A, damp/P + (1.0 - damp) * V)].
\end{Verbatim}
   \caption{Inference rule to update the pagerank of a node.}
   \label{code:pagerank}
\vspace{-5mm}
\end{figure}

\paragraph{Operational Semantics} Each rule in LM has a defined priority that is
inferred from its position in the source file.  Rules at the beginning of the
file have higher priority. We consider all the new facts that have been not
considered yet to create a set of \emph{candidate rules}.  The set of candidate
rules is then applied (by priority) and updated as new facts are derived.  Rules
are inferred atomically and both comprehensions and aggregates cannot use facts
derived during the current rule.

LM also imposes restrictions on the inference rules by forcing that every fact
used in the body of a rule uses the same first argument. This allows the
language to be concurrent since the database of facts is partitioned (by the
first argument) into many small databases that can infer rules independently.
In the remainder of this paper, we ignore this detail and consider the database
as a whole.

\paragraph{Abstract Syntax} 

Table~\ref{tbl:ast} shows the abstract syntax for rules in LM. A LM program
$Prog$ consists of a set of derivation rules $\Sigma$ and a database $D$.
Comprehensions and aggregates only appear on the head of the rule and only use
simple fact expressions.

\noindent \begin{table}[h]
\vspace{-5mm}
\captionsetup[subfloat]{labelformat=empty}
\subfloat[]{
\stuffsize\begin{tabular}{ l l c l }
  Program & $Prog$ & $::=$ & $\Sigma, D$ \\
  Set Of Rules & $\Sigma$ & $::=$ & $\cdot \; | \; \Sigma, R$\\
  Database & $D$ & $::=$ & $\Gamma, \Delta$ \\
  Linear Database & $\Delta$ & $::=$ & $\cdot \; | \; \Delta, l(\hat{t})$ \\
  Persistent Database & $\Gamma$ & $::=$ & $\cdot \; | \; \Gamma, \bang p(\hat{t})$ \\
  Linear Fact & $L$ & $::=$ & $l(\hat{x})$\\
  Persistent Fact & $P$ & $::=$ & $\bang p(\hat{x})$\\
\end{tabular}
} \subfloat[]{
\stuffsize\begin{tabular}{l l c l}
  Rule & $R$ & $::=$ & $BE \lolli HE \; | \; \forall_{x}. R$ \\
  Body & $BE$ & $::=$ & $L \; | \; P \; | \; BE, BE \; | \; \one$\\
  Head & $HE$ & $::=$ & $L \; | \; P \; | \; HE, HE \; | \; CE \; | \; AE \; | \; \one$\\
  
  Comprehension & $CE$ & $::=$ & $\comprehension{\widehat{x}}{SB}{SH}$ \\
  Aggregate & $AE$ & $::=$ & $\aggregate{\mathtt{Op}}{y}{\widehat{x}}{SB}{SH_1}{SH_2}$ \\
  
  Sub-Body & $SB$ & $::=$ & $L \; | \; P \; | \; SB, SB$ \\
  Sub-Head & $SH$ & $::=$ & $L \; | \; P \; | \; SH, SH \; | \; \one$\\
\end{tabular}
}
\caption{Abstract syntax for a fragment of LM.}\label{tbl:ast}
\vspace{-7mm}
\end{table}


\section{The \fragment Logic}
\fragment is based on a fragment of linear plus an extension we call
\emph{iterative definitions}.  The sequent is written as $\Psi;
\seqx{\Gamma}{\Delta}{C}$ and can be read as "assuming persistent resources
$\Psi$ and linear resources $\Delta$ then $C$ is true".  More specifically,
$\Psi$ is the typing context, $\Gamma$ is a multi-set of persistent resources,
$\Delta$ is a multi-set of linear resources while $C$ is the proposition we want
to prove.

{\stuffsize
\[
\infer[\otimes R]
{\Psi ; \seqx{\Gamma}{\Delta, \Delta'}{A \otimes B}}
{\Psi ; \seqx{\Gamma}{\Delta}{A} & \Psi ; \seqx{\Gamma}{\Delta}{B}}
\tab
\infer[\otimes L]
{\Psi ;\seqx{\Gamma}{\Delta, A \otimes B}{C}}
{\Psi ; \seqx{\Gamma}{\Delta, A, B}{C}}
\]

\[
\infer[\lolli R]
{\Psi ; \seqx{\Gamma}{\Delta}{A \lolli B}}
{\Psi ; \seqx{\Gamma}{\Delta, A}{B}}
\tab
\infer[\lolli L]
{\seqx{\Gamma}{\Delta, \Delta', A \lolli B}{C}}
{\Psi ; \seqx{\Gamma}{\Delta}{A} &
   \Psi ; \seqx{\Gamma}{\Delta', B}{C}}
\]

\[
\infer[\one R]
{\Psi ; \seqx{\Gamma}{\cdot}{\one}}
{}
\tab
\infer[\one L]
{\Psi ; \seqx{\Gamma}{\Delta, \one}{C}}
{\Psi ; \seqx{\Gamma}{\Delta}{C}}
\tab
\infer[id_A]
{\Psi ; \seqx{\Gamma}{A}{A}}
{}
\]

\[
\infer[\forall R]
{\Psi ; \seqx{\Gamma}{\Delta}{\forall_{n:\tau}. A}}
{\Psi, m:\tau ; \seqx{\Gamma}{\Delta}{A\{m/n\}}}
\tab
\infer[\forall L]
{\Psi ; \seqx{\Gamma}{\Delta, \forall_{n:\tau}. A}{C}}
{\Psi \vdash M : \tau & \Psi ; \seqx{\Gamma}{\Delta, A\{M/n\}}{C}}
\]
\[
\infer[cut_A]
{\Psi ; \seqx{\Gamma}{\Delta, \Delta'}{C}}
{\Psi ; \seqx{\Gamma}{\Delta}{A} & \Psi ; \seqx{\Gamma}{\Delta', A}{C}}
\tab
\infer[cut\bang_A]
{\Psi ; \seqx{\Gamma}{\Delta}{C}}
{\Psi ; \seqx{\Gamma}{\cdot}{A} & \Psi ; \seqx{\Gamma, A}{\Delta}{C}}
\]
}

Iterative definitions are used to describe comprehensions and aggregates. This
connective is a definition that can be unfolded recursively for an arbitrary
number of times and is inspired in the Baelde's work on least and greatest fixed
points in linear logic~\cite{Baelde:2012:LGF:2071368.2071370}. Baelde's system
goes beyond simple recursive definitions and allows proofs using induction and
co-induction in linear logic.  Iterative definitions are written as
$\iters{name}{\widehat{V}}$, where $name$ is the identifier of the definition.

{\stuffsize
\[
\infer[\itersname^* R]
{\Psi ; \seqx{\Gamma}{\Delta}{\iters{name}{\widehat{V}}}}
{\Psi ; \seqx{\Gamma}{\Delta}{\iter{name}{N}{\widehat{V}}}}
\tab
\infer[\itersname^* L]
{\Psi ; \seqx{\Gamma}{\Delta, \iters{name}{\widehat{V}}}{C}}
{\Psi ; \seqx{\Gamma}{\Delta, \iter{name}{N}{\widehat{V}}}{C}}
\]
}

A definition $name$ has the following definition:

{\stuffsize
\begin{align}
\iter{name}{0}{\widehat{V}} & \defeq \iterunfoldz{C}{V} \\
\iter{name}{N}{\widehat{V}} & \defeq \iterunfold{name}{N-1}{x}{V}{A}
\end{align}
}

Where $\widehat{V}$ is a list of arguments and $\mathtt{op}$ is some function
that merges its arguments. Terms $A$ and $C$ are not allowed to have other
iterative definitions.  Each $\iter{name}{N}{\widehat{V}}$ has left and right
rules for the case when $N
> 0$.

{\stuffsize
\[
\infer[\itersname^N R]
{\Psi ; \seqx{\Gamma}{\Delta}{\iter{name}{N}{\widehat{V}}}}
{\Psi ; \seqx{\Gamma}{\Delta}{\forall_{\widehat{x}}. (A \widehat{x} \otimes \iter{name}{N-1}{(\iterop{x}{V})}})}
\tab
\infer[\itersname^N L]
{\Psi ; \seqx{\Gamma}{\Delta, \iter{name}{N}{\widehat{V}}}{C}}
{\Psi ; \seqx{\Gamma}{\Delta, \forall_{\widehat{x}}. (A \widehat{x} \otimes \iter{name}{N-1}{(\iterop{x}{V})})}{C}}
\]
}

Otherwise, if $N = 0$, then the iteration stops:

{\stuffsize
\[
\infer[\itersname^0 R]
{\Psi ; \seqx{\Gamma}{\Delta}{\iter{name}{0}{\widehat{V}}}}
{\Psi ; \seqx{\Gamma}{\Delta}{(\lambda_{\widehat{x}}. C)\widehat{V}}}
\tab
\infer[\itersname^0 L]
{\Psi ; \seqx{\Gamma}{\Delta, \iter{name}{0}{\widehat{V}}}{C}}
{\Psi ; \seqx{\Gamma}{\Delta, (\lambda_{\widehat{x}}. C)\widehat{V}}{C}}
\]
}

Finally, we complete the linear logic system with the \emph{cut rules} and the
\emph{identity rule}:

\paragraph{Cut Reduction} To use \fragment, we prove that it is
\emph{consistent} by defining a new sequent $\Psi; \seqnocut{\Gamma}{\Delta}{C}$
without the cut rules and then proving that it is possible to prove everything
without cuts.

\begin{theorem}[Cut elimination]
If $\Psi ; \seqx{\Gamma}{\Delta}{A}$ then $\Psi ; \seqnocut{\Gamma}{\Delta}{A}$.
\end{theorem}
\begin{proof}
Induction on the structure of the assumption. All cases are trivial except for
the cut rules, where we use the admissibility of cut.
\end{proof}

\begin{theorem}[Cut Admissibility]
If $\Psi ; \seqnocut{\Gamma}{\Delta}{A}$ and $\Psi ; \seqnocut{\Gamma}{\Delta',
   A}{C}$ then $\Psi ; \seqnocut{\Gamma}{\Delta, \Delta'}{C}$.
\end{theorem}
\begin{proof}
By a nested induction, first on the structure of $A$ and on the structures of
the first or the second assumption.
\end{proof}

\paragraph{From the \fragment sequent calculus to LM}

We translate the rule of the asynchronous PageRank~(Fig.~\ref{code:pagerank}) to
a proposition in \fragment:\footnote{Comprehensions are a special case of
aggregates where nothing is aggregated and nothing is derived at the end.}

{\stuffsize
\begin{align}
\forall_A. \forall_{OldRank}. \mathtt{pagerank}(A, OldRank) \lolli \itersz{agg}
\; A \; 0
\end{align}
}

The translation is fairly straightforward, except for the aggregate. Each
comprehension and aggregate of a LM program must be assigned to an unique name and its
corresponding terms. For the iterative definition $agg$, it is defined as
following:

{\stuffsize
\begin{align}
\iterz{agg}{0} A \; S & \defeq \mathtt{pagerank}(A, damp/P + (1.0 -
         damp) * S)\\
\iterz{agg}{N} A \; S & \defeq \forall_V. \forall_B. ((\mathtt{edge}(A, B) \otimes
         \mathtt{pagerank}(B, V)) \lolli \iterz{agg}{N-1} A \; (S + V)
\end{align}
}

Notice that the argument list of the iterative definition is being used
to pass around terms from outside the definition, in this case, the variable
$A$. The variable $S$ accumulates the aggregate value.


\section{High Level Dynamic Semantics}
The High Level Dynamic~(HLD) semantics formalize the mechanism of matching rules
and deriving new facts. HLD is a simple layer over the sequent calculus that
presents a simplified overview of the semantics of LM. We consider $\Gamma$ and
$\Delta$ the database of our program. $\Gamma$ contains the database of
persistent facts while $\Delta$ the database of linear facts. We assume that the
rules of the program are persistent linear implications of the form $\bang (A
\lolli B)$ that can be used several times. However, we do not put the rules in
the $\Gamma$ context but in a separate context $\Phi$. The persistent terms
associated with each comprehension and aggregate are put in the $\Pi$ dictionary
that maps symbols to persistent terms.

In HLD, we ignore the right side of the sequent calculus and use \emph{chaining}
and \emph{inversion} on the $\Delta$ and $\Gamma$ contexts so that we only have
atomic facts (e.g., the database of facts). To apply rules we use
\emph{focusing}~\cite{Andreoli92logicprogramming} on one of the derivation rules
in $\Phi$. Note that in the focusing process, we assume that all the atoms
(facts) are positive thus the chaining proceeds in a \emph{forward chaining}
fashion.~\footnote{Note that neither HLD or the low level abstract machine
   explicitly use of variable bindings when matching facts from the database. The
   formalization of bindings tends to complicate the formal system and it is not
   necessary for a good understanding of the system.}

An operational step is performed by applying a single inference rule, resulting
in a state transition represented as $\Gamma_1; \Delta_1 \rightarrow \Gamma_2;
\Delta_2$. The entry point of HLD is the judgment
$\doz{\Gamma}{\Delta}{\Phi}{\Pi}{\Xi'}{\Gamma'}{\Delta'}$. The contexts
$\Gamma$, $\Delta$ and $\Phi$ have the meaning explained before, while $\Xi'$,
$\Gamma'$ and $\Delta'$ are output multi-sets from applying one of the rules in
$\Phi$ and are usually written as $\outsem$. $\Xi'$ is the set of consumed
linear resources, $\Gamma'$ is the set of derived persistent facts and $\Delta'$
is the set of derived linear facts.  Note that for HLD semantics there is no
concept of rule priority, therefore a rule is picked non-deterministically.

In HLD computation proceeds as follows:
\begin{itemize}[leftmargin=*]

   \item The rule is unpacked by "guessing" the correct variable assignments for
   the $\forall$ connective;

   \item The linear implication $A \lolli B$ non-deterministically splits the
   set of matched facts $\Delta$ into $\Delta_a$ and $\Delta_b$, where
   $\Delta_1$ is the set of linear facts used for matching the rule body;
   Matching uses the judgment $\mz{\Gamma}{\Delta_1}{A}$ for a given
   body $A$;

   \item $\dz{\Gamma}{\Pi}{\Delta}{\Xi}{\Gamma_1}{\Delta_1}{\Omega}{\outsem}$
   treats the head of the rule $\Omega$ as an ordered context and unpacks facts
   into $\Gamma_1$ and $\Delta_1$.

\end{itemize}

Aggregates are computed during the derivation the head of a rule.  First, we
look into $\Pi$ for the appropriate persistent term and then immediatelly apply
the top linear implication ($\m{copy}$ rule in the $\Pi$ context followed by the
$\lolli L$ rule of the sequent calculus). The aggregate term is substituted by
either the final term (if the aggregate is complete)\footnote{The aggregate
shown collects all values into a list using the cons $::$ operator.} or by the
recursive case. On the recursive case, the implication is deconstructed using
$\dzname \lolli$ where the matching judgment is used again in order to match the
body of the aggregate and then derive the two heads of the aggregate. The
relevant rules are presented as follows:

{\stuffsize
\[
\infer[\dzname \m{agg}_1]
{\dz{\Gamma}{\Pi}{\Delta}{\Xi}{\Gamma_1}{\Delta_1}{\defstwo{agg}{\widehat{V}}{\Sigma},
   \Omega}{\outsem}}
{\begin{gathered}
   \Pi(\m{agg}) = \forall_{\widehat{v}, \Sigma'}.
   (\defstwo{agg}{\widehat{v}}{\Sigma'} \lolli ((\lambda x. C)\Sigma' \with (\forall_{\widehat{x}, \sigma}.
                                                (A \lolli B \otimes
                                                 \defstwo{agg}{\widehat{v}}{\Sigma'
                                                 + \sigma})))) \\
   \dz{\Gamma}{\Pi}{\Delta}{\Xi}{\Gamma_1}{\Delta_1}{\forall_{\widehat{x},
   \sigma}. (A \lolli B \otimes \defstwo{agg}{\widehat{v}}{\Sigma
    + \sigma})\{\widehat{V}/\widehat{v}\}\{\Sigma/\Sigma'\}, \Omega}{\outsem}
   \end{gathered}
}
\]

\vspace{-5mm}

\[
\infer[\dzname \m{agg}_2]
{\dz{\Gamma}{\Pi}{\Delta}{\Xi}{\Gamma_1}{\Delta_1}{\defstwo{agg}{\widehat{V}}{\Sigma},
   \Omega}{\outsem}}
{\begin{gathered}
   \Pi(\m{agg}) = \forall_{\widehat{v}, \Sigma'}.
   (\defstwo{agg}{\widehat{v}}{\Sigma'} \lolli ((\lambda x. C)\Sigma' \with (\forall_{\widehat{x}, \sigma}.
                                                (A \lolli B \otimes
                                                 \defstwo{agg}{\widehat{v}}{\Sigma'
                                                 + \sigma})))) \\
   \dz{\Gamma}{\Pi}{\Delta}{\Xi}{\Gamma_1}{\Delta_1}{(\lambda x.
      C\{\widehat{V}/\widehat{v}\}\{\Sigma/\Sigma'\} x) \Sigma, \Omega}{\outsem}
   \end{gathered}
}
\]
\vspace{-5mm}

\[
\infer[\dzname \lolli]
{\dz{\Gamma}{\Pi}{\Delta_a, \Delta_b}{\Xi}{\Gamma_1}{\Delta_1}{A \lolli B,
   \Omega}{\outsem}}
   {\mz{\Gamma}{\Delta_a}{A} & \dz{\Gamma}{\Pi}{\Delta_b}{\Xi, \Delta_a}
      {\Gamma_1}{\Delta_1}{B, \Omega}{\outsem}}
\]
}


\section{Low Level Dynamic Semantics}
The Low Level Abstract Machine~(LLAM) implements the complete operational
semantics of LM and builds on top of HLD by turning all the non-deterministic
choices of HLD into deterministic steps. LLAM specifies the following:

\begin{itemize}

   \item Rule matching by priority order;

   \item Deterministic matching of rule, comprehension and
      aggregate bodies;

   \item Iteration over all available combinations of facts for comprehensions/aggregates.

\end{itemize}

While HLD was represented as a proof tree with multiple choices, LLAM is defined
as a sequence of state transitions of the form $\trans{S_1}{S_2}$.  The LLAM
presents a complete step by step mechanism that is needed to correctly evaluate
an LM program. For instance, when LLAM tries to apply a rule, it checks if there
are enough facts in the database and backtrack until some rule can be inferred.

\paragraph{Continuation Stack} The core idea of LLAM is the \emph{continuation
stack}. A continuation stack contains \emph{continuation frames} that are
created for each predicate needed from the database. For instance, the first
rule in Fig.~\ref{code:visit} needs 3 frames: \code{visit}, \code{visited} and
\code{edge}. A frame allows the LLAM to search over facts of a given predicate
in order to match terms and thus contains candidate facts that will be
attempted. Each frame stores the contexts required to restart the matching
process. We have \emph{linear} and \emph{persistent frames}, which are created
for linear and persistent fact expressions, respectively.  We discuss only
linear frames, which have the form $\lframe{\Delta}{\Delta''}{p}{\Omega;
\Psi}{\Delta'}{\Omega'}$, where:

\begin{enumerate}

  \item[$\Delta$] multi-set of linear facts that are not of predicate $p$ plus
all the other $p$'s we have tried already, including the current $p$;

  \item[$\Delta''$] all the other $p$ facts we have not tried yet. It is a multi-set
  of linear facts;

  \item[$p$] fact expression that created this frame;

  \item[$\Omega$] ordered list of remaining terms needed to match past this
  frame;

  \item[$\Delta'$] multi-set of linear facts we have consumed to reach this
     point in the process;

  \item[$\Omega'$] terms matched up-to this point using $\Delta'$ and $\Gamma$. 
The frame proposition $\mz{\Gamma}{\Delta'}{\Omega'}$ represents a valid HLD
matching proposition. This will be used during the soundness proof in the next
section.

   \item[$\Psi$] current variable assignments (includes variable and value).
     
\end{enumerate}

\paragraph{Matching} For a rule of the form $\forall_{\widehat{x}}. A \lolli B$,
matching starts with an empty continuation stack and term $A$ has a term to
match that contains the (now) free variables $\widehat{x}$ that need to be
matched, as show in the following LLAM state:

\vspace{-3mm}
\[
   \matstate{A \lolli B}{(\Delta; \Phi)}{\cdot}{\Gamma}{\Delta}{A}{\cdot
   \rightarrow \one}
\]
\vspace{-3mm}

No shown in the matching state is the context $\Psi$ that maps variables to
values. At the start of matching, the $\widehat{x}$ variables are set as
\emph{undefined}. The terms to match are then deconstructed in an ordered
context and continuation frames are pushed onto the stack whenever a new fact
appears in the body. New facts also update the variables in the $\Psi$ contexts
by assigning them concrete values. Like the frames shown before, each matching
state also contains a proposition $\mz{\Gamma}{\Delta'}{\Omega'}$ that is an HLD
proof built by de-constructing and matching terms.

The following transitions model linear expression matching. $p_1, \Delta'' \prec
p$ means that the database facts $p_1, \Delta''$ satisfy the constraints of
$p$\footnote{Constraints such as variable matchings using the current $\Psi$
context.} by using the omitted variable context $\Psi$. The context $\Delta''$
is pushed into the new continuation frame because it is the set of candidate
facts to try next.

\vspace{-5mm}
\input{lld/match-p}
\vspace{-5mm}

LLAM also uses a special \emph{rule continuation stack} represented as $\rulestk
= (\Delta; \Phi)$ that tries all the rules in order. This fulfills the semantics
of LM for deriving the highest priority rule. In terms of inputs and outputs,
LLAM is equal to HLD.

\paragraph{Continuation} If, during the matching process, the machine is unable
to retrieve candidate facts for a given fact using $\Psi$, it backtracks to try
other facts in order to use different variable assignments in $\Psi$. The top
frame of the continuation stack is updated to retrieve the next candidate fact
and then restore the matching process using the frame's contexts.

The following two transitions model LLAM continuation states. In the first
transition, the next candidate fact is retrieved from the linear frame and the
linear frame is updated. In the second transition, since there is no candidate,
the frame is thrown away and the next frame is used.

\vspace{-5mm}
\input{lld/cont-p}
\vspace{-5mm}

\paragraph{Derivation}

Once matching completes, the terms are derived by de-constructing the head terms
into an ordered context. Next, the terms are derived sequentially, including
comprehensions and aggregates. The following rule shows the transition from a
derivation state to the final state of the machine:

\vspace{-5mm}
\input{lld/der-done}
\vspace{-5mm}

The contexts $\Gamma_1$ and $\Delta_1$ contain facts that were derived from head
terms.

\paragraph{Aggregates}

Both aggregates and comprehensions use the same matching and continuation
mechanism shown before.  Consider an aggregate of the form
$\aggregate{\mathtt{cons ::}}{\sigma}{\widehat{x}}{A}{B}{C}$. We start with an
empty continuation stack and match $A$.  The LLAM transition for initializing
the computation of aggregates is as follows:

\vspace{-5mm}
\input{lld/der-agg}
\vspace{-4mm}

If matching is successful then the first application of the aggregate is
completed and $B$ is derived for that particular combination.  At this point,
the aggregate value is saved in the judgment to be aggregated later with all the
remaining values\footnote{Since the aggregate references an accumulator variable
$\sigma$, we use $\Psi(\sigma)$ to retrieve the value.}.

The semantics of aggregates require all the combinations of $A$ from
the database. We reuse the continuation stack created by the first
application. All the frames after the first continuation frame for a linear fact
need to be removed and then the remaining frames need to be updated in order to
delete the consumed facts from the contexts. For example, if the body $A$ is
$\mathtt{b(X)} \otimes \mathtt{c(X)} \otimes \mathtt{b(Y)}$ and the continuation
stack has three frames (one per fact), we cannot backtrack to the frame of
$\mathtt{b(Y)}$ because, at that point, the matching process was assuming that
the previous $\mathtt{c(X)}$ linear fact was still available.  Moreover, we also
need to remove the consumed linear facts from the $\mathtt{b(X)}$ frame
since the candidate context may contain a $\mathtt{b}$ fact that was deleted
for $\mathtt{b(Y)}$. Note that we use two separate continuations in order to
mark the offset in the stack where the first linear continuation frame was
pushed.

The aggregate computation continues by restarting the matching process at the
top of the continuation stack and then matching the body of the aggregate once
again. This repeated process eventually iterates over the database. The
aggregate completes when the continuation stack is exhausted. The final
aggregate head $C$ is then derived with the aggregated value $\Sigma$ as shown
in the transition below:

\vspace{-5mm}
\input{lld/agg-cont-end}
\vspace{-4mm}

Figure~\ref{fig:backtrack} shows how the aggregate in the pagerank code shown in
Fig.~\ref{code:pagerank} is computed. Let's assume that \code{A = 1} and that
there are four facts: \code{edge(1,2)}, \code{edge(1,3)},
\code{pagerank(2,0.5)}, \code{pagerank(3,0.5)}. An initial frame is created for
\code{edge(1,B)} which includes two \code{edge} facts and \code{edge(1,2)} is
selected to continue the process.  Since \code{B = 2}, the frame for
\code{pagerank} includes only \code{pagerank(2,0.5)} which completes the first
application of the aggregate and every linear fact used is re-derived using the
first aggregate head. Computation then proceeds by backtracking to the first
linear frame, namely, the frame of \code{edge(1,B)} and the \code{edge(1,3)} is
selected for matching. A frame for \code{pagerank(3,V)} is created and the
second application of the aggregate re-derives the head. For the third
application, we backtrack again to the first linear frame, but there are no
available candidates and no more frames and thus the second aggregate head is
derived using \code{V = 0.5 + 0.5} as the result, completing the aggregate.

\begin{figure}[ht]
   \begin{center}
      \includegraphics[width=0.7\linewidth]{figures/backtrack.pdf}
   \end{center}
   \caption{Generating the pagerank aggregate.}
   \label{fig:backtrack}
   \vspace{-5mm}
\end{figure}


\section{Soundness Proof}
%The soundness proof proves that if a rule is successfully derived using the
LLAM, then it can also be derived in HLD. Since HLD semantics are so close to
linear logic, we prove that LM has a determined, correct, proof search behavior
when executing programs. However, the completeness theorem cannot be proven
since LLAM lacks the non-determinism present in HLD.

\paragraph{Initial Definitions} We first define equality between two multi-sets
of terms. Two multi-sets $A$ and $B$ are equal, $\feq{A}{B}$, when they have the
same constituent atoms\footnote{Also extends to atoms with variables, in which
there is a valid substitution to make them equal.}. This definition will be
useful to define what it means for a frame and state to be well-formed.

{\scriptsize
\[
\infer[\equiv p]{\feq{p, A}{q, B}}{\feq{A\theta_1}{B\theta_2} &
p\theta_1 \triangleq q\theta_2}
\tab
\infer[\equiv \bang p]{\feq{\bang p, A}{\bang q, B}}{\feq{A\theta_1}{B\theta_2}
& \bang p \theta_1 \triangleq \bang q \theta_2}
\tab
\infer[\equiv \one~L]{\feq{\one, A}{B}}{\feq{A}{B}}
\tab
\infer[\equiv \one~R]{\feq{A}{\one, B}}{\feq{A}{B}}
\]

\[
\infer[\equiv \cdot]{\feq{\cdot}{\cdot}}{}
\tab
\infer[\equiv \otimes~L]{\feq{A \otimes B, C}{D}}{\feq{A, B, C}{D}}
\tab
\infer[\equiv \otimes~R]{\feq{A}{B \otimes C, D}}{\feq{A}{B, C, D}}
\]
\vspace{-5mm}
}

\iffalse
\begin{theorem}[Match equivalence]
If two multi-sets are equivalent, $\feq{A_1, \dotsc, A_n}{B_1, \dotsc, B_m}$,
  and we can match $A_1 \otimes \dotsb \otimes A_n$ in HLD such that
  $\mz{\Gamma}{\Delta}{A_1 \otimes \dotsb \otimes A_n}$ then
     $\mz{\Gamma}{\Delta}{B_1 \otimes \dotsb \otimes B_m}$ is also true.
\end{theorem}
\begin{proof}
By straightforward induction on the first assumption.
\end{proof}
\fi

\begin{definition}[Well-formed frame]
  Consider a triplet $A; \Gamma; \Delta_{N}$ where $A$ is a term, $\Gamma$ is a
  set of persistent resources and $\Delta_{N}$ a multi-set of linear
  resources. A linear frame $f = \lframe{\Delta,
        p_1}{\Delta''}{p}{\Omega}{\Delta'}{\Omega'}$ is well-formed iff:

\begin{enumerate}
  \item $\feq{p, \Omega, \Omega'}{A}$ (the remaining terms and 
           matched terms are equivalent to body $A$);
  \item $\Delta' = \Delta'_1, \dotsc, \Delta'_n$ and $\Omega' =
  \Omega'_1 \otimes \dotsb \otimes \Omega'_n$
  \item $\Delta, \Delta'', \Delta, p_1 = \Delta_{N}$ (available facts, candidate
        facts for $p$, consumed facts and the linear fact used for $p$,
        respectively, are the same as the initial $\Delta_{N}$);
  \item $\mz{\Gamma}{\Delta'}{\Omega'}$ is valid.
\end{enumerate}
\end{definition}

The well-formedness of a continuation stack follows the above definition for all
frames. Moreover, each adjacent continuation frame is related in the sense that
$\Omega'$ of the first frame is always smaller than the $\Omega'$ of the next frame.
Identical relations apply to $\Delta'$ (increases) and $\Omega$ (decreases).

We also need to define what it means for a LLAM state to be well-formed. For
matching states in particular, we need to ensure that the proof term
$\mz{\Gamma}{\Delta}{\Omega}$ is related to the term being matched. We show
the body match definition below.

\begin{definition}[Well-formed body match]
  $\matstate{A \lolli B}{\rulestk}{\lstack{C}}{\Gamma}{\Delta}{\Omega}{\Delta'
  \rightarrow \Omega'}$ is well-formed in relation to a triplet $A; \Gamma;
  \Delta_{N}$ iff:

  \begin{itemize}
     \item $\Delta, \Delta' = \Delta_{N}$ and $\feq{\Omega', \Omega}{A}$ and
        $\mz{\Gamma}{\Delta'}{\Omega'}$ is valid;
     \item $\lstack{C}$ is well-formed in relation to $A; \Gamma; \Delta_{N}$
        and if:
     \begin{itemize}
        \item $\lstack{C} = \cdot$ then $\feq{\Omega}{A}$.

        \item $\lstack{C} = \lframe{\Delta_a,
           p_1}{\Delta_b}{p}{\Omega_a}{\Delta_c}{\Omega_b}, \lstack{C'}$ then
           $\feq{\Omega_a}{\Omega}$ and $\Delta = \Delta_a, \Delta_b$ and
           $\Delta' = \Delta_c, p_1$

      \end{itemize}
   \end{itemize}
\end{definition}

\paragraph{Soundness of Matching} To prove the soundness of matching, we want to
reconstitute a valid match $\mz{\Gamma}{\Delta}{A}$ in HLD from the steps of
LLAM. To accomplish this, we need to prove that each step preserves 
state well-formedness.

\begin{theorem}[Transitions preserve well-formedness]
Given a rule $A \lolli B$, consider a triplet $T = A; \Gamma; \Delta_{N}$.
If a state $s_1$ is well-formed in relation to $T$ and $\trans{s_1}{s_2}$ then
$s_2$ is also well-formed.
\end{theorem}
\begin{proof}
Use case by case analysis. Straightforward use of the well-formedness
assumptions of $s_1$ and term equivalence rules.
\end{proof}

Now, we  prove the lemma that reasons about the progression of matching states.
We use the function $split(\Omega)$ that is defined as $times(flatten(\Omega))$
which de-constructs the atomic terms $A_i$ into an expression of the form $A_1
\otimes \cdots \otimes A_n$. We assume that $\feq{\Omega}{split(\Omega)}$.

\begin{lemma}[Body match result]\label{thm:body_match}
  Given a rule $A \lolli B$, consider a triplet $T = A; \Gamma; \Delta_{N}$ and
  a context $\Delta_{N} = \Delta_1, \Delta_2, \Xi$.  If $s_1 = \matstate{A
  \lolli B}{\rulestk}{\lstack{C}}{\Gamma}{\Delta_1, \Delta_2}{\Omega}{\Delta'
  \rightarrow \Omega'}$ is well-formed in relation to $T$ and
  $\transs{s_1}{s_2}$ then:

  \begin{itemize}
     \item Match succeeds with no backtracking: $s_2 = \matstate{A \lolli B}{\rulestk}{\lstack{C''},
           \lstack{C}}{\Gamma}{\Delta_1}{\cdot}{\Delta', \Delta_2 \rightarrow
              \Omega' \otimes split(\Omega)}$

     
\item Match fails: $s_2 = \contstate{A \lolli B}{\rulestk}{\cdot}{\Gamma}$

\item Match succeeds with backtracking to a linear frame:
\begin{itemize}
   \item $s_2 = \matstate{A \lolli B}{\rulestk}{\lstack{C'''}, f',
      \lstack{C''}}{\Gamma}{\Delta_c}{\cdot}{\Delta'_f, p_2, \Delta''_f \rightarrow
      \Omega_f \otimes p \otimes split(\Omega'_f)}$

   \item $\lstack{C} = \lstack{C'}, f, \lstack{C''}$

   \item $f = \lframe{\Delta_a}{\Delta_{b_1}, p_2, \Delta_{b_2}}{p}{\Omega_f}{\Delta_f'}{\Omega_f'}$
   turns into $f' = \lframe{\Delta_a, \Delta_{b_1},
      p_2}{\Delta_{b_2}}{p}{\Omega_f}{\Delta'_f}{\Omega'_f}$

\end{itemize}


  \end{itemize}

  If $s_1 = \contstate{A \lolli B}{\rulestk}{\lstack{C}}{\Gamma}$
  is well-formed in relation to $T$ and $\transs{s_1}{s_2}$ then either:

  \begin{itemize}
     
\item Match fails: $s_2 = \contstate{A \lolli B}{\rulestk}{\cdot}{\Gamma}$

\item Match succeeds with backtracking to a linear frame:
\begin{itemize}
   \item $s_2 = \matstate{A \lolli B}{\rulestk}{\lstack{C'''}, f',
      \lstack{C''}}{\Gamma}{\Delta_c}{\cdot}{\Delta'_f, p_2, \Delta''_f \rightarrow
      \Omega_f \otimes p \otimes split(\Omega'_f)}$

   \item $\lstack{C} = \lstack{C'}, f, \lstack{C''}$

   \item $f = \lframe{\Delta_a}{\Delta_{b_1}, p_2, \Delta_{b_2}}{p}{\Omega_f}{\Delta_f'}{\Omega_f'}$
   turns into $f' = \lframe{\Delta_a, \Delta_{b_1},
      p_2}{\Delta_{b_2}}{p}{\Omega_f}{\Delta'_f}{\Omega'_f}$

\end{itemize}

  \end{itemize}
\end{lemma}
\begin{proof}
   Proof by mutual lexicographic induction on the state transitions. First on the size
   of $\Omega$, then on the length of the remaining candidate facts of the
   continuation frame and then on the size of the stack $\lstack{C}$.
\end{proof}

For an initial matching state $s_1 = \matstate{A \lolli
B}{\rulestk}{\cdot}{\Gamma}{\Delta_1, \Delta_2}{A}{\cdot \rightarrow \one}$, we
know that it may transition to $s_2 = \matstate{A \lolli
B}{\rulestk}{\lstack{C'}}{\Gamma}{\Delta_1}{\cdot}{\Delta_2 \rightarrow
split(\Omega)}$. The proof $\mz{\Gamma}{\Delta_2}{split(\Omega)}$ constructed using the
steps of the LLAM is used to conclude that $\mz{\Gamma}{\Delta_2}{A}$ since
$\feq{split(\Omega)}{A}$ from the well-formedness of the machine state and by an
auxiliary theorem which proves that equivalent terms are matched in HLD using
the same facts.

\paragraph{Aggregate Soundness} Proving that an LLAM derivation is sound is
trivial except for the case of comprehensions and aggregates.  The idea is to
use lemma~\ref{thm:body_match} for the case of aggregates and use the result to
prove that the continuation stack is updated for the next iteration of the
aggregate. We sketch the proof below.

\begin{theorem}[All aggregates]\label{thm:aggregates}

Consider an aggregate $\m{agg}$, where $\Pi(\m{agg}) = \forall_{\widehat{v}, \Sigma'}.
  (\defstwo{agg}{\widehat{v}}{\Sigma'} \lolli ((\lambda x. C x)\Sigma' \with (\forall_{\widehat{x}, \sigma}.
                                               ((A \lolli B) \otimes
                                                \defstwo{agg}{\widehat{v}}{\sigma
                                                ::\Sigma'}))))$,
and a triplet $T = A; \Gamma; \Delta_{N}$.
Assume that there exists $n \geq 0$ applications of $\m{agg}$
where the $i_{th}$ application is related to the following contexts: 
  $\Delta_i$ (derived linear facts), $\Gamma_i$ (derived persistent facts),
  $\Xi_i$ (consumed linear facts),
  $V_i$ (aggregate value) and
  $\Psi_i$ (variable bindings).
The initial database context is thus $\Delta_N = \Delta, \Xi_1, \dotsc, \Xi_n$.
If $s_1 = \matstatea{\Delta, \Xi_1, \dotsc, \Xi_n}{\cdot;
  \cdot}{\Gamma}{\Delta, \Xi_1, \dotsc, \Xi_n}{A}{\cdot \rightarrow \one}{\cdot}$
  (well-formed in relation to $T$) then $\transs{s_1}{\derstatex{\Gamma}{\Delta}{\Xi, \Xi_1, \dotsc, \Xi_n}{\Gamma_{N1},                                              \Gamma_1, \dotsc, \Gamma_n}{\Delta_{N1}, \Delta_1, \dotsc,
        \Delta_n}{$\\$ (\lambda x.  C\{\Psi(\widehat{v})/\widehat{v}\} x) \Sigma,
           \Omega_N}}$ and for the $n$ aggregate applications we have:
\begin{itemize}[leftmargin=*]
     \item $n$ values $V_i$ ($\Sigma = V_n :: \dots :: V_1 :: \cdot$);
     \item $n$ soundness proofs for the $n$ aggregate matches as $\mz{\Gamma}{\Xi_i}{A}$;

     \item $n$ derivation implications for HLD that, for any $\Omega_x$ and
        $\outsem$:
  \end{itemize}
  \vspace{-2mm}
        {\small if $\dz{\Gamma}{\Pi}{\Delta, \Xi_{i+1}, \dotsc, \Xi_{n}}{\Xi, \Xi_1,
        \dotsc, \Xi_i}{\Gamma_{N1}, \Gamma_1, \dotsc, \Gamma_i}{\Delta_{N1},
           \Delta_1, \dotsc, \Delta_i}{\Omega_x}{\outsem}$ then \\
              $\dz{\Gamma}{\Pi}{\Delta, \Xi_{i+1}, \dotsc, \Xi_{n}}{\Xi, \Xi_1,
              \dotsc, \Xi_i}{\Gamma_{N1}, \Gamma_1, \dotsc, \Gamma_{i-1}}{\Delta_{N1},
              \Delta_1, \dotsc, \Delta_{i-1}}{B, \Omega_x}{\outsem}$}

\end{theorem}

\begin{proof}
Apply the corresponding body match lemma for the aggregate matching state
(similar to Lemma~\ref{thm:body_match}). If the matching process initially fails then
we have 0 aggregate applications, otherwise we have a new aggregate value $V_0$.
After updating the continuation stack, we use a generalized version of this
theorem that uses a matching state with a non-empty continuation stack
in order to retrieve the remaining conclusions for the $n-1$ aggregate applications.
\end{proof}

This theorem gives us the soundness for matching and deriving aggregates. We
use the $n$ applications of the aggregate to reconstruct the proof tree in HLD
using the rules $\dzname \m{agg}_1$, $\dzname \m{agg}_2$ and $\dzname \lolli$.
We use induction on the resulting state $s_2$ and then use the derivation
implication and the matching proof to derive the $n^{th}$ application of the
aggregate. The remaining applications are reconstructed in the same fashion
from $n-1$ to $0$ until the required HLD derivation proof is achieved.

To conclude the proof, we state the main soundness theorem that proves that a
starting LLAM state either executes a rule and builds the required HLD proof tree
or fails with no HLD proof tree.

\begin{theorem}[Soundness]\label{thm:soundness}
  If $s_1 = \dostate{\Delta_N}{\Phi}{\Gamma}{\Pi}$
  then either:
  
  \begin{itemize}
        \item $\transs{s_1}{\finalstate{\Xi}{\Gamma_{N1}}{\Delta_{N1}}}$
  and $\exists_{R \in \Phi}. \dozx{\Gamma}{\Delta_N}{\Pi}{R}{\Xi; \Gamma_{N1};
        \Delta_{N1}}$ (success)
  \item $\transs{s_1}{\failstate{\Gamma}{\Delta_N}}$ (failure)
     \end{itemize}
\end{theorem}

\begin{proof}
   Use induction on the size of the rule context $\Phi$. Apply Lemma~\ref{thm:body_match} to
   get a successful match and the required HLD proof or fail and use the
   induction hypothesis on the smaller rule context $\Phi'$.
\end{proof}


\section{Related Work}
Linear logic has been used in the past as a basis for logic-based programming
languages~\cite{Miller85anoverview}, including forwards-chaining and
backwards-chaining programming languages.
LinLog~\cite{Andreoli92logicprogramming} is a backwards-chaining programming
language that originated from the idea of using focused proofs as a basis for a
programming language.  Lolli, a programming language presented
in~\cite{Hodas94logicprogramming}, is based on a fragment of intuitionistic
linear logic and proves goals by lazily managing the context of linear resources
during backwards-chaining proof search.
LolliMon~\cite{Lopez:2005:MCL:1069774.1069778} is a concurrent linear logic
programming language that integrates both forward-chaining and
backwards-chaining search, where forward-chaining computations are encapsulated
inside a monad and are concurrent, while backwards-chaining search is done
sequentially.

Proof-theoretic methods have been used extensively for specifying the
operational semantics~\cite{Cervesato96efficientresource} and
compilation~\cite{DBLP:journals/corr/abs-1210-1653} aspects of logic programming
languages. However, the available literature for linear logic programming only
deals with backwards-chaining and the resource management issues that arise in
such paradigm.


\section{Conclusions}
In this paper, we described an operational semantics for a forward-chaining
linear logic programming language with support for comprehensions and
aggregates. We started with \fragment, a intuitionistic linear logic system,
which was expanded to HLD semantics, a proof system based on focusing with
forwards-chaining on the sequent calculus. The LLD semantics removed the
non-determinism of HLD to complete the full semantics of LM. LLD specifies in
detail how linear facts are managed when matching a rule, that is, by storing
choice points and backtracking when matching fails. This mechanism is essential
for deriving comprehensions and aggregates, which requires matching all the
combinations from the database.


\bibliographystyle{splncs03}
\bibliography{refs}

\appendix
\section{\fragment}

\[
\infer[\otimes R]
{\Psi ; \seqx{\Gamma}{\Delta, \Delta'}{A \otimes B}}
{\Psi ; \seqx{\Gamma}{\Delta}{A} & \Psi ; \seqx{\Gamma}{\Delta}{B}}
\tab
\infer[\otimes L]
{\Psi ;\seqx{\Gamma}{\Delta, A \otimes B}{C}}
{\Psi ; \seqx{\Gamma}{\Delta, A, B}{C}}
\]


\[
   \infer[\with L_1]
   {\Psi; \seqx{\Gamma}{\Delta, A \with B}{C}}
   {\Psi; \seqx{\Gamma}{\Delta, A}{C}}
   \tab
   \infer[\with L_2]
   {\Psi; \seqx{\Gamma}{\Delta, A \with B}{C}}
   {\Psi; \seqx{\Gamma}{\Delta, B}{C}}
   \tab
   \infer[\with R]
   {\Psi; \seqx{\Gamma}{\Delta}{A \with B}}
   {\Psi; \seqx{\Gamma}{\Delta}{A} & \Psi; \seqx{\Gamma}{\Delta}{B}}
\]

\[
\infer[\lolli R]
{\Psi ; \seqx{\Gamma}{\Delta}{A \lolli B}}
{\Psi ; \seqx{\Gamma}{\Delta, A}{B}}
\tab
\infer[\lolli L]
{\seqx{\Gamma}{\Delta, \Delta', A \lolli B}{C}}
{\Psi ; \seqx{\Gamma}{\Delta}{A} &
   \Psi ; \seqx{\Gamma}{\Delta', B}{C}}
\]

\[
\infer[\bang R]
{\Psi ; \seqx{\Gamma}{\cdot}{\bang A}}
{\Psi ; \seqx{\Gamma}{\cdot}{A}}
\tab
\infer[\bang L]
{\Psi ; \seqx{\Gamma}{\Delta, \bang A}{C}}
{\Psi ; \seqx{\Gamma, A}{\Delta}{C}}
\tab
\infer[\m{copy}]
{\Psi ; \seqx{\Gamma, A}{\Delta}{C}}
{\Psi ; \seqx{\Gamma, A}{\Delta, A}{C}}
\]

\[
\infer[\one R]
{\Psi ; \seqx{\Gamma}{\cdot}{\one}}
{}
\tab
\infer[\one L]
{\Psi ; \seqx{\Gamma}{\Delta, \one}{C}}
{\Psi ; \seqx{\Gamma}{\Delta}{C}}
\tab
\infer[id_A]
{\Psi ; \seqx{\Gamma}{A}{A}}
{}
\]

\[
\infer[\forall R]
{\Psi ; \seqx{\Gamma}{\Delta}{\forall_{n:\tau}. A}}
{\Psi, m:\tau ; \seqx{\Gamma}{\Delta}{A\{m/n\}}}
\tab
\infer[\forall L]
{\Psi ; \seqx{\Gamma}{\Delta, \forall_{n:\tau}. A}{C}}
{\Psi \vdash M : \tau & \Psi ; \seqx{\Gamma}{\Delta, A\{M/n\}}{C}}
\]
\[
\infer[cut_A]
{\Psi ; \seqx{\Gamma}{\Delta, \Delta'}{C}}
{\Psi ; \seqx{\Gamma}{\Delta}{A} & \Psi ; \seqx{\Gamma}{\Delta', A}{C}}
\tab
\infer[cut\bang_A]
{\Psi ; \seqx{\Gamma}{\Delta}{C}}
{\Psi ; \seqx{\Gamma}{\cdot}{A} & \Psi ; \seqx{\Gamma, A}{\Delta}{C}}
\]


\end{document}
