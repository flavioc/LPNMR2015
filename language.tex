Linear Meld (LM) is a logic programming language that offers a declarative and
structured way to manage state using linear logic. A program consists of a
database and a set of inference rules. The database is satured by applying
inference rules to the facts in the database. LM distinguishes between
\emph{linear facts}, which are retracted after an inference, and
\emph{persistent facts} which cannot be retracted and are preceded by a
\code{\bang} symbol.

Figure~\ref{code:visit} presents a LM program that visits all the nodes of a
graph. Given a graph $G = (E, V)$, the program starts with a database that has a
fact \code{edge(a, b)} for every edge $(a, b) \in E$ and a fact
\code{unvisited(a)} for every node $a \in G$. The idea of the program is to
initially visit node 1 (line~\ref{line:visit_axiom}) by applying the first rule
(lines 3-5) which retracts fact \code{visit(1)} and \code{unvisited(1)} and
derives fact \code{visited(1)} and a \code{visit(b)} for all edges $(1, b) \in
E$ using a comprehension (line~\ref{line:visit_comprehension}). The
comprehension has two parts: \code{edge(A, B)} represents the body and
\code{visit(B)} represents the head. Semantically, the comprehension is a rule
of the form \code{edge(A, B) $\lolli$ visit(B)} that is inferred as many times
as the database allows. In this case, all the edges of the node are deleted to
infer \code{visit}. The second rule of the program
(lines~\ref{line:visit_rule21}-\ref{line:visit_rule22}) is inferred when a node
has been visited already. For a connected graph $G$, all the \code{unvisited}
facts will be deleted and a fact \code{visited(a)} will be derived for every
node $a \in G$.

\begin{figure}[h]
\begin{Verbatim}[numbers=left,commandchars=\*\#\&,fontsize=\stuffsize,xleftmargin=\stuffleftmargin]
visit(1). // start at node 1.*label#line:visit_axiom&

visit(A), unvisited(A) // change from unvisited to visited.
   -o visited(A),
      {B | edge(A, B) | visit(B)}.*label#line:visit_comprehension&
 
visit(A), visited(A) // node already visited.*label#line:visit_rule21&
   -o visited(A).*label#line:visit_rule22&
\end{Verbatim}
\caption{LM program to perform a visit of all nodes in a connected graph.}
  \label{code:visit}
\vspace{-5mm}
\end{figure}

Figure~\ref{code:pagerank} presents a fragment of an asynchronous PageRank
program that sums all the pagerank values of neighbor nodes. The rule deletes
the old \code{pagerank} fact and then derives an aggregate. The first argument
contains \code{sum} (the aggregate operator) and $V$ (the aggregate value). The
second argument \code{edge(A, B), pagerank(B, V)} is the body of the aggregate
that is iterated over to derive \code{edge(A, B), pagerank(B, V)}. The values
$V_i$ of each combination of \code{edge} and \code{pagerank} are summed up and
instantiated as $V = \sum_i V_i$ in \code{pagerank(A, damping/PAGES + (1.0 -
damping) * V)} to compute the new pagerank as the sum of the pageranks of the
neighborhood.

\begin{figure}[h]
\begin{Verbatim}[numbers=left,fontsize=\stuffsize,xleftmargin=\stuffleftmargin]
pagerank(A, OldRank)
      -o [sum => V | B | edge(A, B), pagerank(B, V) |
            edge(A, B), pagerank(B, V) |
            pagerank(A, damp/P + (1.0 - damp) * V)].
\end{Verbatim}
   \caption{Inference rule to update the pagerank of a node.}
   \label{code:pagerank}
\vspace{-5mm}
\end{figure}

\paragraph{Operational Semantics} Each rule in LM has a defined priority that is
inferred from its position in the source file.  Rules at the beginning of the
file have higher priority. We consider all the new facts that have been not
considered yet to create a set of \emph{candidate rules}.  The set of candidate
rules is then applied (by priority) and updated as new facts are derived.  Rules
are inferred atomically and both comprehensions and aggregates cannot use facts
derived during the current rule.

LM also imposes restrictions on the inference rules by forcing that every fact
used in the body of a rule uses the same first argument. This allows the
language to be concurrent since the database of facts is partitioned (by the
first argument) into many small databases that can infer rules independently.
In the remainder of this paper, we ignore this detail and consider the database
as a whole.

\paragraph{Abstract Syntax} 

Table~\ref{tbl:ast} shows the abstract syntax for rules in LM. A LM program
$Prog$ consists of a set of derivation rules $\Sigma$ and a database $D$.
Comprehensions and aggregates only appear on the head of the rule and only use
simple fact expressions.

\noindent \begin{table}[h]
\vspace{-5mm}
\captionsetup[subfloat]{labelformat=empty}
\subfloat[]{
\stuffsize\begin{tabular}{ l l c l }
  Program & $Prog$ & $::=$ & $\Sigma, D$ \\
  Set Of Rules & $\Sigma$ & $::=$ & $\cdot \; | \; \Sigma, R$\\
  Database & $D$ & $::=$ & $\Gamma, \Delta$ \\
  Linear Database & $\Delta$ & $::=$ & $\cdot \; | \; \Delta, l(\hat{t})$ \\
  Persistent Database & $\Gamma$ & $::=$ & $\cdot \; | \; \Gamma, \bang p(\hat{t})$ \\
  Linear Fact & $L$ & $::=$ & $l(\hat{x})$\\
  Persistent Fact & $P$ & $::=$ & $\bang p(\hat{x})$\\
\end{tabular}
} \subfloat[]{
\stuffsize\begin{tabular}{l l c l}
  Rule & $R$ & $::=$ & $BE \lolli HE \; | \; \forall_{x}. R$ \\
  Body & $BE$ & $::=$ & $L \; | \; P \; | \; BE, BE \; | \; \one$\\
  Head & $HE$ & $::=$ & $L \; | \; P \; | \; HE, HE \; | \; CE \; | \; AE \; | \; \one$\\
  
  Comprehension & $CE$ & $::=$ & $\comprehension{\widehat{x}}{SB}{SH}$ \\
  Aggregate & $AE$ & $::=$ & $\aggregate{\mathtt{Op}}{y}{\widehat{x}}{SB}{SH_1}{SH_2}$ \\
  
  Sub-Body & $SB$ & $::=$ & $L \; | \; P \; | \; SB, SB$ \\
  Sub-Head & $SH$ & $::=$ & $L \; | \; P \; | \; SH, SH \; | \; \one$\\
\end{tabular}
}
\caption{Abstract syntax for a fragment of LM.}\label{tbl:ast}
\vspace{-7mm}
\end{table}
