Linear Meld (LM) is a logic programming language that offers a declarative and
structured way to manage state using linear logic. A program consists of a
database and a set of inference rules. The database is satured by applying
inference rules to the facts in the database. Since LM is based on linear logic
facts are consumed when used in inference rules.

Fig.~\ref{code:visit} presents a LM program that visits all the nodes of a
graph. Given a graph $G = (E, V)$, the program starts with a database that has a
fact \texttt{edge(a, b)} for every edge $(a, b) \in E$ and a fact
\texttt{unvisited(a)} for every node $a \in G$. The idea of the program to
initially visit node 1 (line 1) by applying the first rule (lines 3-5) which
deletes fact \texttt{visit(1)} and \texttt{unvisited(1)} and derives fact
\texttt{visited(1)} and a \texttt{visit(b)} for all edges $(1, b) \in E$ using a
comprehension (line 5). The comprehension has two parts: \texttt{edge(A, B)}
represents the body and \texttt{visit(B)} represents the head. Semantically, the
comprehension is a rule of the form \texttt{edge(A, B) $\lolli$ visit(B)} that
is inferred as many times as the database allows. In this case, all the edges of
the node are deleted to infer a new \texttt{visit} fact.  The second rule of the
program (lines 7-8) is inferred when a node has been visited already.  For a
connected graph $G$, all the \texttt{unvisited} facts will be deleted and a fact
\texttt{visited(a)} will be derived for every node $a \in G$.


\begin{figure}[h]
\stuffsize\begin{Verbatim}[numbers=left]
visit(1). // start at node 1.

visit(A), unvisited(A) // change from unvisited to visited.
   -o visited(A),
      {B | edge(A, B) | visit(B)}.
 
visit(A), visited(A) // node already visited.
   -o visited(A).
\end{Verbatim}
\caption{LM program to perform a visit of all nodes in a connected graph.}
  \label{code:visit}
\end{figure}

Fig.~\ref{code:pagerank} presents a fragment of an asynchronous PageRank
program that sums all the pagerank values of neighbor nodes. The rule deletes
the old \texttt{pagerank} fact and then derives an aggregate. The first
argument contains \texttt{sum} (the aggregate operator) and $V$ (the
aggregate value). The second argument \texttt{edge(A, B), pagerank(B, V)}
is the body of the aggregate that is iterated over to derive \texttt{edge(A, B),
pagerank(B, V)}. The values $V_i$ of each combination of \texttt{edge}
and \texttt{pagerank} are summed up and instantiated as $V = \sum_i V_i$ in
\texttt{pagerank(A, damping/PAGES + (1.0 - damping) * V)} in order to compute
the new pagerank values that is the sum of the neighborhood.

\begin{figure}[h]
\stuffsize\begin{Verbatim}[numbers=left]
pagerank(A, OldRank)
      -o [sum => V | B | edge(A, B), pagerank(B, V) |
            edge(A, B), pagerank(B, V) |
            pagerank(A, damp/P + (1.0 - damp) * V)].
\end{Verbatim}
\caption{Inference rule to update the pagerank of a node.}
  \label{code:pagerank}
\end{figure}

\paragraph{Operational Semantics} Each rule in LM has a defined priority that is
inferred from its position in the source file.  Rules at the beginning of the
file have higher priority. We consider all the new facts that have been not
considered yet to create a set of \emph{candidate rules}.  The set of candidate
rules is then applied (by priority) and updated as new facts are derived.  Rules
are inferred atomically and both comprehensions and aggregates cannot use facts
derived for the current rule, only facts derived in the past.

\paragraph{Abstract Syntax}

Table~\ref{tbl:ast} shows the abstract syntax for rules in LM.
A LM program $Prog$ consists of a set of derivation rules $\Sigma$ and a database $D$.
Comprehensions and aggregates only appear on the head of the rule and such
constructs only use simple facts.

%\renewcommand{\arraystretch}{1.5}
\noindent \begin{table}[!htb]
\captionsetup[subfloat]{labelformat=empty}
\subfloat[]{
\scriptsize\begin{tabular}{ l l c l }
  Program & $Prog$ & $::=$ & $\Sigma, D$ \\
  Set Of Rules & $\Sigma$ & $::=$ & $\cdot \; | \; \Sigma, R$\\
  Database & $D$ & $::=$ & $\Gamma, \Delta$ \\
  Linear Database & $\Delta$ & $::=$ & $\cdot \; | \; \Delta, l(\hat{t})$ \\
  Persistent Database & $\Gamma$ & $::=$ & $\cdot \; | \; \Gamma, \bang p(\hat{t})$ \\
  Linear Fact & $L$ & $::=$ & $l(\hat{x})$\\
  Persistent Fact & $P$ & $::=$ & $\bang p(\hat{x})$\\
\end{tabular}
} \subfloat[]{
\scriptsize\begin{tabular}{l l c l}
  Rule & $R$ & $::=$ & $BE \lolli HE \; | \; \forall_{x}. R$ \\
  Body & $BE$ & $::=$ & $L \; | \; P \; | \; BE, BE \; | \; \one$\\
  Head & $HE$ & $::=$ & $L \; | \; P \; | \; HE, HE \; | \; CE \; | \; AE \; | \; \one$\\
  
  Comprehension & $CE$ & $::=$ & $\comprehension{\widehat{x}}{SB}{SH}$ \\
  Aggregate & $AE$ & $::=$ & $\aggregate{\mathtt{Op}}{y}{\widehat{x}}{SB}{SH_1}{SH_2}$ \\
  
  Sub-Body & $SB$ & $::=$ & $L \; | \; P \; | \; SB, SB$ \\
  Sub-Head & $SH$ & $::=$ & $L \; | \; P \; | \; SH, SH \; | \; \one$\\
\end{tabular}
}
\vspace{-20pt}
\caption{Abstract syntax for a fragment of LM.}\label{tbl:ast}
\end{table}
%\renewcommand{\arraystretch}{1.0}
