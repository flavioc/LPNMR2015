The Low Level Abstract Machine~(LLAM) implements the complete operational
semantics of LM and builds on top of HLD by turning all the non-deterministic
choices of HLD into deterministic steps. LLAM specifies the following:

\begin{itemize}
   \item Rule matching by priority order;

   \item Deterministic matching of rule bodies and bodies of comprehensions and
      aggregates;

   \item Applications of as many comprehensions and aggregates as the database
      allows. Aggregates are applied depending on the number of available
      combinations in the database.

\end{itemize}

HLD had many different proof trees for a given triplet $\Gamma; \Delta; \Phi$
because HLD allows choices to be made during the inference rules. For instance,
in HLD any rule could be selected to be executed. LLAM is defined as a
deterministic sequence of computation steps and there is no guessing involved.
LLAM semantics present a complete step by step mechanism that is needed to
correctly evaluate a LM program. For instance, when LLAM tries to apply a rule,
it will check if there are enough facts in the database and backtrack until a
rule can be inferred.

\paragraph{Continuation Stack} The core idea of LLAM is the \emph{continuation
stack}. A continuation stack contains \emph{continuation frames} that are
created for each predicate needed from the database. For instance, the first
rule in Fig.~\ref{code:visit} needs 3 frames: \texttt{visit}, \texttt{visited}
and \texttt{edge}. A frame allows the LLAM to search over the facts of a given
predicate in order to match bodies of rules and thus contains candidate facts
that will be attempted. Each frame stores the contexts required to restart the
matching process when it was created. A frame has the form $(\Delta; \Delta';
\Xi; p; \Omega; \Lambda; \Upsilon; \Psi)$, where:

\begin{enumerate}

   \item[$\Delta$] multi-set of linear facts that are not of type $p$ plus all
   the other $p$'s we have already tried, including the current $p$;

   \item[$\Delta'$] all the other $p$'s we haven't tried yet. It is a multi-set
   of linear facts;

   \item[$\Xi$] multi-set of linear facts we have consumed to reach this point;

   \item[$p$] current fact expression that originated this choice point;

   \item[$\Omega$] ordered list of remaining terms needed to match past this
   choice point;

   \item[$\Lambda$] multi-set of linear fact expressions that we have matched to
   reach this choice point. All the linear facts that match these terms are
   located in $\Xi$;

   \item[$\Upsilon$] multi-set of persistent fact expressions that we matched up
   to this point;

   \item[$\Psi$] current variable assignments (includes variable and value).

\end{enumerate}

LLD also uses a special \emph{rule continuation stack} that tries all the rules
in order. This fulfills the semantics of LM for deriving the highest priority
rule. In terms of inputs and outputs, LLD is equal to HLD.

\paragraph{Matching} Matching starts with an empty continuation stack. The terms
to match are deconstructed in an ordered context and continuation frames are
pushed into the stack whenever a new fact appears in the body. In the rule
below, $p_1, \Delta'' \prec p$ means that the database facts $p_1, \Delta''$
fullfill the constraints of $p$\footnote{Constraints such as variable
matchings.} and thus $\Delta''$ is pushed into the new continuation frame
because it is the set of candidate facts to try next.

\vspace{-10pt}
{\stuffsize
\[
\infer[\mo p~\m{on}~q]
{\mo \Psi ; \Gamma ; \Delta, p_1, \Delta'' ; \Xi; p, \Omega; H; f, \lstack{C};
   \lstack{R} \rightarrow \outsem}
{
\begin{gathered}
   p_1, \Delta'' \prec p \\
   f = (\Delta_{old}; \Delta'_{old}; q; \Omega_{old}; \Xi_{old}; \Lambda;
         \Upsilon; \Psi_{old}) \\
   f' = (\Delta, p_1; \Delta''; p; \Omega; \Xi; q, \Lambda; \Upsilon; \Psi) \\
   \mo \Psi ; \Gamma ; \Delta, \Delta''; \Xi, p_1; \Omega; H; f', f, \lstack{C}; \lstack{R} \rightarrow \outsem
\end{gathered}
}
\]
}
\vspace{-15pt}

Matching finishes when there is no more terms to match.

\paragraph{Continuation} If, during the matching process, the mechanism is
unable to retrieve candidate facts for a given fact, we need to backtrack to try
other combinations. The top frame of the continuation stack is updated to
retrieve the next candidate fact and then restore the matching process using the
frame's contexts.

\vspace{-10pt}
{\stuffsize
\[
\infer[\cont p~\m{next}]
{\cont (\Delta; p_1, \Delta''; p, \Omega; \Xi; \Lambda; \Upsilon; \Psi), \lstack{C};
   H; \lstack{R}; \Gamma \rightarrow \outsem}
{\mo \Psi; \Gamma ; \Delta, \Delta''; \Xi, p_1; \Omega; H; (\Delta, p_1; \Delta''; p,
      \Omega; H; \Xi; \Lambda; \Upsilon; \Psi), \lstack{C}; \lstack{R} \rightarrow \outsem}
\]
}
\vspace{-20pt}

\paragraph{Derivation}

Once matching completes, the terms are derived by deconstructing the head terms
into an ordered context. Next, the terms are derived sequentially, including
comprehensions and aggregates.

\paragraph{Aggregates}

Both aggregates and comprehensions use the same computation mechanism as before.
First, we start with an empty continuation stack and then the aggregate's body
$SB$ is matched. If matching is successful, then the first application of the
aggregate is completed and then the aggregate head $SH_1$ is derived for this
particular combination. At this point, the aggregate value is saved in the
judgment.\footnote{Since the aggregate references a variable $x$, we look into
   $\Psi$ to retrieve the value.}

However, the semantics of aggregates require all the combinations of $SB$ from
the database. For this, we reuse the continuation stack created by the first
application. All the frames after the first continuation frame for a linear fact
need to be removed and then the remaining frames need to be updated in order to
delete the consumed facts from the contexts. For example, if the body is
$\mathtt{b(X)} \otimes \mathtt{c(X)} \otimes \mathtt{b(Y)}$ and the continuation
stack has three frames (one per fact), we cannot backtrack to the frame of
$\mathtt{b(Y)}$ because, at that point, the matching process was assuming that
the previous $\mathtt{c(X)}$ linear fact was still available.  Moreover, we also
need to remove the consumed linear facts from the $\mathtt{b(X)}$ frame
since the candidate context may contain a $\mathtt{b}$ fact that was deleted
for $\mathtt{b(Y)}$.

Afterwards, the next application of the aggregate is done by restarting the
matching process at the top of the continuation stack. This repeated process
eventually iterates over the database since all candidate facts are matched.
The aggregate completes when the continuation stack is exhausted and then the
final aggregate head $SH_2$ is derived with the computed aggregated value.


