The Low Level Abstract Machine~(LLAM) implements the complete operational
semantics of LM and builds on top of HLD by turning all the non-deterministic
choices of HLD into deterministic steps. LLAM specifies the following:

\begin{itemize}

   \item Rule matching by priority order;

   \item Deterministic matching of rule bodies and bodies of comprehensions and
      aggregates;

   \item Applications of as many comprehensions and aggregates as the database
      allows. Aggregates are applied depending on the number of available
      combinations in the database.

\end{itemize}

LLAM is defined as a sequence of state transitions of the form $\trans{S_1}{S_2}$,
while HLD had many different proof trees for a given triplet $\Gamma; \Delta; \Phi$
since HLD allows choices to be made during the inference rules.
LLAM semantics present a complete step by step mechanism that is needed to
correctly evaluate a LM program. For instance, when LLAM tries to apply a rule,
it will check if there are enough facts in the database and backtrack until a
rule can be inferred.

\paragraph{Continuation Stack} The core idea of LLAM is the \emph{continuation
stack}. A continuation stack contains \emph{continuation frames} that are
created for each predicate needed from the database. For instance, the first
rule in Fig.~\ref{code:visit} needs 3 frames: \code{visit}, \code{visited} and
\code{edge}. A frame allows the LLAM to search over facts of a given
predicate in order to match terms and thus contains candidate facts
that will be attempted. Each frame stores the contexts required to restart the
matching process when it was created. We have \emph{linear} and \emph{persistent
frames}, which are created for linear and persistent fact expressions,
respectively. For brevity purposes we discuss only linear frames, which have the
form $\lframe{\Delta}{\Delta''}{p}{\Omega; \Psi}{\Delta'}{\Omega'}$, where:

\begin{enumerate}

  \item[$\Delta$] multi-set of linear facts that are not of predicate $p$ plus
all the other $p$'s we have tried already, including the current $p$;

  \item[$\Delta''$] all the other $p$' facts we haven't tried yet. It is a multi-set
  of linear facts;

  \item[$p$] fact expression that created this frame;

  \item[$\Omega$] ordered list of remaining terms needed to match past this
  frame;

  \item[$\Delta'$] multi-set of linear facts we have consumed to reach this point;

  \item[$\Omega'$] terms matched up-to this point using $\Delta'$ and $\Gamma$. 
The frame proposition $\mz{\Gamma}{\Delta'}{\Omega'}$ represent a valid HLD
matching proposition. This will be used during the soundness proof in the next
section.

   \item[$\Psi$] current variable assignments (includes variable and value).
     
\end{enumerate}

We also have a \emph{persistent frame} that is used for matching persistent fact
expressions. For brevity purposes, we will use only linear frames.

\paragraph{Matching} For a rule of the form $A \lolli B$, matching starts with
an empty continuation stack and term $A$ has a term to match that contains free
variables that need to be matched, as show in the following LLAM state:

\[
   \matstate{A \lolli B}{(\Delta; \Phi)}{\cdot}{\Gamma}{\Delta}{A}{\cdot
   \rightarrow \one}
\]

The terms to match are deconstructed in an ordered context and continuation
frames are pushed onto the stack whenever a new fact appears in the body. Like
the frames shown before, each matching state also contains the proposition
$\mz{\Gamma}{\Delta'}{\Gamma'}$ that is an HLD proposition built by
de-constructing and matching terms.

The following transitions model linear expression matching. $p_1, \Delta'' \prec
p$ means that the database facts $p_1, \Delta''$ satisfy the constraints of
$p$\footnote{Constraints such as variable matchings.} by using the omitted
variable context $\Psi$. The context $\Delta''$ is pushed into the new
continuation frame because it is the set of candidate facts to try next.

\[
\infer[\mo p~\m{first}]
{\mo \Gamma ; \Delta, p_1, \Delta'' ; \Xi; p, \Omega; H; \cdot; \lstack{R} \rightarrow
   \outsem}
{
 \begin{gathered}
   p_1, \Delta'' \prec p \\
   \mo \Gamma ; \Delta, \Delta''; \Xi, p_1; \Omega; H;
      (\Delta, p_1; \Delta''; p; \Omega; \Xi; \cdot; \cdot); \lstack{R} \rightarrow
      \outsem
  \end{gathered}
}
\]

\[
\infer[\mo p~\m{on}~q]
{\mo \Gamma ; \Delta, p_1, \Delta'' ; \Xi; p, \Omega; H; f, \lstack{C};
   \lstack{R} \rightarrow \outsem}
{
\begin{gathered}
   p_1, \Delta'' \prec p \\
   f = (\Delta_{old}; \Delta'_{old}; q; \Omega_{old}; \Xi_{old}; \Lambda;
         \Upsilon) \\
   \mo \Gamma ; \Delta, \Delta''; \Xi, p_1; \Omega; H; (\Delta, p_1; \Delta'';
         p; \Omega; \Xi; q, \Lambda; \Upsilon), f, \lstack{C}; \lstack{R} \rightarrow \outsem
\end{gathered}
}
\]


\[
\infer[\mo p~\m{on}~\bang q]
{\mo \Gamma ; \Delta, p_1, \Delta'' ; \Xi; p, \Omega; H; f, \lstack{C};
   \lstack{R} \rightarrow \outsem}
{
\begin{gathered}
   p_1, \Delta'' \prec p \\
   f = [\Gamma_{old}; \Delta_{old}; \bang q; \Omega_{old}; \Xi_{old}; \Lambda; \Upsilon] \\
   \mo \Gamma ; \Delta, \Delta''; \Xi, p_1; \Omega; H; (\Delta, p_1; \Delta''; p; \Omega; \Xi;
         \Lambda; q, \Upsilon), f, \lstack{C}; \lstack{R} \rightarrow \outsem
\end{gathered}}
\]

\[
\infer[\mo p~\m{fail}]
{\mo \Gamma ; \Delta; \Xi; p, \Omega; H; \lstack{C}; \lstack{R} \rightarrow \outsem}
{\Delta \npreceq p & \cont \lstack{C} ; H; \lstack{R}; \Gamma \rightarrow \outsem}
\]


LLAM also uses a special \emph{rule continuation stack} represented as $\rulestk
= (\Delta; \Phi)$ that tries all the rules in order. This fulfills the semantics
of LM for deriving the highest priority rule. In terms of inputs and outputs,
LLAM is equal to HLD.

\paragraph{Continuation} If, during the matching process, the machine is unable
to retrieve candidate facts for a given fact using $\Psi$, it backtracks to try
other facts in order to use different variable assignments in $\Psi$. The top
frame of the continuation stack is updated to retrieve the next candidate fact
and then restore the matching process using the frame's contexts.

The following two transitions model LLAM continuation states. In the first
transition, the next candidate fact is retrieved from the linear frame and the
linear frame is updated. In the second transition, since there is no available
candidate, the frame is thrown away and the next frame is inspected.

\[
\infer[\cont p~\m{next}]
{\cont (\Delta; p_1, \Delta''; p, \Omega; \Xi; \Lambda; \Upsilon), \lstack{C};
   H; \lstack{R}; \Gamma \rightarrow \outsem}
{\mo \Gamma ; \Delta, \Delta''; \Xi, p_1; \Omega; H; (\Delta, p_1; \Delta''; p,
      \Omega; H; \Xi; \Lambda; \Upsilon), \lstack{C}; \lstack{R} \rightarrow \outsem}
\]

\[
\infer[\cont p~\m{no~more}]
{\cont (\Delta; \cdot; p, \Omega; \Xi; \Lambda; \Upsilon), \lstack{C}; H;
   \lstack{R}; \Gamma
   \rightarrow \outsem}
{\cont \lstack{C}; H; \lstack{R}; \Gamma \rightarrow \outsem}
\]


\paragraph{Derivation}

Once matching completes, the terms are derived by deconstructing the head terms
into an ordered context. Next, the terms are derived sequentially, including
comprehensions and aggregates. The following rule shows the transition from a
derivation state to the final state of the machine:

\[
\infer[\done \m{end}]
{\done \Gamma; \Delta; \Xi; \Gamma_1; \Delta_1; \cdot \rightarrow \Xi; \Gamma_1;
\Delta_1}
{}
\]


The contexts $\Gamma_1$ and $\Delta_1$ contain facts that were derived using the
head terms, incluing aggregates and comprehensions.

\paragraph{Aggregates}

Both aggregates and comprehensions use the same matching and continuation
mechanism shown before.  Consider an aggregate of the form
$\aggregate{\mathtt{cons ::}}{\sigma}{\widehat{x}}{A}{B}{C}$. We start with an
empty continuation stack and match $A$.  The LLAM transition for initializing
the computation of aggregates is as follows:

\[
\infer[\done \m{agg}]
{\done \Gamma; \Delta ; \Xi; \Gamma_1; \Delta_1; \aggsz{A}{B}{C}, \Omega
   \rightarrow \outsem}
{\ma \Gamma; \Delta; \Xi; \Gamma_1; \Delta_1; \cdot; A ; \cdot; \cdot;
   \aggsz{A}{B}{C}; \Omega; \Delta; \cdot \rightarrow \outsem}
\]


If matching is successful then the first application of the aggregate is
completed and the aggregate head $B$ is derived for that particular combination.
At this point, the aggregate value is saved in the judgment to be aggregated
later with all the remaining values\footnote{Since the aggregate references an
accumulator variable $\sigma$, we use $\Psi(\sigma)$ to retrieve the value.}.

The semantics of aggregates require all the combinations of $A$ from
the database. We reuse the continuation stack created by the first
application. All the frames after the first continuation frame for a linear fact
need to be removed and then the remaining frames need to be updated in order to
delete the consumed facts from the contexts. For example, if the body $A$ is
$\mathtt{b(X)} \otimes \mathtt{c(X)} \otimes \mathtt{b(Y)}$ and the continuation
stack has three frames (one per fact), we cannot backtrack to the frame of
$\mathtt{b(Y)}$ because, at that point, the matching process was assuming that
the previous $\mathtt{c(X)}$ linear fact was still available.  Moreover, we also
need to remove the consumed linear facts from the $\mathtt{b(X)}$ frame
since the candidate context may contain a $\mathtt{b}$ fact that was deleted
for $\mathtt{b(Y)}$. Note that we use two separate continuations in order to
mark the offset in the stack where the first linear continuation frame was
pushed.

The aggregate computation continues by restarting the matching process at the
top of the continuation stack and then matching the body of the aggregate once
again. This repeated process eventually iterates over the database. The
aggregate completes when the continuation stack is exhausted. The final
aggregate head $C$ is then derived with the aggregated value $\Sigma$ as shown
in the transition below:

\[
\infer[\conta{\aggsz{A}{B}{C}} \m{end}]
{\conta{\aggsz{A}{B}{C}} \Gamma; \Delta_N; \Xi_N; \Gamma_{N1}; \Delta_{N1}; \cdot; \cdot;
   \Omega; \Sigma \rightarrow \outsem}
{\done \Gamma; \Delta_N; \Xi_N; \Gamma_{N1}; \Delta_{N1}; (\lambda x. C
      x)\Sigma,
   \Omega \rightarrow \outsem}
\]

