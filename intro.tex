
Datalog~\cite{Ramakrishnan93asurvey} is a forward-chaining logic programming
language originally designed for deductive databases. In Datalog, the program is
composed of a database of facts and a set of rules. Datalog programs first
populate the database with axioms and then saturate the database using rule
inference. In Datalog, logical facts are persistent and thus once a fact is
derived, it cannot be deleted. However, there has been a growing interest in
integrating linear logic~\cite{girard-87} into bottom-up logic
programming~\cite{Chang03ajudgmental,cruz-iclp14,Lopez:2005:MCL:1069774.1069778,simmons-lla},
allowing for both fact assertion and retraction.

Linear logic programming greatly increases the expressiveness of traditional
Datalog since it allows the programmer to manage state in a structured fashion.
However, such approach still lacks some facilities that are common in other
languages such as comprehensions and aggregates. Comprehensions have been
popular among functional programmers for years~\cite{npl1977} and aggregates
have found its way into
Datalog~\cite{Consens93lowcomplexity,Greco:1999:DPD:627321.627989} and also
functional programming in the form of
\emph{folds}~\cite{Hutton:1999:TUE:968578.968579}.

In this paper, we present LM~\cite{cruz-iclp14}, a linear logic programming
language that supports comprehensions, aggregates and rule priorities and is
suited for programming over graph data structures. LM is based on a small
fragment of intuitionistic linear logic and both comprehensions and aggregates
are encoded using persistent terms that allow the programmer to iterate over
combinations of facts. We present a high level dynamic semantics and then a low
level abstract machine using proof-theoretic methods. The high level semantics
are closely related to the sequent calculus since it amounts to focusing with
forward-chaining~\cite{Andreoli92logicprogramming,laurent2004proof}, where atoms
have a positive polarity.  The low level abstract machine is much closer to a
real implementation since it remove most of non-determinism of the high level
semantics and also describe in detail how the focused proof search mechanism
works, including backtracking and resource management. We also relate both
systems by proving that the abstract machine is sound in relation to the high
level dynamic semantics.

The contributions of this paper are 2-fold: (1) a low level abstract machine for
a forward-chaining linear logic programming language and (2) the connection to
its logic based on proof-theoretic methods, namely, the sequent calculus.  The
paper is organized as follows. In Section 2 we present a simple LM program.
Next, in Section 3 we present the fragment of linear logic used for LM. In
Sections 4 and 5 we present the high level dynamic semantics and the low level
abstract machine. The soundness proof is presented in
Section~\ref{sec:soundness} and then we conclude with related work and
conclusions.
