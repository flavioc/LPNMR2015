
%\section{Step}
%\[
\infer[\stepo]
{\begin{split}
\stepo [\Gamma_1, \dotsc, \Gamma_i, \dotsc, \Gamma_n] &; [\Delta_1, \dotsc,
   \Delta_i, \dotsc, \Delta_n];
   \Phi \\ \Longrightarrow& \\ [\Gamma_1, \Gamma'_1, \dotsc, \Gamma_i, \Gamma'_i,
   \dotsc,
   \Gamma_n, \Gamma'_n]; & [\Delta_1, \Delta'_1, \dotsc, (\Delta_i - \Xi'),
   \Delta'_i, \dotsc, \Delta_n, \Delta'_n]
\end{split}
}
{
   \doo \Gamma_i; \Delta_i; \Phi \rightarrow \Xi'; \Delta'_1, \dotsc, \Delta'_n;
   \Gamma'_1, \dotsc, \Gamma'_n
}
\]


\newcommand{\trans}[2]{#1 \;\; \mapsto \;\; #2}
\newcommand{\dostate}[3]{\mathtt{infer} \; #1; #2; #3}
\newcommand{\appstate}[4]{\mathtt{apply} \; #1; #2; #3; #4}
\newcommand{\rulestk}[0]{\lstack{R}}
\newcommand{\matstate}[7]{{\blacktriangleright}_{#1}^{#7} \; #2; #3; #4; #5; #6}
\newcommand{\lframe}[6]{\underset{#5 \rightarrow #6}{(#1; #2; #3; #4)}}
\newcommand{\pframe}[6]{\underset{#5 \rightarrow #6}{(#1; #2; #3; #4)}}
\newcommand{\contstate}[4]{\triangleleft_{#1} \; #2; #3; #4}
\newcommand{\failstate}[2]{\m{fail}_{#1; #2}}
\newcommand{\derstate}[4]{\curvearrowright_{#1} #2 ; #3; #4}
\newcommand{\finalstate}[3]{\circlearrowleft #1 ; #2; #3}

\subsection{Abstract Machine}
\subsubsection{States}

The available machine states are as follows:

\begin{description}
   \item[$\dostate{\Delta}{\Phi}{\Gamma}$] Executes the highest priority rule.
   \item[$\appstate{\Delta}{\Phi}{\Gamma}{R}$] Executes one rule.
   \item[$\matstate{A \lolli
      B}{\rulestk}{\lstack{C}}{\Gamma}{\Delta}{\Omega}{\Delta' \rightarrow
         \Omega'}$]
   Matches the body of the rule.
   \item[$\contstate{A \lolli B}{\rulestk}{\lstack{C}}{\Gamma}$] Restores
      matching to the next choice point of the machine.
   \item[$\derstate{\Xi}{\Gamma_1}{\Delta_1}{\Omega}$] Derives the head of the
   rule.
   \item[$\finalstate{\Xi}{\Gamma_1}{\Delta_1}$] Final state.
   \item[$\failstate{\Gamma}{\Delta}$] Fail state.
\end{description}

The machine starts with a database $(\Gamma; \Delta)$ and a queue of rules
$\Phi$. The initial state is always $\dostate{\Delta}{\Phi}{\Gamma}$.
We start by picking the first rule $R$ from $\Phi$:

\[
\trans{\dostate{\Delta}{R, \Phi}{\Gamma}}
   {\appstate{\Delta}{\Phi}{\Gamma}{R}}
\]
\[
\trans{\dostate{\Delta}{\cdot}{\Gamma}}
   {\failstate{\Gamma}{\Delta}}
\]
\[
\trans{\appstate{\Delta}{\Phi}{\Gamma}{A \lolli B}}
      {\matstate{A \lolli B}{(\Delta; \Phi)}{\cdot}{\Gamma}{\Delta}{A}{\cdot \rightarrow
                                                            \one}}
\]

\subsubsection{Matching}

Matching is the most crucial part of the machine since the derivation of the
head is trivial (except if we add comprehensions). For each atomic term in the
body we look for candidate facts in the database, add a new frame to the
continuation stack and then select the first candidate fact.

There are two types of continuation frames. Linear frames use the form
$\lframe{\Delta}{\Delta''}{p}{\Omega}{\Delta'}{\Omega}$, where:

\begin{description}
   \item[$\Delta$] Linear facts that are not candidate facts.
   \item[$\Delta''$] Candidate facts.
   \item[$p$] Linear fact to match.
   \item[$\Omega$] Rest of the body to match.
   \item[$\Delta'$] Facts consumed up-to this point.
   \item[$\Omega'$] Terms matched up-to this point using $\Delta'$.
\end{description}

Persistent frames are slightly different since they only need to keep track of
remaining persistent candidates. They are structured as $\pframe{\Gamma''}{\Delta}{\bang
   p}{\Omega}{\Delta'}{\Omega'}$:

\begin{description}
   \item[$\Gamma''$] Candidate facts.
   \item[$\Delta$] Remaining linear facts.
   \item[$\bang p$] Persistent fact to match.
   \item[$\Omega$] Rest of the body to match.
   \item[$\Delta'$] Facts consumed up-to this point.
   \item[$\Omega'$] Terms matched up-to this point using $\Delta'$.
\end{description}

A matching state is structured as $\matstate{A \lolli
   B}{\rulestk}{\lstack{C}}{\Gamma}{\Delta}{\Omega}{\Delta' \rightarrow
      \Omega'}$, where:

\begin{description}
   \item[$A \lolli B$] Rule being matched. $A$ is the body and $B$ the head.
   \item[$\rulestk$] Rule continuation. Contains the original $\Delta_N$ and the
   rest of the rules $\Phi$.
   \item[$\lstack{C}$] The continuation stack for matching body $A$.
   \item[$\Gamma$] persistent context.
   \item[$\Delta$] remaining linear context after matching up to this point.
   \item[$\Omega$] Rest of terms from $A$ to match.
   \item[$\Delta'$] Facts from the original $\Delta$ that were already matched
      ($\Delta', \Delta = \Delta_N$).
   \item[$\Omega'$] Parts of $A$ already matched. They are in the form $P_1
      \otimes \dotsb \otimes P_n$. The idea is to use term equivalence and the
      fact that $\feq{\Omega, \Omega'}{A}$ to justify $\mz \Gamma; \Delta'
      \rightarrow A$ when the matching process completes.
\end{description}

The first transition matches $p$ against $p_1, \Delta''$ from the database:

\[
\trans{\matstate{A \lolli B}{\rulestk}{\lstack{C}}{\Gamma}{\Delta, p_1, \Delta''}{p,
   \Omega}{\Delta' \rightarrow \Omega'}}
{\matstate{A \lolli B}{\rulestk}{\lframe{\Delta,
   p_1}{\Delta''}{p}{\Omega}{\Delta'}{\Omega'}, \lstack{C}}{\Gamma}{\Delta,
   \Delta''}{\Omega}{\Delta', p_1 \rightarrow \Omega' \otimes p}} \;\;\; (p_1, \Delta'' \prec p)
\]

The next matches $\bang p$ against $\bang p_1, \Gamma''$ from the database:

\[
\trans{\matstate{A \lolli B}{\rulestk}{\lstack{C}}{\Gamma}{\Delta}{\bang p,
   \Omega}{\Delta' \rightarrow \Omega'}}
{\matstate{A \lolli B}{\rulestk}{\pframe{\Gamma''}{\Delta}{\bang
   p}{\Omega}{\Delta'}{\Omega'}, \lstack{C}}{\Gamma, p_1,
      \Gamma''}{\Delta}{\Omega}{\Delta' \rightarrow \Omega' \otimes \bang p}}
      \;\;\; (\bang p_1, \Gamma'' \prec \bang p)
\]

If we do not have facts matching $p$, we must fail:

\[
\trans{\matstate{A \lolli B}{\rulestk}{\lstack{C}}{\Gamma}{\Delta}{p,
   \Omega}{\Delta' \rightarrow \Omega'}}
{\contstate{A \lolli B}{\rulestk}{\lstack{C}}{\Gamma}}
\]

Likewise, we also need to fail for $\bang p$:

\[
\trans{\matstate{A \lolli B}{\rulestk}{\lstack{C}}{\Gamma}{\Delta}{\bang p,
   \Omega}{\Delta' \rightarrow \Omega'}}
{\contstate{A \lolli B}{\rulestk}{\lstack{C}}{\Gamma}}
\]

The case for $\one$:

\[
\trans{\matstate{A \lolli B}{\rulestk}{\lstack{C}}{\Gamma}{\Delta}{\one,
   \Omega}{\Delta' \rightarrow \Omega'}}
{\matstate{A \lolli B}{\rulestk}{\lstack{C}}{\Gamma}{\Delta}{\Omega}{\Delta'
   \rightarrow \Omega'}}
\]

... and for $X \otimes Y$:

\[
\trans{\matstate{A \lolli B}{\rulestk}{\lstack{C}}{\Gamma}{\Delta}{X \otimes Y,
   \Omega}{\Delta' \rightarrow \Omega'}}
{\matstate{A \lolli B}{\rulestk}{\lstack{C}}{\Gamma}{\Delta}{X, Y,
   \Omega}{\Delta' \rightarrow \Omega;}}
\]

The following transition completes the matching process:

\[
\trans{\matstate{A \lolli
   B}{\rulestk}{\lstack{C}}{\Gamma}{\Delta}{\cdot}{\Delta' \rightarrow \Omega'}}
{
   \derstate{\Delta'}{\cdot}{\cdot}{B}
}
\]

\subsubsection{Backtracking}

The backtracking state of the machine reads the top of the continuation stack
$\lstack{C}$ and restores the matching process with a different candidate fact
from the continuation frame. The state is written as $\contstate{A \lolli
   B}{\rulestk}{\lstack{C}}{\Gamma}$:

\begin{description}
   \item[$A \lolli B$] The rule being matched.
   \item[$\rulestk$] The rule continuation.
   \item[$\lstack{C}$] The continuation stack for matching body $A$.
   \item[$\Gamma$] Persistent context.
\end{description}

\[
\trans{\contstate{A \lolli B}{(\Delta; \Phi)}{\cdot}{\Gamma}}
   {\dostate{\Delta}{\Phi}{\Gamma}}
\]

\[
\trans{\contstate{A \lolli B}{\rulestk}{\lframe{\Delta}{p_2,
   \Delta''}{p}{\Omega}{\Delta'}{\Omega'}, \lstack{C}}{\Gamma}}
{
   \matstate{A \lolli B}{\rulestk}{\lframe{\Delta,
      p_2}{\Delta''}{p}{\Omega}{\Delta'}{\Omega'},
   \lstack{C}}{\Gamma}{\Delta}{\Omega}{\Delta', p_2 \rightarrow \Omega' \otimes p}}
\]

\[
\trans{\contstate{A \lolli
   B}{\rulestk}{\lframe{\Delta}{\cdot}{p}{\Omega}{\Delta'}{\Omega'},
      \lstack{C}}{\Gamma}}
{
   \contstate{A \lolli B}{\rulestk}{\lstack{C}}{\Gamma}}
\]

\[
\trans{\contstate{A \lolli B}{\rulestk}{\pframe{\bang p_2,
   \Gamma''}{\Delta}{\bang p}{\Omega}{\Delta'}{\Omega'}, \lstack{C}}{\Gamma}}
{
   \matstate{A \lolli B}{\rulestk}{\pframe{\Gamma''}{\Delta}{\bang
      p}{\Omega}{\Delta'}{\Omega'}, \lstack{C}}{\Gamma}{\Delta}{\Omega}{\Delta'
         \rightarrow \Omega' \otimes \bang p_2}}
\]

\[
\trans{\contstate{A \lolli B}{\rulestk}{\pframe{\cdot}{\Delta}{\bang
   p}{\Omega}{\Delta'}{\Omega'}, \lstack{C}}{\Gamma}}
{
   \contstate{A \lolli B}{\rulestk}{\lstack{C}}{\Gamma}}
\]

\subsubsection{Derivation}

The derivation state simply iterates over $B$, the head of the rule, and derives
terms into the corresponding new contexts. $\Xi$ represents the facts consumed
by the matching process.

\[
\trans{\derstate{\Xi}{\Gamma_1}{\Delta_1}{p, \Omega}}
{\derstate{\Xi}{\Gamma_1}{\Delta_1, p}{\Omega}}
\]

\[
\trans{\derstate{\Xi}{\Gamma_1}{\Delta_1}{\bang p, \Omega}}
{\derstate{\Xi}{\Gamma_1, \bang p}{\Delta_1}{\Omega}}
\]

\[
\trans{\derstate{\Xi}{\Gamma_1}{\Delta_1}{\one, \Omega}}
{\derstate{\Xi}{\Gamma_1}{\Delta_1}{\Omega}}
\]

\[
\trans{\derstate{\Xi}{\Gamma_1}{\Delta_1}{A \otimes B, \Omega}}
{\derstate{\Xi}{\Gamma_1}{\Delta_1}{A, B, \Omega}}
\]

\[
\trans{\derstate{\Xi}{\Gamma_1}{\Delta_1}{\cdot}}
{\finalstate{\Delta'}{\Gamma_1}{\Delta_1}}
\]

This completes the specification of the abstract machine.

\subsection{Application}

\[
\trans{\dostate{\Delta}{R_1, \Phi}{\Gamma}{\Pi}}
   {\appstate{\Delta}{\Phi}{\Pi}{\Gamma}{R}} \tag{select rule}
\]
\[
\trans{\dostate{\Delta}{\cdot}{\Gamma}{\Pi}}
   {\failstate{\Gamma}{\Delta}} \tag{fail}
\]
\[
\trans{\appstate{\Delta}{\Phi}{\Pi}{\Gamma}{A \lolli B}}
      {\matstate{A \lolli B}{(\Delta; \Phi)}{\cdot}{\Gamma}{\Delta}{A}{\cdot \rightarrow
                                                            \one}} \tag{init rule}
\]


\subsection{Match}

\[
\infer[\mo p~\m{first}]
{\mo \Gamma ; \Delta, p_1, \Delta'' ; \Xi; p, \Omega; H; \cdot; \lstack{R} \rightarrow
   \outsem}
{
 \begin{gathered}
   p_1, \Delta'' \prec p \\
   \mo \Gamma ; \Delta, \Delta''; \Xi, p_1; \Omega; H;
      (\Delta, p_1; \Delta''; p; \Omega; \Xi; \cdot; \cdot); \lstack{R} \rightarrow
      \outsem
  \end{gathered}
}
\]

\[
\infer[\mo p~\m{on}~q]
{\mo \Gamma ; \Delta, p_1, \Delta'' ; \Xi; p, \Omega; H; f, \lstack{C};
   \lstack{R} \rightarrow \outsem}
{
\begin{gathered}
   p_1, \Delta'' \prec p \\
   f = (\Delta_{old}; \Delta'_{old}; q; \Omega_{old}; \Xi_{old}; \Lambda;
         \Upsilon) \\
   \mo \Gamma ; \Delta, \Delta''; \Xi, p_1; \Omega; H; (\Delta, p_1; \Delta'';
         p; \Omega; \Xi; q, \Lambda; \Upsilon), f, \lstack{C}; \lstack{R} \rightarrow \outsem
\end{gathered}
}
\]


\[
\infer[\mo p~\m{on}~\bang q]
{\mo \Gamma ; \Delta, p_1, \Delta'' ; \Xi; p, \Omega; H; f, \lstack{C};
   \lstack{R} \rightarrow \outsem}
{
\begin{gathered}
   p_1, \Delta'' \prec p \\
   f = [\Gamma_{old}; \Delta_{old}; \bang q; \Omega_{old}; \Xi_{old}; \Lambda; \Upsilon] \\
   \mo \Gamma ; \Delta, \Delta''; \Xi, p_1; \Omega; H; (\Delta, p_1; \Delta''; p; \Omega; \Xi;
         \Lambda; q, \Upsilon), f, \lstack{C}; \lstack{R} \rightarrow \outsem
\end{gathered}}
\]

\[
\infer[\mo p~\m{fail}]
{\mo \Gamma ; \Delta; \Xi; p, \Omega; H; \lstack{C}; \lstack{R} \rightarrow \outsem}
{\Delta \npreceq p & \cont \lstack{C} ; H; \lstack{R}; \Gamma \rightarrow \outsem}
\]


\[
\infer[\mo \bang p~\m{first}]
{\mo \Gamma, p_1, \Gamma'' ; \Delta; \Xi; \bang p, \Omega; H; \cdot; \lstack{R}
   \rightarrow \outsem}
{
   \begin{gathered}
      p_1, \Gamma'' \prec \bang p \\
      \mo \Gamma, p_1, \Gamma'' ; \Delta; \Xi; \Omega;
      H; [\Gamma''; \Delta; \bang p ; \Omega; \Xi; \cdot; \cdot]; \lstack{R} \rightarrow \outsem
   \end{gathered}
}
\]

\[
\infer[\mo \bang p~\m{on}~q]
{\mo \Gamma, p_1, \Gamma'' ; \Delta; \Xi; \bang p, \Omega; H; f, \lstack{C};
   \lstack{R}
   \rightarrow \outsem}
{
   \begin{gathered}
      p_1, \Gamma'' \prec \bang p \\
      f = (\Delta_{old}; \Delta'_{old};
         q; \Omega_{old}; \Xi_{old}; \Lambda; \Upsilon) \\
      \mo \Gamma, p_1,
         \Gamma'' ; \Delta; \Xi; \Omega; H; [\Gamma''; \Delta; \bang p ; \Omega; \Xi; q,
      \Lambda; \Upsilon], f, \lstack{C}; \lstack{R} \rightarrow \outsem
   \end{gathered}
}
\]


\[
\infer[\mo \bang p~\m{on}~\bang q]
{\mo \Gamma, p_1, \Gamma'' ; \Delta; \Xi; \bang p, \Omega; H; f, \lstack{C};
   \lstack{R}
   \rightarrow \outsem}
{
   \begin{gathered}
      p_1, \Gamma'' \prec \bang p \\
      f = [\Gamma_{old}; \Delta_{old}; \bang q; \Omega_{old}; \Xi_{old}; \Lambda; \Upsilon] \\
      \mo \Gamma, p_1, \Gamma'' ; \Delta; \Xi; \Omega; H; [\Gamma''; \Delta;
      \bang p ; \Omega; \Xi; \Lambda; q, \Upsilon], f, \lstack{C}; \lstack{R} \rightarrow \outsem
   \end{gathered}
}
\]

\[
\infer[\mo \bang p~\m{fail}]
{\mo \Gamma ; \Delta; \Xi; \bang p, \Omega; H; \lstack{C}; \lstack{R} \rightarrow \outsem}
{\Gamma \npreceq \bang p & \cont \lstack{C}; H; \lstack{R}; \Gamma \rightarrow \outsem}
\]


\begin{align}
\trans{\matstate{A \lolli B}{\rulestk}{\lstack{C}}{\Gamma}{\Delta}{\one,
   \Omega}{\Delta' \rightarrow \Omega'}}
{\matstate{A \lolli B}{\rulestk}{\lstack{C}}{\Gamma}{\Delta}{\Omega}{\Delta'
   \rightarrow \Omega'}} \tag{match $\one$}
\end{align}

\begin{align}
\trans{\matstate{A \lolli B}{\rulestk}{\lstack{C}}{\Gamma}{\Delta}{X \otimes Y,
   \Omega}{\Delta' \rightarrow \Omega'}}
{\matstate{A \lolli B}{\rulestk}{\lstack{C}}{\Gamma}{\Delta}{X, Y,
   \Omega}{\Delta' \rightarrow \Omega;}} \tag{match $\otimes$}
\end{align}

\begin{align}
\trans{\matstate{A \lolli
   B}{\rulestk}{\lstack{C}}{\Gamma}{\Delta}{\cdot}{\Delta' \rightarrow \Omega'}}
{
   \derstatex{\Gamma}{\Delta}{\Delta'}{\cdot}{\cdot}{B}
} \tag{match end}
\end{align}


\subsection{Continuation}
\[
\infer[\cont p~\m{next}]
{\cont (\Delta; p_1, \Delta''; p, \Omega; \Xi; \Lambda; \Upsilon), \lstack{C};
   H; \lstack{R}; \Gamma \rightarrow \outsem}
{\mo \Gamma ; \Delta, \Delta''; \Xi, p_1; \Omega; H; (\Delta, p_1; \Delta''; p,
      \Omega; H; \Xi; \Lambda; \Upsilon), \lstack{C}; \lstack{R} \rightarrow \outsem}
\]

\[
\infer[\cont p~\m{no~more}]
{\cont (\Delta; \cdot; p, \Omega; \Xi; \Lambda; \Upsilon), \lstack{C}; H;
   \lstack{R}; \Gamma
   \rightarrow \outsem}
{\cont \lstack{C}; H; \lstack{R}; \Gamma \rightarrow \outsem}
\]

\[
\infer[\cont \bang p~\m{next}]
{\cont [p_1, \Gamma'; \Delta; \bang p, \Omega; \Xi; \Lambda; \Upsilon],
   \lstack{C}; H; \lstack{R};
   \Gamma \rightarrow \outsem}
{\mo \Gamma; \Delta; \Xi; \Omega; H; [\Gamma'; \Delta; \bang p, \Omega; \Xi;
   \Lambda; \Upsilon], \lstack{C}; \lstack{R} \rightarrow \outsem}
\]

\[
\infer[\cont \bang p~\m{no~more}]
{\cont [\cdot; \Delta; \bang p, \Omega; \Xi; \Lambda; \Upsilon], \lstack{C}; H;
   \lstack{R}; \Gamma
   \rightarrow \outsem}
{\cont \lstack{C}; H; \lstack{R}; \Gamma \rightarrow \outsem}
\]

\[
\infer[\cont \m{next~rule}]
{\cont \cdot; H; (\Phi, \Delta); \Gamma \rightarrow \outsem}
{\doo \Gamma; \Delta; \Phi \rightarrow \outsem}
\]


\subsection{Derivation}
\[
\infer[\done p]
{\done \Gamma ; \Delta; \Xi; \Gamma_1 ; \Delta_1; p, \Omega \rightarrow \outsem}
{\done \Gamma ; \Delta; \Xi; \Gamma_1 ; p, \Delta_1; \Omega \rightarrow \outsem}
\tab
\infer[\done \bang p]
{\done \Gamma ; \Delta ; \Xi; \Gamma_1 ; \Delta_1; \bang p, \Omega \rightarrow
   \outsem}
{\done \Gamma ; \Delta ; \Xi; \Gamma_1, p; \Delta_1; \Omega \rightarrow \outsem}
\]

\[
\infer[\done 1]
{\done \Gamma; \Delta; \Xi; \Gamma_1 ; \Delta_1; 1, \Omega \rightarrow \outsem}
{\done \Gamma; \Delta; \Xi; \Gamma_1 ; \Delta_1; \Omega \rightarrow \outsem}
\tab
\infer[\done \otimes]
{\done \Gamma ; \Delta; \Xi; \Gamma_1; \Delta_1; A \otimes B, \Omega \rightarrow
   \outsem}
{\done \Gamma ; \Delta; \Xi; \Gamma_1; \Delta_1; A, B, \Omega \rightarrow
   \outsem}
\]

%\[
\infer[\done \m{comp}]
{\done \Gamma; \Delta ; \Xi; \Gamma_1; \Delta_1; \compsz{A}{B}, \Omega
   \rightarrow \outsem}
{\mc \Gamma; \Delta; \Xi; \Gamma_1; \Delta_1; \cdot; A ; \cdot; \cdot;
   \compsz{A}{B}; \Omega; \Delta \rightarrow \outsem}
\]

%\[
\infer[\done \m{agg}]
{\done \Gamma; \Delta ; \Xi; \Gamma_1; \Delta_1; \aggsz{A}{B}{C}, \Omega
   \rightarrow \outsem}
{\ma \Gamma; \Delta; \Xi; \Gamma_1; \Delta_1; \cdot; A ; \cdot; \cdot;
   \aggsz{A}{B}{C}; \Omega; \Delta; \cdot \rightarrow \outsem}
\]

\[
\infer[\done \m{end}]
{\done \Gamma; \Delta; \Xi; \Gamma_1; \Delta_1; \cdot \rightarrow \Xi; \Gamma_1;
\Delta_1}
{}
\]


\iffalse
\section{Comprehensions}
\subsection{Match}

\[
\infer[\mc{AB} p~\m{first}]
{\mc{AB} \Gamma; \Delta, p_1, \Delta''; \Xi_N; \Gamma_{N1}; \Delta_{N1}; \cdot; p,
   \Omega; \cdot; \cdot; \Omega_N; \Delta_N \rightarrow \outsem}
{
   \begin{gathered}
      p_1, \Delta'' \prec p \\
      \mc{AB} \Gamma; \Delta, \Delta''; \Xi_N; \Gamma_{N1};
      \Delta_{N1}; \Xi, p_1; \Omega; (\Delta, p_1; \Delta''; \cdot; p; \Omega;
            \cdot; \cdot); \cdot; \Omega_N; \Delta_N \rightarrow \outsem
   \end{gathered}}
\]

\[
\infer[\mc{AB} p~\m{on}~q]
{\mc{AB} \Gamma; \Delta, p_1, \Delta''; \Xi_N; \Gamma_{N1}; \Delta_{N1}; \Xi; p,
   \Omega; f, \lstack{C}; \lstack{P}; \Omega_N; \Delta_N \rightarrow \outsem}
{
   \begin{gathered}
      p_1, \Delta'' \prec p \\
      f = (\Delta_{old}; \Delta'_{old}; \Xi_{old}; q; \Omega_{old}; \Lambda; \Upsilon) \\
      f' = (\Delta, p_1; \Delta''; \Xi; p; \Omega; q, \Lambda; \Upsilon) \\
      \mc{AB} \Gamma; \Delta, \Delta''; \Xi_N; \Gamma_{N1}; \Delta_{N1}; \Xi, p_1; \Omega;
      f', f, \lstack{C}; \lstack{P}; \Omega_N; \Delta_N \rightarrow \outsem
   \end{gathered}
}
\]


\[
\infer[\mc{AB} p~\m{on}~\bang q~\lstack{C}]
{\mc{AB} \Gamma; \Delta, p_1, \Delta''; \Xi_N; \Gamma_{N1}; \Delta_{N1}; \Xi; p,
   \Omega; f, \lstack{C}; \lstack{P}; \Omega_N; \Delta_N \rightarrow \outsem}
{
   \begin{gathered}
      p_1, \Delta'' \prec p \\
      f = [\Gamma_{old}; \Delta_{old}; \Xi_{old}; q; \Omega_{old}; \Lambda; \Upsilon] \\
      f' = (\Delta, p_1; \Delta''; \Xi; p; \Omega; \Lambda; q, \Upsilon) \\
      \mc{AB} \Gamma; \Delta, \Delta''; \Xi_N; \Gamma_{N1}; \Delta_{N1}; \Xi,
      p_1; \Omega; f', f, \lstack{C}; \lstack{P}; \Omega_N; \Delta_N \rightarrow \outsem
   \end{gathered}
}
\]


\[
\infer[\mc{AB} p~\m{on}~\bang q~\lstack{P}]
{\mc{AB} \Gamma; \Delta, p_1, \Delta''; \Xi_N; \Gamma_{N1}; \Delta_{N1}; \cdot; p,
   \Omega; \cdot; f, \lstack{P}; \Omega_N; \Delta_N \rightarrow \outsem}
{
   \begin{gathered}
      p_1, \Delta'' \prec p \\
      f = [\Gamma_{old}; \Delta_N; \cdot; q; \Omega_{old}; \cdot; \Upsilon]\\
      \Delta_N = \Delta, p_1, \Delta'' \\
      f' = (\Delta, p_1; \Delta''; \cdot; p; \Omega; \cdot; q, \Upsilon) \\
      \mc{AB} \Gamma; \Delta, \Delta''; \Xi_N; \Gamma_{N1}; \Delta_{N1}; p_1; \Omega;
         f'; f, \lstack{P};
         \Omega_N; \Delta_N \rightarrow \outsem
   \end{gathered}
}
\]


\[
\infer[\mc{AB} p~\m{fail}]
{\mc{AB} \Gamma; \Delta; \Xi_N; \Gamma_{N1}; \Delta_{N1}; \Xi; p, \Omega;
   \lstack{C}; \lstack{P}; \Omega_N; \Delta_N \rightarrow \outsem}
{\Delta \npreceq p & \contc \Gamma; \Delta_N; \Xi_N; \Gamma_{N1}; \Delta_{N1};
   \lstack{C}; \lstack{P}; \Omega_N \rightarrow \outsem}
\]


\[
\infer[\mc{AB} \bang p~\m{first}]
{\mc{AB} \Gamma, p_1, \Gamma''; \Delta_N; \Xi_N; \Gamma_{N1}; \Delta_{N1};
   \cdot; \bang p, \Omega; \cdot; \cdot; \Omega_N; \Delta_N \rightarrow \outsem}
{\begin{gathered}
   p_1, \Gamma'' \prec \bang p \\
   f = [\Gamma''; \Delta_N; \cdot; \bang p; \cdot; \Omega; \cdot; \cdot] \\
   \mc{AB} \Gamma, p_1, \Gamma''; \Delta_N; \Xi_N; \Gamma_{N1}; \Delta_{N1};
   \cdot; \Omega; \cdot; f;
   \Omega_N; \Delta_N \rightarrow \outsem
\end{gathered}
}
\]

\[
\infer[\mc{AB} \bang p~\m{on}~\bang q~\lstack{P}]
{\mc{AB} \Gamma, p_1, \Gamma''; \Delta_N; \Xi_N; \Gamma_{N1}; \Delta_{N1};
   \cdot; \bang p, \Omega; \cdot; f, \lstack{P}; \Omega_N; \Delta_N \rightarrow \outsem}
{
   \begin{gathered}
      p_1, \Gamma'' \prec \bang p \\
      f = [\Gamma_{old}; \Delta_N; \cdot; \bang q; \Omega_{old}; \cdot; \Upsilon] \\
      f' = [\Gamma''; \Delta_N; \cdot; \bang p; \cdot; \Omega; \cdot; q,
      \Upsilon] \\
      \mc{AB} \Gamma, p_1, \Gamma''; \Delta_N; \Xi_N; \Gamma_{N1}; \Delta_{N1};
      \cdot; \Omega; f', f, \lstack{P}; \Omega_N; \Delta_N \rightarrow \outsem
   \end{gathered}}
\]


\[
\infer[\mc{AB} \bang p~\m{on}~\bang q~\lstack{C}]
{\mc{AB} \Gamma, p_1, \Gamma''; \Delta; \Xi_N; \Gamma_{N1}; \Delta_{N1}; \Xi;
   \bang p, \Omega; f, \lstack{C}; \lstack{P}; \Omega_N; \Delta_N \rightarrow \outsem}
{
   \begin{gathered}
      p_1, \Gamma'' \prec \bang p \\
      f = [\Gamma_{old}; \Delta_{old}; \Xi_{old}; \bang q; \Omega_{old};
      \Lambda; \Upsilon] \\
      f' = [\Gamma''; \Delta; \Xi; \bang p; \cdot; \Omega; \Lambda; q, \Upsilon] \\
      \mc{AB} \Gamma,
      p_1, \Gamma''; \Delta; \Xi_N; \Gamma_{N1}; \Delta_{N1}; \Xi; \Omega; f', f,
      \lstack{C}; \lstack{P}; \Omega_N; \Delta_N \rightarrow \outsem
   \end{gathered}
}
\]


\[
\infer[\mc{AB} \bang p~\m{on}~q~\lstack{C}]
{\mc{AB} \Gamma, p_1, \Gamma''; \Delta; \Xi_N; \Gamma_{N1}; \Delta_{N1}; \Xi;
   \bang p, \Omega; f, \lstack{C}; \lstack{P}; \Omega_N; \Delta_N \rightarrow \outsem}
{
   \begin{gathered}
      p_1, \Gamma'' \prec \bang p \\
      f = (\Delta_{old}; \Delta'_{old}; \Xi_{old}; q; \Omega_{old}; \Lambda; \Upsilon) \\
      f' = [\Gamma''; \Delta; \Xi; \bang p; \cdot; \Omega; \Lambda, q; \Upsilon] \\
      \mc{AB} \Gamma, p_1, \Gamma''; \Delta; \Xi_N; \Gamma_{N1}; \Delta_{N1}; \Xi; \Omega;
      f', f, \lstack{C}; \lstack{P}; \Omega_N; \Delta_N \rightarrow \outsem
   \end{gathered}
}
\]

\[
\infer[\mc{AB} \bang p~\m{fail}]
{\mc \Gamma; \Delta; \Xi_N; \Gamma_{N1}; \Delta_{N1}; \Xi; \bang p, \Omega;
   \lstack{C}; \lstack{P}; \Omega_N; \Delta_N \rightarrow \outsem}
{\Gamma \npreceq \bang p & \contc{AB} \Gamma; \Delta_N; \Xi_N; \Gamma_{N1};
   \Delta_{N1}; \lstack{C}; \lstack{P}; \Omega_N \rightarrow \outsem}
\]

\[
\infer[\mc{AB} \otimes]
{\mc{AB} \Gamma; \Delta; \Xi_N; \Delta_{N1}; \Xi; X \otimes Y, \Omega; \lstack{C};
   \lstack{P}; \Omega_N;
   \Delta_N \rightarrow \outsem}
{\mc{AB} \Gamma; \Delta; \Xi_N; \Delta_{N1}; \Xi; X, Y, \Omega; \lstack{C}; \lstack{P}; \Omega_N; \Delta_N
   \rightarrow \outsem}
\]

\[
\infer[\mc{AB} \m{end}]
{\mc{AB} \Gamma; \Delta; \Xi_N; \Gamma_{N1}; \Delta_{N1}; \Xi; \cdot; \lstack{C};
   \lstack{P}; \Omega_N; \Delta_N \rightarrow \outsem}
{\fix{AB} \Gamma; \Delta; \Xi_N; \Gamma_{N1}; \Delta_{N1}; \Xi; \lstack{C}; \lstack{P}; \Omega_N; \Delta_N
   \rightarrow \outsem}
\]


\subsection{Stack transformation}
\[
\infer[\fix{\compsz{A}{B}} \m{end~linear}]
{\fix{\compsz{A}{B}} \Gamma; \Delta; \Xi_N; \Gamma_{N1}; \Delta_{N1}; \Xi; f; \lstack{P};  \Omega_N; \Delta_N \rightarrow \outsem}
{
   \begin{gathered}
      \strans \Xi; \lstack{P}; \lstack{P'} \\
      f = (\Delta_x; \Delta''; \cdot; p; \Omega; \cdot; \Upsilon) \\
      f' = (\Delta_x - \Xi; \Delta'' - \Xi; \cdot; p; \Omega; \cdot;
            \Upsilon) \\
      \dc{\compsz{A}{B}} \Gamma; \Delta; \Xi_N, \Xi; \Gamma_{N1};
      \Delta_{N1}; B; f' ; \lstack{P'} ; \Omega_N \rightarrow \outsem
   \end{gathered}
}
\]

\[
\infer[\fix{AB} \m{more}]
{\fix{AB} \Gamma; \Delta; \Xi_N; \Gamma_{N1}; \Delta_{N1}; \Xi; \_, f, \lstack{C};
   \lstack{P}; \Omega_N;
   \Delta_N \rightarrow \outsem}
{\fix{AB} \Gamma; \Delta; \Xi_N; \Gamma_{N1}; \Delta_{N1}; \Xi; f, \lstack{C};
   \lstack{P}; \Omega_N;
   \Delta_N \rightarrow \outsem}
\]

\[
\infer[\fix{\compsz{A}{B}} \m{end~empty}]
{\fix{\compsz{A}{B}} \Gamma; \Delta; \Xi_N; \Gamma_{N1}; \Delta_{N1}; \Xi; \cdot;
   \lstack{P}; \Omega_N; \Delta_N \rightarrow \outsem}
{\begin{gathered}
   \strans \Xi; \lstack{P}; \lstack{P'} \\
   \dc{\compsz{A}{B}} \Gamma; \Delta ; \Xi_N, \Xi; \Gamma_{N1};
      \Delta_{N1}; B; \cdot ; \lstack{P'} ; \Omega_N \rightarrow \outsem
 \end{gathered}
}
\]

\[
\infer[\strans]
{\strans \Xi; [\Gamma'; \Delta_N; \cdot; \bang p; \Omega; \cdot; \Upsilon],
   \lstack{P}; [\Gamma'; \Delta_N - \Xi; \cdot; \bang p, \Omega; \cdot;
   \Upsilon], \lstack{P'}}
{\strans \Xi; \lstack{P}; \lstack{P'}}
\]

\[
\infer[\strans \m{end}]
{\strans \Xi; \cdot; \cdot}
{\strans \Xi; \cdot; \cdot}
\]


\subsection{Continuation}
\[
\infer[\contc{AB} \m{next}~\lstack{C}~p]
{\contc{AB} \Gamma; \Delta_N; \Xi_N; \Gamma_{N1}; \Delta_{N1}; f, \lstack{C};
   \lstack{P}; \Omega_N \rightarrow
\outsem}
{\begin{gathered}
   f = (\Delta; p_1, \Delta''; \Xi; p; \Omega; \Lambda; \Upsilon) \\
   f' =  (\Delta, p_1; \Delta''; \Xi; p; \Omega; \Lambda; \Upsilon) \\
   \mc{AB} \Gamma; \Delta; \Xi_N; \Gamma_{N1}; \Delta_{N1}; \Xi; \Omega; f',
     \lstack{C}; \lstack{P}; \Omega_N;
\Delta_N \rightarrow \outsem
\end{gathered}
}
\]

\[
\infer[\contc{AB} \m{next}~\lstack{C}~\bang p]
{\contc{AB} \Gamma; \Delta_N; \Xi_N; \Gamma_{N1}; \Delta_{N1}; f, \lstack{C};
   \lstack{P}; \Omega_N \rightarrow \outsem}
{
\begin{gathered}
   f =  [p_1, \Gamma'; \Delta; \Xi; \bang p; \Omega; \Lambda; \Upsilon] \\
   f' = [\Gamma'; \Delta; \Xi; \bang p; \Omega; \Lambda; \Upsilon] \\
   \mc{AB} \Gamma; \Delta; \Xi_N; \Gamma_{N1}; \Delta_{N1}; \Xi; \Omega; f',
     \lstack{C}; \lstack{P}; \Omega_N; \Delta_N
   \rightarrow \outsem
\end{gathered}
}
\]

\[
\infer[\contc{AB} \m{next}~\lstack{C}~\m{empty}~p]
{\contc{AB} \Gamma; \Delta_N; \Xi_N; \Gamma_{N1}; \Delta_{N1}; (\Delta; \cdot; \Xi;
      p; \Omega; \Lambda; \Upsilon), \lstack{C}; \lstack{P}; \Omega_N \rightarrow \outsem}
{\contc{AB} \Gamma; \Delta_N; \Xi_N; \Gamma_{N1}; \Delta_{N1}; \lstack{C};
   \lstack{P}; \Omega_N \rightarrow \outsem}
\]

\[
\infer[\contc{AB} \m{next}~\lstack{C}~\m{empty}~\bang p]
{\contc{AB} \Gamma; \Delta_N; \Xi_N; \Gamma_{N1}; \Delta_{N1}; [\cdot; \Delta; \Xi;
   \bang p; \Omega; \Lambda; \Upsilon], \lstack{C}; \lstack{P}; \Omega_N \rightarrow \outsem}
{\contc{AB} \Gamma; \Delta_N; \Xi_N; \Gamma_{N1}; \Delta_{N1}; \lstack{C};
   \lstack{P}; \Omega_N
   \rightarrow \outsem}
\]

\[
\infer[\contc{AB} \m{next}~\lstack{P}~\bang p]
{\contc{AB} \Gamma; \Delta_N; \Xi_N; \Gamma_{N1}; \Delta_{N1}; \cdot; f,
   \lstack{P}; \Omega_N
   \rightarrow \outsem}
{
   \begin{gathered}
      f = [p_1, \Gamma'; \Delta_N; \cdot; \bang p; \Omega; \cdot; \Upsilon] \\
      f' = [\Gamma'; \Delta_N; \cdot; \bang p; \Omega; \cdot; \Upsilon] \\
      \mc{AB} \Gamma; \Delta_N; \Xi_N; \Gamma_{N1}; \Delta_{N1}; \cdot; \Omega;
      \cdot; f, \lstack{P}; \Omega_N; \Delta_N \rightarrow \outsem
   \end{gathered}
}
\]

\[
\infer[\contc{AB} \m{next}~\lstack{P}~\m{empty}~\bang p]
{\contc{AB} \Gamma; \Delta_N; \Xi_N; \Gamma_{N1}; \Delta_{N1}; \cdot; [\cdot;
   \Delta_N; \cdot; \bang p; \Omega; \cdot; \Upsilon], \lstack{P}; \Omega_N
   \rightarrow \outsem}
{\contc{AB} \Gamma; \Delta_N; \Xi_N; \Gamma_{N1}; \Delta_{N1}; \cdot; \lstack{P};
   \Omega_N \rightarrow \outsem}
\]

\[
\infer[\contc{AB} \m{end}]
{\contc{AB} \Gamma; \Delta_N; \Xi_N; \Gamma_{N1}; \Delta_{N1}; \cdot; \cdot;
   \Omega \rightarrow \outsem}
{\done \Gamma; \Delta_N; \Xi_N; \Gamma_{N1}; \Delta_{N1}; \Omega \rightarrow
   \outsem}
\]


\subsection{Derivation}
\[
\infer[\dc{AB} p]
{\dc{AB} \Gamma; \Delta_N; \Xi_N; \Gamma_1; \Delta_1; p, \Omega; \lstack{C}; \lstack{P}; \Omega_N
   \rightarrow \outsem}
{\dc{AB} \Gamma; \Delta_N; \Xi_N; \Gamma_1; \Delta_1, p; \Omega; \lstack{C}; \lstack{P}; \Omega_N
   \rightarrow \outsem}
\]

\[
\infer[\dc{AB} \bang p]
{\dc{AB} \Gamma; \Delta_N; \Xi_N; \Gamma_1; \Delta_1; \bang p, \Omega; \lstack{C};
   \lstack{P}; \Omega_N \rightarrow \outsem}
{\dc{AB} \Gamma; \Delta_N; \Xi_N; \Gamma_1, p; \Delta_1; \Omega; \lstack{C}; \lstack{P}; \Omega_N
   \rightarrow \outsem}
\]

\[
\infer[\dc{AB} \otimes]
{\dc{AB} \Gamma; \Delta_N; \Xi_N; \Gamma_1; \Delta_1; A \otimes B, \Omega; \lstack{C}; \lstack{P}; \Omega_N
   \rightarrow \outsem}
{\dc{AB} \Gamma; \Delta_N; \Xi_N; \Gamma_1; \Delta_1; A, B, \Omega; \lstack{C}; \lstack{P}; \Omega_N
   \rightarrow \outsem}
\]

\[
\infer[\dc{AB} \one]
{\dc{AB} \Gamma; \Delta_N; \Xi_N; \Gamma_1; \Delta_1; 1, \Omega; \lstack{C}; \lstack{P}; \Omega_N
   \rightarrow \outsem}
{\dc{AB} \Gamma; \Delta_N; \Xi_N; \Gamma_1; \Delta_1; \Omega; \lstack{C}; \lstack{P}; \Omega_N
   \rightarrow \outsem}
\]

\[
\infer[\dc{AB} \m{end}]
{\dc{AB} \Gamma; \Delta_N; \Xi_N; \Gamma_1; \Delta_1; \cdot; \lstack{C}; \lstack{P}; \Omega_N
   \rightarrow \outsem}
{\contc{AB} \Gamma; \Delta_N; \Xi_N; \Gamma_1; \Delta_1; \lstack{C}; \lstack{P}; \Omega_N
   \rightarrow \outsem}
\]



\section{Aggregates}
\subsection{Match}
\[
\infer[\ma{AG} p~\m{first}]
{\ma{AG} \Gamma; \Delta, p_1, \Delta''; \Xi_N; \Gamma_{N1}; \Delta_{N1}; \cdot; p,
   \Omega; \cdot; \cdot; \Omega_N; \Delta_N; \Sigma \rightarrow \outsem}
{
   \begin{gathered}
      p_1, \Delta'' \prec p \\
      f = (\Delta, p_1; \Delta''; \cdot; p; \Omega;
            \cdot; \cdot) \\
      \ma{AG} \Gamma; \Delta, \Delta''; \Xi_N; \Gamma_{N1};
         \Delta_{N1}; \Xi, p_1; \Omega; f; \cdot; \Omega_N; \Delta_N; \Sigma \rightarrow \outsem
   \end{gathered}
}
\]

\[
\infer[\ma{AG} p~\m{on}~q]
{\ma{AG} \Gamma; \Delta, p_1, \Delta''; \Xi_N; \Gamma_{N1}; \Delta_{N1}; \Xi; p,
   \Omega; C_1, \lstack{C}; \lstack{P}; \Omega_N; \Delta_N; \Sigma \rightarrow \outsem}
{
   \begin{gathered}
      p_1, \Delta'' \prec p \\
      f = (\Delta_{old}; \Delta'_{old}; \Xi_{old}; q; \Omega_{old}; \Lambda; \Upsilon) \\
      f' =  (\Delta, p_1; \Delta''; \Xi; p; \Omega; q, \Lambda; \Upsilon) \\
      \ma{AG} \Gamma; \Delta, \Delta''; \Xi_N; \Gamma_{N1};
         \Delta_{N1}; \Xi, p_1; \Omega; f', f, \lstack{C}; \lstack{P}; \Omega_N;
         \Delta_N; \Sigma \rightarrow \outsem
   \end{gathered}
}
\]

\[
\infer[\ma{AG} p~\m{on}~\bang q~\lstack{C}]
{\ma{AG} \Gamma; \Delta, p_1, \Delta''; \Xi_N; \Gamma_{N1}; \Delta_{N1}; \Xi; p,
   \Omega; C_1, \lstack{C}; \lstack{P}; \Omega_N; \Delta_N; \Sigma \rightarrow \outsem}
{
   \begin{gathered}
      p_1, \Delta'' \prec p \\
      f = [\Gamma_{old}; \Delta_{old}; \Xi_{old}; q;
         \Omega_{old}; \Lambda; \Upsilon]\\
      f' = (\Delta, p_1; \Delta''; \Xi; p; \Omega; \Lambda; q, \Upsilon) \\
      \ma{AG} \Gamma; \Delta, \Delta''; \Xi_N; \Gamma_{N1};
         \Delta_{N1}; \Xi, p_1; \Omega;
         f', f, \lstack{C}; \lstack{P}; \Omega_N; \Delta_N;
         \Sigma \rightarrow \outsem
   \end{gathered}
}
\]
\[
\infer[\ma{AG} p~\m{on}~\bang q~\lstack{P}]
{\ma{AG} \Gamma; \Delta, p_1, \Delta''; \Xi_N; \Gamma_{N1}; \Delta_{N1}; \cdot; p,
   \Omega; \cdot; f, \lstack{P}; \Omega_N; \Delta_N; \Sigma \rightarrow \outsem}
{
   \begin{gathered}
      p_1, \Delta'' \prec p \\
      f = [\Gamma_{old}; \Delta_N; \cdot; q; \Omega_{old}; \cdot; \Upsilon]\\
      f' = (\Delta, p_1; \Delta''; \cdot; p; \Omega; \cdot; q, \Upsilon) \\
      \ma{AG} \Gamma; \Delta, \Delta''; \Xi_N;
            \Gamma_{N1}; \Delta_{N1}; p_1; \Omega; f'; f, \lstack{P}; \Omega_N;
            \Delta_N; \Sigma \rightarrow \outsem
   \end{gathered}
}
\]

\[
\infer[\ma{AG} p~\m{fail}]
{\ma{AG} \Gamma; \Delta; \Xi_N; \Gamma_{N1}; \Delta_{N1}; \Xi; p, \Omega;
   \lstack{C}; \lstack{P}; \Omega_N; \Delta_N; \Sigma \rightarrow \outsem}
{\conta{AG} \Gamma; \Delta_N; \Xi_N; \Gamma_{N1}; \Delta_{N1}; \lstack{C};
   \lstack{P}; \Omega_N;
   \Sigma \rightarrow \outsem}
\]



\begin{multline}
\transx{
   \matstatea{\Delta_N}{\cdot;
      \lstack{P}}{\Gamma, p_1, \Gamma''}{\Delta}{\bang p, \Omega}{\Delta' \rightarrow
         \Omega'}{\Sigma}
}
{
   \matstatea{\Delta_N}{\cdot; \pframe{\Gamma''}{\Delta}{\bang
   p}{\Omega}{\Delta'}{\Omega'}, \lstack{P}}{\Gamma, p_1, \Gamma''}{\Delta}{\Omega}
   {\Delta' \rightarrow \Omega' \otimes \bang p}{\Sigma}
}
\end{multline}

\begin{multline}
\transx{
   \matstatea{\Delta_N}{\lstack{C};
      \lstack{P}}{\Gamma, p_1, \Gamma''}{\Delta}{\bang p, \Omega}{\Delta' \rightarrow
         \Omega'}{\Sigma}
}
{
   \matstatea{\Delta_N}{\pframe{\Gamma''}{\Delta}{\bang
   p}{\Omega}{\Delta'}{\Omega'}, \lstack{C} ; \lstack{P}}{\Gamma, p_1, \Gamma''}{\Delta}{\Omega}
   {\Delta' \rightarrow \Omega' \otimes \bang p}{\Sigma}
}
\end{multline}


\[
\trans{
   \matstatea{\Delta_N}{\lstack{C}; \lstack{P}}{\Gamma}{\Delta}{\bang p,
      \Omega}{\Delta' \rightarrow \Omega'}{\Sigma}
}
{
   \contstatea{\Delta_N}{\lstack{C} ; \lstack{P}}{\Gamma}{\Sigma}
}
\]


\[
\infer[\ma{AG} \otimes]
{\ma{AG} \Gamma; \Delta; \Xi_N; \Gamma_{N1}; \Delta_{N1}; \Xi; X \otimes Y, \Omega;
   \lstack{C}; \lstack{P}; \Omega_N;
   \Delta_N; \Sigma \rightarrow \outsem}
{\ma{AG} \Gamma; \Delta; \Xi_N; \Gamma_{N1}; \Delta_{N1}; \Xi; X, Y, \Omega; \lstack{C};
   \lstack{P}; \Omega_N; \Delta_N; \Sigma \rightarrow \outsem}
\]

\subsection{Stack Transformation}

\[
\underset{
   \begin{gathered}
   \Pi(\m{agg}) = \forall_{\widehat{v}, \Sigma'}.
   (\defstwo{agg}{\widehat{v}}{\Sigma'} \lolli ((\lambda x. C x)\Sigma' \with (\forall_{\widehat{x}, \sigma}.
                                                ((A \lolli B) \otimes
                                                 \defstwo{agg}{\widehat{v}}{\sigma
                                                 ::\Sigma'})))) \\
   f' = remove(f, \Delta') \\
   \lstack{P'} = remove(\lstack{P}, \Delta') \\
   V = \Psi(\sigma)
   \end{gathered}
}
{
   \trans{
      \fixstatea{\Delta}{\Xi; \Delta'}{f; \lstack{P}}{\Gamma}{\Sigma}
   }
   {
      \derstatea{\Delta}{\Xi; \Delta'}{\Gamma_1}{\Delta_1}{V :: \Sigma}{f';
         \lstack{P'}}{B\{\Psi(\widehat{x}), V / \widehat{x}, \sigma \}}
   }
}
\]

\[
\trans{
   \fixstatea{\Delta}{\Xi; \Delta'}{\_, f, \lstack{C}; \lstack{P}}{\Gamma}{\Sigma}
}
{
   \fixstatea{\Delta}{\Xi; \Delta'}{f, \lstack{C}; \lstack{P}}{\Gamma}{\Sigma}
}
\]

\[
\underset{
   \begin{gathered}
   \Pi(\m{agg}) = \forall_{\widehat{v}, \Sigma'}.
   (\defstwo{agg}{\widehat{v}}{\Sigma'} \lolli ((\lambda x. C x)\Sigma' \with (\forall_{\widehat{x}, \sigma}.
                                                ((A \lolli B) \otimes
                                                 \defstwo{agg}{\widehat{v}}{\sigma
                                                 ::\Sigma'})))) \\
   \lstack{P'} = remove(\lstack{P}, \Delta') \\
   V = \Psi(\sigma)
   \end{gathered}
}
{
   \trans{
      \fixstatea{\Delta}{\Xi; \Delta'}{\cdot; \lstack{P}}{\Gamma}{\Sigma}
   }
   {
      \derstatea{\Delta}{\Xi, \Delta'}{\Gamma_{N1}}{\Delta_{N1}}{V :: \Sigma}{\cdot;
         \lstack{P}'}{B\{\Psi(\widehat{x}), V / \widehat{x}, \sigma \}}
   }
}
\]

\subsection{Continuation}
\[
\infer[\conta{AG} \m{next}~\lstack{C}~p]
{\conta{AG} \Gamma; \Delta_N; \Xi_N; \Gamma_{N1}; \Delta_{N1}; f, \lstack{C}; \lstack{P}; \Omega_N; \Sigma
\rightarrow \outsem}
{
   \begin{gathered}
      f = (\Delta; p_1, \Delta''; \Xi; p; \Omega; \Lambda; \Upsilon) \\
      f' =  (\Delta, p_1; \Delta''; \Xi; p; \Omega; \Lambda; \Upsilon) \\
      \ma{AG} \Gamma; \Delta; \Xi_N; \Gamma_{N1}; \Delta_{N1}; \Xi; \Omega; f', \lstack{C}; \lstack{P}; \Omega_N;
         \Delta_N; \Sigma \rightarrow \outsem
   \end{gathered}
}
\]

\[
\infer[\conta{AG} \m{next}~\lstack{C}~\bang p]
{\conta{AG} \Gamma; \Delta_N; \Xi_N; \Gamma_{N1}; \Delta_{N1}; f, \lstack{C};
   \lstack{P}; \Omega_N; \Sigma \rightarrow \outsem}
{
   \begin{gathered}
      f =  [p_1, \Gamma'; \Delta; \Xi; \bang p; \Omega; \Lambda; \Upsilon] \\
      f' = [\Gamma'; \Delta; \Xi; \bang p; \Omega; \Lambda; \Upsilon] \\
      \ma{AG} \Gamma; \Delta; \Xi_N; \Gamma_{N1}; \Delta_{N1}; \Xi; \Omega; f', \lstack{C}; \lstack{P}; \Omega_N;
         \Delta_N; \Sigma \rightarrow \outsem
   \end{gathered}
}
\]

\[
\infer[\conta{AG} \m{next}~\lstack{C}~\m{empty}~p]
{\conta{AG} \Gamma; \Delta_N; \Xi_N; \Gamma_{N1}; \Delta_{N1}; f, \lstack{C}; \lstack{P}; \Omega_N; \Sigma \rightarrow
\outsem}
{
   \begin{gathered}
      f = (\Delta; \cdot; \Xi; p; \Omega; \Lambda; \Upsilon) \\
      \conta{AG} \Gamma; \Delta_N; \Xi_N; \Gamma_{N1}; \Delta_{N1}; \lstack{C};
         \lstack{P}; \Omega_N; \Sigma \rightarrow \outsem
   \end{gathered}
}
\]

\[
\infer[\conta{AG} \m{next}~\lstack{C}~\m{empty}~\bang p]
{\conta{AG} \Gamma; \Delta_N; \Xi_N; \Gamma_{N1}; \Delta_{N1}; f, \lstack{C}; \lstack{P}; \Omega_N; \Sigma \rightarrow
   \outsem}
{
   \begin{gathered}
      f = [\cdot; \Delta; \Xi;
         \bang p; \Omega; \Lambda; \Upsilon] \\
      \conta{AG} \Gamma; \Delta_N; \Xi_N; \Gamma_{N1}; \Delta_{N1}; \lstack{C};
         \lstack{P}; \Omega_N; \Sigma \rightarrow \outsem
   \end{gathered}
}
\]

\[
\infer[\conta{AG} \m{next}~\lstack{P}~\bang p]
{\conta{AG} \Gamma; \Delta_N; \Xi_N; \Gamma_{N1}; \Delta_{N1}; \cdot; f, \lstack{P}; \Omega_N; \Sigma \rightarrow \outsem}
{\begin{gathered}
   f = [p_1, \Gamma'; \Delta_N; \cdot; \bang p; \Omega; \cdot; \Upsilon] \\
   f' = [\Gamma'; \Delta_N; \cdot; \bang p; \Omega; \cdot; \Upsilon] \\
   \ma{AG} \Gamma; \Delta_N; \Xi_N; \Gamma_{N1}; \Delta_{N1}; \cdot; \Omega; \cdot;
      f', \lstack{P}; \Omega_N; \Delta_N; \Sigma \rightarrow \outsem
 \end{gathered}
}
\]

\[
\infer[\conta{AG} \m{next}~\lstack{P}~\m{empty}~\bang p]
{\conta{AG} \Gamma; \Delta_N; \Xi_N; \Gamma_{N1}; \Delta_{N1}; \cdot; f, \lstack{P}; \Omega_N; \Sigma
   \rightarrow \outsem}
{\begin{gathered}
   f =  [\cdot; \Delta_N; \cdot; \bang p; \Omega; \cdot; \Upsilon] \\
   \conta{AG} \Gamma; \Delta_N; \Xi_N; \Gamma_{N1}; \Delta_{N1}; \cdot; \lstack{P};
      \Omega_N; \Sigma \rightarrow \outsem
 \end{gathered}
}
\]

\[
\infer[\conta{\aggsz{A}{B}{C}} \m{end}]
{\conta{\aggsz{A}{B}{C}} \Gamma; \Delta_N; \Xi_N; \Gamma_{N1}; \Delta_{N1}; \cdot; \cdot;
   \Omega; \Sigma \rightarrow \outsem}
{\done \Gamma; \Delta_N; \Xi_N; \Gamma_{N1}; \Delta_{N1}; (\lambda x. C
      x)\Sigma,
   \Omega \rightarrow \outsem}
\]

\subsection{Derivation}

\[
\trans{
   \derstatea{\Delta}{\Xi}{\Gamma_1}{\Delta_1}{\Sigma}{\lstack{C};
      \lstack{P}}{p, \Omega}
}
{
   \derstatea{\Delta}{\Xi}{\Gamma_1}{\Delta_1, p}{\Sigma}{\lstack{C};
      \lstack{P}}{\Omega}
}
\]

\[
\trans{
   \derstatea{\Delta}{\Xi}{\Gamma_1}{\Delta_1}{\Sigma}{\lstack{C};
      \lstack{P}}{\bang p, \Omega}
}
{
   \derstatea{\Delta}{\Xi}{\Gamma_1, p}{\Delta_1}{\Sigma}{\lstack{C};
      \lstack{P}}{\Omega}
}
\]

\[
\trans{
   \derstatea{\Delta}{\Xi}{\Gamma_1}{\Delta_1}{\Sigma}{\lstack{C};
      \lstack{P}}{X \otimes Y, \Omega}
}
{
   \derstatea{\Delta}{\Xi}{\Gamma_1, p}{\Delta_1}{\Sigma}{\lstack{C};
      \lstack{P}}{X, Y, \Omega}
}
\]

\[
\trans{
   \derstatea{\Delta}{\Xi}{\Gamma_1}{\Delta_1}{\Sigma}{\lstack{C};
      \lstack{P}}{\one, \Omega}
}
{
   \derstatea{\Delta}{\Xi}{\Gamma_1, p}{\Delta_1}{\Sigma}{\lstack{C};
      \lstack{P}}{\Omega}
}
\]

\[
\trans{
   \derstatea{\Delta}{\Xi}{\Gamma_1}{\Delta_1}{\Sigma}{\lstack{C};
      \lstack{P}}{\cdot}
}
{
   \contstatea{\Delta}{\lstack{C} ; \lstack{P}}{\Gamma}{\Sigma}
}
\]

\fi
