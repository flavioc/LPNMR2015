Aggregates can be seen as a more general case of comprehensions. The major
differences are: (1) aggregates accumulate a list of values for each combination
and (2) aggregates have a second head to be derived before deriving the
remaining head terms.

The match judgment $\ma{AG} \Gamma; \Delta; \Xi_N; \Gamma_{N1};
\Delta_{N1}; \Xi; \Omega; \lstack{C}; \lstack{P}; \Omega_N; \Delta_N; \Sigma \rightarrow
\outsem$ is as follows:

\begin{enumerate}

   \item[$\Delta$] multi-set of linear facts remaining up to this point in the
   matching process;

   \item[$\Xi_N$] multi-set of linear facts used during the matching process of
   the body of the rule and all the previous aggregates;

   \item[$\Gamma_{N1}$] set of persistent facts derived up to this point in the
   head of the rule;

   \item[$\Delta_{N1}$] multi-set of linear facts derived up to this point in
   the head of the rule;

   \item[$\Xi$] multi-set of linear facts consumed up to this point;

   \item[$\Omega$] ordered list of terms that need to be matched for the
   comprehension to be applied;

   \item[$\lstack{C}$] continuation stack that contains both linear and persistent
   frames. The first frame must be linear;

   \item[$\lstack{P}$] initial part of the continuation stack with only persistent
   frames;

   \item[$AG$] aggregate $\aggsz{A}{B}{C}$ that is being matched;

   \item[$\Omega_N$] ordered list of remaining terms of the head of the rule to
   be derived;

   \item[$\Delta_N$] multi-set of linear facts that were still available after
   matching the body of the rule and all the previous aggregates. Note that
   $\Delta, \Xi = \Delta_N$;

   \item[$\Sigma$] the list of aggregated values.

\end{enumerate}

The judgment $\ma{AG}$is similar to $\mc{AB}$but it has the extra $\Sigma$ argument that
represents the list of aggregated values. We extend the $\Psi$ context to
include triplets $x : M : \tau$ (variable, term and type) instead of pairs $M :
\tau$ in order to be able to retrieve bound variables for all $\ma{AG}$judgments.
Remember that $\Psi$ is used for the quantification connectives in \fragment.

\subsubsection{Linear fact expressions}

\[
\infer[\ma{AG} p~\m{first}]
{\ma{AG} \Gamma; \Delta, p_1, \Delta''; \Xi_N; \Gamma_{N1}; \Delta_{N1}; \cdot; p,
   \Omega; \cdot; \cdot; \Omega_N; \Delta_N; \Sigma \rightarrow \outsem}
{
   \begin{gathered}
      p_1, \Delta'' \prec p \\
      f = (\Delta, p_1; \Delta''; \cdot; p; \Omega;
            \cdot; \cdot) \\
      \ma{AG} \Gamma; \Delta, \Delta''; \Xi_N; \Gamma_{N1};
         \Delta_{N1}; \Xi, p_1; \Omega; f; \cdot; \Omega_N; \Delta_N; \Sigma \rightarrow \outsem
   \end{gathered}
}
\]

\[
\infer[\ma{AG} p~\m{on}~q]
{\ma{AG} \Gamma; \Delta, p_1, \Delta''; \Xi_N; \Gamma_{N1}; \Delta_{N1}; \Xi; p,
   \Omega; C_1, \lstack{C}; \lstack{P}; \Omega_N; \Delta_N; \Sigma \rightarrow \outsem}
{
   \begin{gathered}
      p_1, \Delta'' \prec p \\
      f = (\Delta_{old}; \Delta'_{old}; \Xi_{old}; q; \Omega_{old}; \Lambda; \Upsilon) \\
      f' =  (\Delta, p_1; \Delta''; \Xi; p; \Omega; q, \Lambda; \Upsilon) \\
      \ma{AG} \Gamma; \Delta, \Delta''; \Xi_N; \Gamma_{N1};
         \Delta_{N1}; \Xi, p_1; \Omega; f', f, \lstack{C}; \lstack{P}; \Omega_N;
         \Delta_N; \Sigma \rightarrow \outsem
   \end{gathered}
}
\]

\[
\infer[\ma{AG} p~\m{on}~\bang q~\lstack{C}]
{\ma{AG} \Gamma; \Delta, p_1, \Delta''; \Xi_N; \Gamma_{N1}; \Delta_{N1}; \Xi; p,
   \Omega; C_1, \lstack{C}; \lstack{P}; \Omega_N; \Delta_N; \Sigma \rightarrow \outsem}
{
   \begin{gathered}
      p_1, \Delta'' \prec p \\
      f = [\Gamma_{old}; \Delta_{old}; \Xi_{old}; q;
         \Omega_{old}; \Lambda; \Upsilon]\\
      f' = (\Delta, p_1; \Delta''; \Xi; p; \Omega; \Lambda; q, \Upsilon) \\
      \ma{AG} \Gamma; \Delta, \Delta''; \Xi_N; \Gamma_{N1};
         \Delta_{N1}; \Xi, p_1; \Omega;
         f', f, \lstack{C}; \lstack{P}; \Omega_N; \Delta_N;
         \Sigma \rightarrow \outsem
   \end{gathered}
}
\]
\[
\infer[\ma{AG} p~\m{on}~\bang q~\lstack{P}]
{\ma{AG} \Gamma; \Delta, p_1, \Delta''; \Xi_N; \Gamma_{N1}; \Delta_{N1}; \cdot; p,
   \Omega; \cdot; f, \lstack{P}; \Omega_N; \Delta_N; \Sigma \rightarrow \outsem}
{
   \begin{gathered}
      p_1, \Delta'' \prec p \\
      f = [\Gamma_{old}; \Delta_N; \cdot; q; \Omega_{old}; \cdot; \Upsilon]\\
      f' = (\Delta, p_1; \Delta''; \cdot; p; \Omega; \cdot; q, \Upsilon) \\
      \ma{AG} \Gamma; \Delta, \Delta''; \Xi_N;
            \Gamma_{N1}; \Delta_{N1}; p_1; \Omega; f'; f, \lstack{P}; \Omega_N;
            \Delta_N; \Sigma \rightarrow \outsem
   \end{gathered}
}
\]

\[
\infer[\ma{AG} p~\m{fail}]
{\ma{AG} \Gamma; \Delta; \Xi_N; \Gamma_{N1}; \Delta_{N1}; \Xi; p, \Omega;
   \lstack{C}; \lstack{P}; \Omega_N; \Delta_N; \Sigma \rightarrow \outsem}
{\conta{AG} \Gamma; \Delta_N; \Xi_N; \Gamma_{N1}; \Delta_{N1}; \lstack{C};
   \lstack{P}; \Omega_N;
   \Sigma \rightarrow \outsem}
\]



\subsubsection{Persistent fact expressions}


\begin{multline}
\transx{
   \matstatea{\Delta_N}{\cdot;
      \lstack{P}}{\Gamma, p_1, \Gamma''}{\Delta}{\bang p, \Omega}{\Delta' \rightarrow
         \Omega'}{\Sigma}
}
{
   \matstatea{\Delta_N}{\cdot; \pframe{\Gamma''}{\Delta}{\bang
   p}{\Omega}{\Delta'}{\Omega'}, \lstack{P}}{\Gamma, p_1, \Gamma''}{\Delta}{\Omega}
   {\Delta' \rightarrow \Omega' \otimes \bang p}{\Sigma}
}
\end{multline}

\begin{multline}
\transx{
   \matstatea{\Delta_N}{\lstack{C};
      \lstack{P}}{\Gamma, p_1, \Gamma''}{\Delta}{\bang p, \Omega}{\Delta' \rightarrow
         \Omega'}{\Sigma}
}
{
   \matstatea{\Delta_N}{\pframe{\Gamma''}{\Delta}{\bang
   p}{\Omega}{\Delta'}{\Omega'}, \lstack{C} ; \lstack{P}}{\Gamma, p_1, \Gamma''}{\Delta}{\Omega}
   {\Delta' \rightarrow \Omega' \otimes \bang p}{\Sigma}
}
\end{multline}


\[
\trans{
   \matstatea{\Delta_N}{\lstack{C}; \lstack{P}}{\Gamma}{\Delta}{\bang p,
      \Omega}{\Delta' \rightarrow \Omega'}{\Sigma}
}
{
   \contstatea{\Delta_N}{\lstack{C} ; \lstack{P}}{\Gamma}{\Sigma}
}
\]



\subsubsection{Deconstruct body}

\[
\infer[\ma{AG} \otimes]
{\ma{AG} \Gamma; \Delta; \Xi_N; \Gamma_{N1}; \Delta_{N1}; \Xi; X \otimes Y, \Omega;
   \lstack{C}; \lstack{P}; \Omega_N;
   \Delta_N; \Sigma \rightarrow \outsem}
{\ma{AG} \Gamma; \Delta; \Xi_N; \Gamma_{N1}; \Delta_{N1}; \Xi; X, Y, \Omega; \lstack{C};
   \lstack{P}; \Omega_N; \Delta_N; \Sigma \rightarrow \outsem}
\]


\subsubsection{Successful match}

When the aggregate body finally matches, we retrieve the term for variable $x$
(the aggregate variable) and add it to the list $\Sigma$.

\[
\infer[\ma{AG} \m{end}]
{\ma{AG} \Psi; \Gamma; \Delta; \Xi_N; \Gamma_{N1}; \Delta_{N1}; \Xi; \cdot;
   \lstack{C}; \lstack{P}; \Omega_N; \Delta_N; \Sigma \rightarrow \outsem}
{\fixa{AG} \Gamma; \Delta; \Xi_N; \Gamma_{N1}; \Delta_{N1}; \Xi; \lstack{C}; \lstack{P}; \Omega_N;
   \Delta_N; V :: \Sigma \rightarrow \outsem & x : V : \tau \in \Psi}
\]


\subsubsection{Continuation stack update}

After matching a single aggregate, the stack is updated as if it was a
comprehension: we drop all but the first linear continuation frame and then fix
the contexts of the remaining stack.  The judgment that updates the stack has
the form $\fixa{AG} \Gamma; \Delta; \Xi_N; \Gamma_{N1}; \Delta_{N1}; \Xi; \lstack{C};
\lstack{P}; \Omega_N; \Delta_N; \Sigma \rightarrow \outsem$ and every argument
has the usual meaning.

\subsubsection{Remove linear continuation frames}


\[
\underset{
   \begin{gathered}
   \Pi(\m{agg}) = \forall_{\widehat{v}, \Sigma'}.
   (\defstwo{agg}{\widehat{v}}{\Sigma'} \lolli ((\lambda x. C x)\Sigma' \with (\forall_{\widehat{x}, \sigma}.
                                                ((A \lolli B) \otimes
                                                 \defstwo{agg}{\widehat{v}}{\sigma
                                                 ::\Sigma'})))) \\
   f' = remove(f, \Delta') \\
   \lstack{P'} = remove(\lstack{P}, \Delta') \\
   V = \Psi(\sigma)
   \end{gathered}
}
{
   \trans{
      \fixstatea{\Delta}{\Xi; \Delta'}{f; \lstack{P}}{\Gamma}{\Sigma}
   }
   {
      \derstatea{\Delta}{\Xi; \Delta'}{\Gamma_1}{\Delta_1}{V :: \Sigma}{f';
         \lstack{P'}}{B\{\Psi(\widehat{x}), V / \widehat{x}, \sigma \}}
   }
}
\]

\[
\trans{
   \fixstatea{\Delta}{\Xi; \Delta'}{\_, f, \lstack{C}; \lstack{P}}{\Gamma}{\Sigma}
}
{
   \fixstatea{\Delta}{\Xi; \Delta'}{f, \lstack{C}; \lstack{P}}{\Gamma}{\Sigma}
}
\]

\[
\underset{
   \begin{gathered}
   \Pi(\m{agg}) = \forall_{\widehat{v}, \Sigma'}.
   (\defstwo{agg}{\widehat{v}}{\Sigma'} \lolli ((\lambda x. C x)\Sigma' \with (\forall_{\widehat{x}, \sigma}.
                                                ((A \lolli B) \otimes
                                                 \defstwo{agg}{\widehat{v}}{\sigma
                                                 ::\Sigma'})))) \\
   \lstack{P'} = remove(\lstack{P}, \Delta') \\
   V = \Psi(\sigma)
   \end{gathered}
}
{
   \trans{
      \fixstatea{\Delta}{\Xi; \Delta'}{\cdot; \lstack{P}}{\Gamma}{\Sigma}
   }
   {
      \derstatea{\Delta}{\Xi, \Delta'}{\Gamma_{N1}}{\Delta_{N1}}{V :: \Sigma}{\cdot;
         \lstack{P}'}{B\{\Psi(\widehat{x}), V / \widehat{x}, \sigma \}}
   }
}
\]


\subsubsection{Aggregate continuation}

If the aggregate match fails, we need to backtrack. The judgment for
backtracking has the form $\conta{AG} \Gamma; \Delta_N; \Xi_N; \Delta_{N1};
\lstack{C}; \lstack{P}; \Omega_N; \Sigma \rightarrow \outsem$.

\paragraph{Using the $\lstack{C}$ stack}

\[
\infer[\conta{AG} \m{next}~\lstack{C}~p]
{\conta{AG} \Gamma; \Delta_N; \Xi_N; \Gamma_{N1}; \Delta_{N1}; f, \lstack{C}; \lstack{P}; \Omega_N; \Sigma
\rightarrow \outsem}
{
   \begin{gathered}
      f = (\Delta; p_1, \Delta''; \Xi; p; \Omega; \Lambda; \Upsilon) \\
      f' =  (\Delta, p_1; \Delta''; \Xi; p; \Omega; \Lambda; \Upsilon) \\
      \ma{AG} \Gamma; \Delta; \Xi_N; \Gamma_{N1}; \Delta_{N1}; \Xi; \Omega; f', \lstack{C}; \lstack{P}; \Omega_N;
         \Delta_N; \Sigma \rightarrow \outsem
   \end{gathered}
}
\]

\[
\infer[\conta{AG} \m{next}~\lstack{C}~\bang p]
{\conta{AG} \Gamma; \Delta_N; \Xi_N; \Gamma_{N1}; \Delta_{N1}; f, \lstack{C};
   \lstack{P}; \Omega_N; \Sigma \rightarrow \outsem}
{
   \begin{gathered}
      f =  [p_1, \Gamma'; \Delta; \Xi; \bang p; \Omega; \Lambda; \Upsilon] \\
      f' = [\Gamma'; \Delta; \Xi; \bang p; \Omega; \Lambda; \Upsilon] \\
      \ma{AG} \Gamma; \Delta; \Xi_N; \Gamma_{N1}; \Delta_{N1}; \Xi; \Omega; f', \lstack{C}; \lstack{P}; \Omega_N;
         \Delta_N; \Sigma \rightarrow \outsem
   \end{gathered}
}
\]

\[
\infer[\conta{AG} \m{next}~\lstack{C}~\m{empty}~p]
{\conta{AG} \Gamma; \Delta_N; \Xi_N; \Gamma_{N1}; \Delta_{N1}; f, \lstack{C}; \lstack{P}; \Omega_N; \Sigma \rightarrow
\outsem}
{
   \begin{gathered}
      f = (\Delta; \cdot; \Xi; p; \Omega; \Lambda; \Upsilon) \\
      \conta{AG} \Gamma; \Delta_N; \Xi_N; \Gamma_{N1}; \Delta_{N1}; \lstack{C};
         \lstack{P}; \Omega_N; \Sigma \rightarrow \outsem
   \end{gathered}
}
\]

\[
\infer[\conta{AG} \m{next}~\lstack{C}~\m{empty}~\bang p]
{\conta{AG} \Gamma; \Delta_N; \Xi_N; \Gamma_{N1}; \Delta_{N1}; f, \lstack{C}; \lstack{P}; \Omega_N; \Sigma \rightarrow
   \outsem}
{
   \begin{gathered}
      f = [\cdot; \Delta; \Xi;
         \bang p; \Omega; \Lambda; \Upsilon] \\
      \conta{AG} \Gamma; \Delta_N; \Xi_N; \Gamma_{N1}; \Delta_{N1}; \lstack{C};
         \lstack{P}; \Omega_N; \Sigma \rightarrow \outsem
   \end{gathered}
}
\]


\paragraph{Using the $\lstack{P}$ stack}

\[
\infer[\conta{AG} \m{next}~\lstack{P}~\bang p]
{\conta{AG} \Gamma; \Delta_N; \Xi_N; \Gamma_{N1}; \Delta_{N1}; \cdot; f, \lstack{P}; \Omega_N; \Sigma \rightarrow \outsem}
{\begin{gathered}
   f = [p_1, \Gamma'; \Delta_N; \cdot; \bang p; \Omega; \cdot; \Upsilon] \\
   f' = [\Gamma'; \Delta_N; \cdot; \bang p; \Omega; \cdot; \Upsilon] \\
   \ma{AG} \Gamma; \Delta_N; \Xi_N; \Gamma_{N1}; \Delta_{N1}; \cdot; \Omega; \cdot;
      f', \lstack{P}; \Omega_N; \Delta_N; \Sigma \rightarrow \outsem
 \end{gathered}
}
\]

\[
\infer[\conta{AG} \m{next}~\lstack{P}~\m{empty}~\bang p]
{\conta{AG} \Gamma; \Delta_N; \Xi_N; \Gamma_{N1}; \Delta_{N1}; \cdot; f, \lstack{P}; \Omega_N; \Sigma
   \rightarrow \outsem}
{\begin{gathered}
   f =  [\cdot; \Delta_N; \cdot; \bang p; \Omega; \cdot; \Upsilon] \\
   \conta{AG} \Gamma; \Delta_N; \Xi_N; \Gamma_{N1}; \Delta_{N1}; \cdot; \lstack{P};
      \Omega_N; \Sigma \rightarrow \outsem
 \end{gathered}
}
\]


\paragraph{Aggregate done}

\[
\infer[\conta{\aggsz{A}{B}{C}} \m{end}]
{\conta{\aggsz{A}{B}{C}} \Gamma; \Delta_N; \Xi_N; \Gamma_{N1}; \Delta_{N1}; \cdot; \cdot;
   \Omega; \Sigma \rightarrow \outsem}
{\done \Gamma; \Delta_N; \Xi_N; \Gamma_{N1}; \Delta_{N1}; (\lambda x. C
      x)\Sigma,
   \Omega \rightarrow \outsem}
\]


\subsubsection{Aggregate Derivation}


\[
\trans{
   \derstatea{\Delta}{\Xi}{\Gamma_1}{\Delta_1}{\Sigma}{\lstack{C};
      \lstack{P}}{p, \Omega}
}
{
   \derstatea{\Delta}{\Xi}{\Gamma_1}{\Delta_1, p}{\Sigma}{\lstack{C};
      \lstack{P}}{\Omega}
}
\]

\[
\trans{
   \derstatea{\Delta}{\Xi}{\Gamma_1}{\Delta_1}{\Sigma}{\lstack{C};
      \lstack{P}}{\bang p, \Omega}
}
{
   \derstatea{\Delta}{\Xi}{\Gamma_1, p}{\Delta_1}{\Sigma}{\lstack{C};
      \lstack{P}}{\Omega}
}
\]

\[
\trans{
   \derstatea{\Delta}{\Xi}{\Gamma_1}{\Delta_1}{\Sigma}{\lstack{C};
      \lstack{P}}{X \otimes Y, \Omega}
}
{
   \derstatea{\Delta}{\Xi}{\Gamma_1, p}{\Delta_1}{\Sigma}{\lstack{C};
      \lstack{P}}{X, Y, \Omega}
}
\]

\[
\trans{
   \derstatea{\Delta}{\Xi}{\Gamma_1}{\Delta_1}{\Sigma}{\lstack{C};
      \lstack{P}}{\one, \Omega}
}
{
   \derstatea{\Delta}{\Xi}{\Gamma_1, p}{\Delta_1}{\Sigma}{\lstack{C};
      \lstack{P}}{\Omega}
}
\]

\[
\trans{
   \derstatea{\Delta}{\Xi}{\Gamma_1}{\Delta_1}{\Sigma}{\lstack{C};
      \lstack{P}}{\cdot}
}
{
   \contstatea{\Delta}{\lstack{C} ; \lstack{P}}{\Gamma}{\Sigma}
}
\]


This completes the specification of LLD.
