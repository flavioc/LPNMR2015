
Proving that deriving a comprehension in LLD is sound in relation to HLD is
built from 4 results: (1) proving that matching the body of a comprehension is
sound in relation to HLD; (2) proving that updating the continuation stacks
makes them suitable for use in the next comprehension applications; (3) proving
that deriving the head of the comprehension is sound in relation to HLD; (4)
proving that we can apply as many comprehensions as the database allows.

\begin{lemma}[Comprehension body match]\label{thm:comprehension_body_match}
Given a comprehension $AB = \compsz{A}{B}$, consider a triplet $T = A; \Gamma;
\Delta_{N}$ and a context $\Delta_{N} = \Delta_1, \Delta_2, \Xi$.

If $\mc{AB} \Gamma; \Delta_1, \Delta_2; \Xi_N; \Gamma_{N1}; \Delta_{N1}; \Xi;
\Omega; \lstack{C}; \lstack{P}; \Omega_N; \Delta_N \rightarrow \outsem$ is
well-formed in relation to $T$ then either:

\begin{itemize}[leftmargin=*]
   \item Match fails:
   \begin{itemize}[leftmargin=\secondm]
      \item $\done \Gamma; \Delta_N; \Xi_N; \Gamma_{N1}; \Delta_{N1}; \Omega_N \rightarrow \outsem$
   \end{itemize}
   
   \item Match succeeds with no backtracking to frames of stack $\lstack{C}$ or
   $\lstack{P}$ ($\lstack{C} \neq \cdot$):

   \begin{itemize}[leftmargin=\secondm]
      \item $\mz \Gamma; \Delta_2 \rightarrow A$
      \item $\mc{AB} \Gamma; \Delta_1; \Xi_N; \Gamma_{N1}; \Delta_{N1}; \Xi,
         \Delta_2; \cdot; \lstack{C'}, \lstack{C}; \lstack{P'}, \lstack{P}; \Omega_N; \Delta_N
         \rightarrow \outsem$ (well-formed in relation to $T$)
   \end{itemize}

   \item Match succeeds with no backtracking to frames of stack $\lstack{P}$ ($\lstack{C} =
         \cdot$):
   \begin{itemize}[leftmargin=\secondm]
      \item $\mz \Gamma; \Delta_2 \rightarrow A$
      \item $\mc{AB} \Gamma; \Delta_1; \Xi_N; \Gamma_{N1}; \Delta_{N1}; \Xi,
         \Delta_2; \cdot; \lstack{C'}; \lstack{P'}, \lstack{P}; \Omega_N; \Delta_N
         \rightarrow \outsem$ (well-formed in relation to $T$)
   \end{itemize}

      \item Match succeeds with backtracking to a linear continuation frame in
   stack $\lstack{C}$ ($\lstack{C} \neq \cdot$):

   \begin{itemize}[leftmargin=\secondm]
      \item $\mz \Gamma; \Xi_1, \dotsc, \Xi_m, p_2, \Xi_c$
      \item $\exists_{f \in C}. f = (\Delta_a; \Delta_{b_1}, p_2,
            \Delta_{b_2}; p; \Xi_1, \dotsc, \Xi_m; \Omega_1, \dotsc,
            \Omega_k; \Lambda_1, \dotsc, \Lambda_m; \Upsilon_1, \dotsc,
            \Upsilon_n)$
      \item $\lstack{C} = \lstack{C'}, f, \lstack{C''}$
      \item $f$ turns into $f' = (\Delta_a, \Delta_{b_1}, p_2;
            \Delta_{b_2}; p; \Xi_1, \dotsc, \Xi_m;
            \Omega_1, \dotsc, \Omega_k; \Lambda_1, \dotsc, \Lambda_m;
            \Upsilon_1, \dotsc, \Upsilon_n)$
      \item $\mc{AB} \Gamma; \Delta_c; \Xi_N; \Gamma_{N1}; \Delta_{N1}; \Xi_1,
         \dotsc, \Xi_m, p_2, \Xi_c; \cdot; \lstack{C'''}, f', \lstack{C''}; \lstack{P};
         \Omega_N; \Delta_N \rightarrow \outsem$ (well-formed in relation to $T$)
      \item $\Delta_c = (\Delta_1, \Delta_2, \Xi) - (\Xi_1, \dotsc, \Xi_m, p_2, \Xi_c)$
   \end{itemize}

   \item Match succeeds with backtracking to a persistent continuation frame
   in stack $\lstack{C}$ ($\lstack{C} \neq \cdot$):
   \begin{itemize}[leftmargin=\secondm]
      \item $\mz \Gamma; \Delta_{c_2}, \Xi_1, \dotsc, \Xi_m \rightarrow A$
      \item $\exists_{f \in C}. f = [\Gamma_1, p_2, \Gamma_2; \Delta_{c_1},
         \Delta_{c_2}; \Xi_1, \dotsc, \Xi_m; \bang p; \Omega_1, \dotsc, \Omega_k;
         \Lambda_1, \dotsc, \Lambda_m; \Upsilon_1, \dotsc, \Upsilon_n]$
      \item $\lstack{C} = \lstack{C'}, f, \lstack{C''}$
      \item $f$ turns into $f' = [\Gamma_2; \Delta_{c_1}, \Delta_{c_2};
         \Xi_1, \dotsc, \Xi_m; \bang p; \Omega_1, \dotsc, \Omega_k; \Lambda_1,
         \dotsc, \Lambda_m; \Upsilon_1, \dotsc, \Upsilon_n]$
      \item $\mc{AB} \Gamma; \Delta_{c_1}; \Xi_N; \Gamma_{N1}; \Delta_{N1};
         \Delta_{c_2}, \Xi_1, \dotsc, \Xi_m; \cdot; \lstack{C'''}, f',
         \lstack{C''}; \lstack{P};
         \Omega_N; \Delta_N \rightarrow \outsem$ (well-formed in relation to $T$)
      \item $\Delta_{c_1}, \Delta_{c_2} = (\Delta_1, \Delta_2, \Xi) - (\Xi_1, \dotsc, \Xi_m)$
   \end{itemize}

   \item Match succeeds with backtracking to a persistent continuation frame
   in stack $\lstack{P}$ ($\lstack{C} = \cdot$):
   \begin{itemize}[leftmargin=\secondm]
      \item $\mz \Gamma; \Delta_{c_2}, \Xi_1, \dotsc, \Xi_m \rightarrow A$
      \item $\exists_{f \in P}. f = [\Gamma_1, p_2, \Gamma_2; \Delta_{c_1}, \Delta_{c_2};
   \Xi_1, \dotsc, \Xi_m; \bang p; \Omega_1, \dotsc, \Omega_k; \Lambda_1,
   \dotsc, \Lambda_m; \Upsilon_1, \dotsc, \Upsilon_n]$
      \item $\lstack{P} = \lstack{P'}, f, \lstack{P''}$
      \item $f$ turns into $f' = [\Gamma_2; \Delta_{c_1},
         \Delta_{c_2}; \Xi_1, \dotsc, \Xi_m; \bang p; \Omega_1, \dotsc,
         \Omega_k; \Lambda_1, \dotsc, \Lambda_m; \Upsilon_1, \dotsc, \Upsilon_n]$
      \item $\mc{AB} \Gamma; \Delta_{c_1}; \Xi_N; \Gamma_{N1}; \Delta_{N1};
      \Delta_{c_2}, \Xi_1, \dotsc, \Xi_m; \cdot; \lstack{C'}; \lstack{P'''}, f',
      \lstack{P''};
      \Omega_N; \Delta_N \rightarrow \outsem$
      (well-formed in relation to $T$)
      \item $\Delta_{c_1}, \Delta_{c_2} = (\Delta_1, \Delta_2, \Xi) - (\Xi_1, \dotsc,
            \Xi_m)$
   \end{itemize}


\end{itemize}
   
If $\contc{AB} \Gamma; \Delta_{N}; \Xi_{N}; \Gamma_{N1}; \Delta_{N1};
\lstack{C}; \lstack{P}; \Omega_N \rightarrow \outsem$ and $\lstack{C}$ and
$\lstack{P}$ are well-formed in relation to $T$ then either:

\begin{itemize}[leftmargin=*]
   \item Match fails:
   \begin{itemize}[leftmargin=\secondm]
      \item $\done \Gamma; \Delta_N; \Xi_N; \Gamma_{N1}; \Delta_{N1}; \Omega_N \rightarrow \outsem$
   \end{itemize}

      \item Match succeeds with backtracking to a linear continuation frame in
   stack $\lstack{C}$ ($\lstack{C} \neq \cdot$):

   \begin{itemize}[leftmargin=\secondm]
      \item $\mz \Gamma; \Xi_1, \dotsc, \Xi_m, p_2, \Xi_c$
      \item $\exists_{f \in C}. f = (\Delta_a; \Delta_{b_1}, p_2,
            \Delta_{b_2}; p; \Xi_1, \dotsc, \Xi_m; \Omega_1, \dotsc,
            \Omega_k; \Lambda_1, \dotsc, \Lambda_m; \Upsilon_1, \dotsc,
            \Upsilon_n)$
      \item $\lstack{C} = \lstack{C'}, f, \lstack{C''}$
      \item $f$ turns into $f' = (\Delta_a, \Delta_{b_1}, p_2;
            \Delta_{b_2}; p; \Xi_1, \dotsc, \Xi_m;
            \Omega_1, \dotsc, \Omega_k; \Lambda_1, \dotsc, \Lambda_m;
            \Upsilon_1, \dotsc, \Upsilon_n)$
      \item $\mc{AB} \Gamma; \Delta_c; \Xi_N; \Gamma_{N1}; \Delta_{N1}; \Xi_1,
         \dotsc, \Xi_m, p_2, \Xi_c; \cdot; \lstack{C'''}, f', \lstack{C''}; \lstack{P};
         \Omega_N; \Delta_N \rightarrow \outsem$ (well-formed in relation to $T$)
      \item $\Delta_c = (\Delta_1, \Delta_2, \Xi) - (\Xi_1, \dotsc, \Xi_m, p_2, \Xi_c)$
   \end{itemize}

   \item Match succeeds with backtracking to a persistent continuation frame
   in stack $\lstack{C}$ ($\lstack{C} \neq \cdot$):
   \begin{itemize}[leftmargin=\secondm]
      \item $\mz \Gamma; \Delta_{c_2}, \Xi_1, \dotsc, \Xi_m \rightarrow A$
      \item $\exists_{f \in C}. f = [\Gamma_1, p_2, \Gamma_2; \Delta_{c_1},
         \Delta_{c_2}; \Xi_1, \dotsc, \Xi_m; \bang p; \Omega_1, \dotsc, \Omega_k;
         \Lambda_1, \dotsc, \Lambda_m; \Upsilon_1, \dotsc, \Upsilon_n]$
      \item $\lstack{C} = \lstack{C'}, f, \lstack{C''}$
      \item $f$ turns into $f' = [\Gamma_2; \Delta_{c_1}, \Delta_{c_2};
         \Xi_1, \dotsc, \Xi_m; \bang p; \Omega_1, \dotsc, \Omega_k; \Lambda_1,
         \dotsc, \Lambda_m; \Upsilon_1, \dotsc, \Upsilon_n]$
      \item $\mc{AB} \Gamma; \Delta_{c_1}; \Xi_N; \Gamma_{N1}; \Delta_{N1};
         \Delta_{c_2}, \Xi_1, \dotsc, \Xi_m; \cdot; \lstack{C'''}, f',
         \lstack{C''}; \lstack{P};
         \Omega_N; \Delta_N \rightarrow \outsem$ (well-formed in relation to $T$)
      \item $\Delta_{c_1}, \Delta_{c_2} = (\Delta_1, \Delta_2, \Xi) - (\Xi_1, \dotsc, \Xi_m)$
   \end{itemize}

   \item Match succeeds with backtracking to a persistent continuation frame
   in stack $\lstack{P}$ ($\lstack{C} = \cdot$):
   \begin{itemize}[leftmargin=\secondm]
      \item $\mz \Gamma; \Delta_{c_2}, \Xi_1, \dotsc, \Xi_m \rightarrow A$
      \item $\exists_{f \in P}. f = [\Gamma_1, p_2, \Gamma_2; \Delta_{c_1}, \Delta_{c_2};
   \Xi_1, \dotsc, \Xi_m; \bang p; \Omega_1, \dotsc, \Omega_k; \Lambda_1,
   \dotsc, \Lambda_m; \Upsilon_1, \dotsc, \Upsilon_n]$
      \item $\lstack{P} = \lstack{P'}, f, \lstack{P''}$
      \item $f$ turns into $f' = [\Gamma_2; \Delta_{c_1},
         \Delta_{c_2}; \Xi_1, \dotsc, \Xi_m; \bang p; \Omega_1, \dotsc,
         \Omega_k; \Lambda_1, \dotsc, \Lambda_m; \Upsilon_1, \dotsc, \Upsilon_n]$
      \item $\mc{AB} \Gamma; \Delta_{c_1}; \Xi_N; \Gamma_{N1}; \Delta_{N1};
      \Delta_{c_2}, \Xi_1, \dotsc, \Xi_m; \cdot; \lstack{C'}; \lstack{P'''}, f',
      \lstack{P''};
      \Omega_N; \Delta_N \rightarrow \outsem$
      (well-formed in relation to $T$)
      \item $\Delta_{c_1}, \Delta_{c_2} = (\Delta_1, \Delta_2, \Xi) - (\Xi_1, \dotsc,
            \Xi_m)$
   \end{itemize}

\end{itemize}
\end{lemma}

Proving that matching the body of a comprehension is sound in relation to HLD
follows the structure of the Lemma~\ref{thm:body_match}. The lemma uses mutual
induction on the recursive judgments $\mc{AB}$and $\contc{AB}$and considers the three
possible results of matching: failure, success with no backtracking and success
with backtracking.

In order to apply a comprehension again, we need to reuse the continuation
stacks. However, in order to use $\lstack{C}$ and $\lstack{P}$ safely, we need
to prove that $\lstack{C}$ will have at most one updated linear continuation
frame and $\lstack{P}$ will have all its frames updated to account the
consumption of the facts from the previous application of the comprehension.

We first prove some auxiliary theorems.

\begin{theorem}[Full stack update]\label{thm:stack_update}
If $\strans \Xi; \lstack{P}; \lstack{P'}$ then $\lstack{P'}$ will be the
transformation of stack $\lstack{P}$ where
every frame $f \in \lstack{P}$, where $f = [\Gamma'; \Delta_N; \cdot; \bang p; \Omega; \cdot;
      \Upsilon])$, will turn into $f' = [\Gamma'; \Delta_N - \Xi; \cdot;
      \bang p; \Omega; \cdot; \Upsilon]$, where $f' \in \lstack{P'}$.
\end{theorem}
\begin{proof}
Straightforward induction on the size of $\lstack{P}$.
\end{proof}

\begin{theorem}[From update to derivation]\label{thm:from_update_to_derivation}

Assume a comprehension $AB = \compsz{A}{B}$. \\ A stack update
$\fix{AB} \Gamma; \Delta; \Xi_N; \Gamma_{N1}; \Delta_{N1}; \Xi; \lstack{C};
\lstack{P}; \Omega_N; \Delta_N \rightarrow \outsem$ implies a derivation \texttab$\dc{AB}
\Gamma; \Delta_N - \Xi; \Xi_N, \Xi; \Gamma_{N1}; \Delta_{N1}; B; \lstack{C'} ;
\lstack{P'}; \Omega_N \rightarrow \outsem$, where:

\begin{itemize}[leftmargin=*]
   \item If $\lstack{C} = \cdot$ then $\lstack{C'} = \cdot$

   \item If $\lstack{C} = \lstack{C_1}, (\Delta_a; \Delta_b; \cdot; p; \Omega; \cdot; \Upsilon)$
   then $\lstack{C'} = (\Delta_a - \Xi; \Delta_b - \Xi; \cdot; p; \Omega; \cdot;
         \Upsilon)$

   \item $\lstack{P'}$ is the transformation of stack $\lstack{P}$, where for every frame $f \in
   \lstack{P}$ of the form $[\Gamma'; \Delta_N; \cdot; \bang p; \Omega; \cdot; \Upsilon]$
   will turn into $f' = [\Gamma';\Delta_N-\Xi;\cdot;\bang p;\Omega;\cdot;\Upsilon]$

\end{itemize}
\end{theorem}
\begin{proof}
Use induction on the size of the stack $\lstack{C}$ to get the result of
$\lstack{C'}$ and then apply Theorem~\ref{thm:stack_update} to get
$\lstack{P'}$.
\end{proof}


Now we prove that a match of a comprehension's body implies the start of a
derivation of the comprehension's head with correct continuation stacks. Note
that $\Omega = \cdot$ in $\matchlldc$, so there is nothing left to match.

\begin{corollary}[Match to derivation]\label{thm:match_to_derivation}
If $\mc{AB} \Gamma; \Delta; \Xi_N; \Gamma_{N1}; \Delta_{N1}; \Xi; \cdot; B;
\lstack{C}; \lstack{P};
\Omega_N; \Delta_N \rightarrow \outsem$ then\\
$\dc{AB} \Gamma; \Delta_N - \Xi; \Xi_N, \Xi; \Gamma_{N1}; \Delta_{N1}; B; \lstack{C'};
\lstack{P'}; \Omega_N \rightarrow \outsem$ where:
   
\begin{itemize}[leftmargin=*]
   \item If $\lstack{C} = \cdot$ then $\lstack{C'} = \cdot$
   \item If $\lstack{C} = \lstack{C_1}, (\Delta_a; \Delta_b; \cdot; p; \Omega;
         \cdot; \Upsilon)$ then $\lstack{C'} = (\Delta_a - \Xi; \Delta_b - \Xi; \cdot; p;
            \Omega; \cdot; \Upsilon)$
   \item $\lstack{P'}$ is the transformation of stack $\lstack{P}$, where for every frame $f \in
   \lstack{P}$ of the form $[\Gamma'; \Delta_N; \cdot; \bang p; \Omega; \cdot; \Upsilon]$
   will turn into $f' = [\Gamma';\Delta_N-\Xi;\cdot;\bang p;\Omega;\cdot;\Upsilon]$
\end{itemize}
\end{corollary}

\begin{proof}
Invert the assumption and then apply Theorem~\ref{thm:from_update_to_derivation}.
\end{proof}


\paragraph{Comprehension Derivation}

We also need to prove that deriving the head of a comprehension is sound in
relation to HLD.  With the results of the next theorem we can reuse the
continuation stacks to start the comprehension process all over again, but now
with a non-empty continuation stack.

\begin{theorem}[Comprehension derivation soundness]\label{thm:comprehension_derivation}
If $\dc{AB} \Gamma; \Delta_N; \Xi_N; \Gamma_{N1}; \Delta_{N1}; \Omega_1, \dotsc,
   \Omega_n; \lstack{C}; \lstack{P}; \Omega_N \rightarrow \outsem$ then:

\begin{itemize}[leftmargin=*]
   \item $\dc{AB} \Gamma; \Delta_N; \Xi_N; \Gamma_{N1}, \Gamma_1, \dotsc, \Gamma_n; \Delta_{N1},
   \Delta_1, \dotsc, \Delta_n; \cdot; \lstack{C}; \lstack{P}; \Omega_N \rightarrow
   \outsem$;

   \item $\forall_{\Omega_x}.($ if $\dz \Gamma; \Delta_N; \Xi_N;
   \Gamma_{N1}, \Gamma_1, \dotsc, \Gamma_n; \Delta_{N1}, \Delta_1, \dotsc,
   \Delta_n; \Omega_x \rightarrow \outsem$ then

   $\dz \Gamma; \Delta_N; \Xi_N; \Gamma_{N1}; \Delta_{N1}; \Omega_1, \dotsc,
   \Omega_n, \Omega_x \rightarrow \outsem)$.

\end{itemize}
\end{theorem}

\begin{proof}
Straightforward induction on $\Omega_1, \dotsc, \Omega_n$.
\end{proof}

The second result of this theorem is the soundness result we need because it
will allow us to reconstruct the derivation tree in HLD.


\paragraph{Multiple Comprehension Derivation} We are interested in proving that
if we start with a given comprehension match $\matchlldc$ then we can apply the
comprehension several times.

\begin{theorem}[Multiple comprehension derivation]\label{thm:multiple_comprehension_derivation}
Consider a triplet $T = A; \Gamma; \Delta_{N}$ and a comprehension $AB =
\compsz{A}{B}$. Assume that there exists $n \geq 0$ applications of $AB$
where the $i_{th}$ application is related to the following contexts:
\begin{description}
   \item[$\Delta_i$]: context of derived linear facts;
   \item[$\Gamma_i$]: context of derived persistent facts;
   \item[$\Xi_i$]: context of consumed linear facts.
\end{description}

Since each application consumes $\Xi_i$ then the initial context $\Delta_N =
\Delta, \Xi_1, \dotsc, \Xi_n$. We now define the two main implications of the
theorem.

\begin{itemize}[leftmargin=*]
   \item Assume that $\Delta_N = \Delta_a, \Delta_b$, $\Delta_b =
   \Delta'_b, p_1$ and there is a frame $f = (\Delta_a, p_1; \Delta'_b; \cdot;
         p; \Omega; \cdot; \Upsilon)$.

   If $\mc{AB} \Gamma; \Delta_a, \Delta'_b; \Xi_N; \Gamma_{N1}; \Delta_{N1};
      p_1; \Omega; f; \lstack{P}; \Omega_N; \Delta, \Xi_1, \dotsc, \Xi_n \rightarrow
      \outsem$ (well-formed in relation to $T$) then:

   \begin{itemize}[leftmargin=\secondm]
      \item $n$ comprehensions are derived:\\
      $\done \Gamma; \Delta_N; \Xi_N, \Xi_1, \dotsc, \Xi_n; \Gamma_{N1},
      \Gamma_1, \dotsc, \Gamma_n; \Delta_{N1}, \Delta_1, \dotsc, \Delta_n; \Omega_N \rightarrow \outsem$
      \item $n$ $\mz$propositions for the $n$ comprehension matches:
      \begin{itemize}[leftmargin=\thirdm]
         \item $\mz \Gamma; \Xi_1 \rightarrow A$
         \item $\dots$
         \item $\mz \Gamma; \Xi_n \rightarrow A$
      \end{itemize}
      \item $n$ derivation implications for HLD: \\
      $\forall_{\Omega_x}.($ if $\dz \Gamma; \Delta, \Xi_{i+1}, \dotsc, \Xi_{n}; \Xi_N, \Xi_1,
            \dotsc, \Xi_i; \Gamma_{N1}, \Gamma_1, \dotsc, \Gamma_i; \Delta_{N1},
            \Delta_1, \dotsc, \Delta_i; \Omega_x \rightarrow \outsem$ then $\dz \Gamma; \Delta, \Xi_{i+1}, \dotsc, \Xi_{n}; \Xi_N, \Xi_1,
            \dotsc,
            \Xi_i; \Gamma_{N1}, \Gamma_1, \dotsc, \Gamma_{i-1}; \Delta_{N1},
            \Delta_1, \dotsc, \Delta_{i-1}; B, \Omega_x \rightarrow \outsem)$
   \end{itemize}

   \item If $\mc{AB} \Gamma; \Delta_N; \Xi_N; \Gamma_{N1}; \Delta_{N1}; \cdot; \Omega;
      \cdot; \lstack{P}; \Omega_N; \Delta, \Xi_1, \dotsc, \Xi_n \rightarrow \outsem$ (well-formed in relation to $T$) then:

   \begin{itemize}[leftmargin=\secondm]
      \item $n$ comprehensions are derived:\\
      $\done \Gamma; \Delta_N; \Xi_N, \Xi_1, \dotsc, \Xi_n; \Gamma_{N1},
      \Gamma_1, \dotsc, \Gamma_n; \Delta_{N1}, \Delta_1, \dotsc, \Delta_n; \Omega_N \rightarrow \outsem$

      \item $n$ $\mz$propositions for the $n$ comprehension matches:
      \begin{itemize}[leftmargin=\thirdm]
         \item $\mz \Gamma; \Xi_1 \rightarrow A$
         \item \dots
         \item $\mz \Gamma; \Xi_n \rightarrow A$
      \end{itemize}

      \item $n$ derivation implications for HLD: \\
      $\forall_{\Omega_x}.($ if $\dz \Gamma; \Delta, \Xi_{i+1}, \dotsc, \Xi_{n}; \Xi_N, \Xi_1,
            \dotsc, \Xi_i; \Gamma_{N1}, \Gamma_1, \dotsc, \Gamma_i; \Delta_{N1},
            \Delta_1, \dotsc, \Delta_i; \Omega_x \rightarrow \outsem$ then $\dz \Gamma; \Delta, \Xi_{i+1}, \dotsc, \Xi_{n}; \Xi_N, \Xi_1,
            \dotsc,
            \Xi_i; \Gamma_{N1}, \Gamma_1, \dotsc, \Gamma_{i-1}; \Delta_{N1},
            \Delta_1, \dotsc, \Delta_{i-1}; B, \Omega_x \rightarrow \outsem)$
   \end{itemize}

\end{itemize}
   
\end{theorem}
\begin{proof}

By mutual induction, first on either the size of $\Delta'_b$ (second argument of
the linear continuation frame) or $\Gamma'$ (second argument of the
persistent frame in $\lstack{P}$) and then on the size of $\lstack{C},
\lstack{P}$.  We only show how to prove the first implication since the
second implication is proven in a similar way.

$\mc{AB} \Gamma; \Delta_a, \Delta'_b; \Xi_N; \Gamma_{N1}; \Delta_{N1}; p_1;
\Omega; f; \lstack{P}; \Omega_N; \Delta, \Xi_1, \dotsc, \Xi_n \rightarrow \outsem$ \hfill (1) assumption\\

By applying Lemma~\ref{thm:comprehension_body_match} to (1), we get:

\begin{itemize}[leftmargin=*]
   \item Failure:
   
   $\done \Gamma; \Delta_N; \Xi_N; \Gamma_{N1}; \Delta_{N1}; \Omega_N
   \rightarrow \outsem$ \hfill (2) from lemma, thus $n = 0$\\
   
   \item Success with no backtracking to frames of stack $\lstack{C}$ or
   $\lstack{P}$:
   
      $\mz \Gamma; \Xi_1 \rightarrow A$ \hfill (2) from lemma \\

      $\Xi_1 = \Xi'_1, p_1$ \hfill (3) from the well-formedness of (1) \\
      $f = (\Delta_a, p_1; \Delta'_b; \cdot; p; \Omega; \cdot; \Upsilon)$ \\

      $\mc{AB} \Gamma; \Delta, \Xi_2, \dotsc, \Xi_n; \Xi_N; \Gamma_{N1};
            \Delta_{N1}; p_1, \Xi'_1; \cdot; \lstack{C'}, f; \lstack{P}; \Omega_N; \Delta_N \rightarrow
            \outsem$ \\
      \dots \hfill (4) from lemma (1) \\

      $f' = (\Delta_a, p_1 - \Xi_1; \Delta_b - \Xi_1; \cdot; p; \Omega; \cdot;
            \Upsilon)$ \\

      $\dc{AB} \Gamma; \Delta, \Xi_2, \dotsc, \Xi_n; \Xi_N, \Xi_1; \Gamma_{N1}; \Delta_{N1}; B; f';
      \lstack{P'}; \Omega_N \rightarrow \outsem$ \\
      \dots \hfill (5) using Corollary~\ref{thm:match_to_derivation} on (4) \\

      $\dc{AB} \Gamma; \Delta, \Xi_2, \dotsc, \Xi_n; \Xi_N, \Xi_1; \Gamma_{N1}, \Gamma_1; \Delta_{N1}, \Delta_1;
            \cdot; f'; \lstack{P'}; \Omega_N \rightarrow \outsem$
      \\ \dots \hfill (6) applying Theorem~\ref{thm:comprehension_derivation} on (5)

      $\forall_{\Omega_x}. ($ if $\dz \Gamma; \Delta, \Xi_2, \dotsc, \Xi_n; \Xi_N, \Xi_1;
            \Gamma_{N1}, \Gamma_1; \Delta_{N1}, \Delta_1; \Omega_x \rightarrow
            \outsem$ then \\ \hspace*{0.5cm} $\dz \Gamma;
            \Delta, \Xi_2, \dotsc, \Xi_n; \Xi_N, \Xi_1; \Gamma_{N1}; \Delta_{N1}; B, \Omega_x
            \rightarrow \outsem)$ \hfill (7) from
      Theorem~\ref{thm:comprehension_derivation} on (5) \\

      $\contc{AB} \Gamma; \Delta, \Xi_2, \dotsc, \Xi_n; \Xi_N, \Xi_1; \Gamma_{N1},
         \Gamma_1; \Delta_{N1}, \Delta_1; f'; \lstack{P'}; \Omega_N
         \rightarrow \outsem$\\ \dots \hfill (8) inversion of (6) \\
        
        By inverting (8) we either fail (thus $n = 1$) or we get a new match.
        For the latter case, we apply mutual induction to get the remaining $n -
        1$ comprehensions.
      
   \item Success with backtracking to the linear continuation frame of stack $\lstack{C}$:
      
      $\mz \Gamma; \Xi_1 \rightarrow A$ \hfill (2) from lemma \\

      $f = (\Delta_a, p_1; \Delta'_b; \cdot; p; \Omega; \cdot; \Upsilon)$ \hfill (3) frame to backtrack to \\
      turns into $f' = (\Delta_a, p_1, \Delta'''_b, p_2; \Delta''_b; \cdot; p; \Omega; \cdot; \Upsilon)$ \hfill (4) resulting frame \\

      $\mc{AB} \Gamma; \Delta, \Xi_2, \dotsc, \Xi_n; \Xi_N; \Gamma_{N1};
\Delta_{N1}; p_2, \Xi'_1; \cdot; \lstack{C'}, f'; \lstack{P}; \Omega_N; \Delta_N \rightarrow
\outsem$\\ \dots \hfill (5) from lemma (1) \\
      
      Use the same approach as the case with no backtracking.
      
   \item Success with backtracking to a persistent continuation frame of stack
   $\lstack{P}$:

      $\mz \Gamma; \Xi_1 \rightarrow A$ \hfill (2) from lemma \\

      $f = [\Gamma''_1, p_2, \Gamma''_2; \Delta_N; \cdot; \bang p; \Omega; \cdot; \Upsilon]$ \hfill (4) from theorem \\
      turns into $f' = [\Gamma''_2; \Delta_N; \cdot; \bang p; \Omega; \cdot;
      \Upsilon]$ \hfill (5) from theorem \\

      $\mc{AB} \Gamma; \Delta, \Xi_2, \dotsc, \Xi_n; \Xi_N; \Gamma_{N1};
\Delta_{N1}; \Xi_1; \cdot; \lstack{C'}; \lstack{P'}, f', \lstack{P''}; \Omega_N; \Delta_N \rightarrow
\outsem$ \\ \dots \hfill (6) from theorem \\
         
      Use the same approach as the case with no backtracking.
      
\end{itemize}
\end{proof}

From this theorem, we derive three important propositions for HLD: (1) the final
derivation proposition; (2) the matching propositions for each comprehension
application; (2) derivation implications to get from (1) to a derivation
judgment without any derivations of the comprehension. However, the theorem
starts from an initial stack with frames and the comprehension process starts
with an empty stack. We need another theorem that gives us one application of
the comprehension plus the other $n$ that we get from this theorem.

\begin{lemma}[All comprehensions]\label{thm:comprehension}
Consider a triplet $T = A; \Gamma; \Delta_{N}$ and a comprehension $AB =
\compsz{A}{B}$. Assume that there exists $n \geq 0$ applications of $AB$
where the $i_{th}$ application is related to the following contexts:
\begin{description}
   \item[$\Delta_i$]: context of derived linear facts;
   \item[$\Gamma_i$]: context of derived persistent facts;
   \item[$\Xi_i$]: context of consumed linear facts.
\end{description}

Since each application consumes $\Xi_i$ then the initial context $\Delta_N =
\Delta, \Xi_1, \dotsc, \Xi_n$.

If $\mc{AB} \Gamma; \Delta, \Xi_1, \dotsc, \Xi_n;
\Xi_N; \Gamma_{N1}; \Delta_{N1}; \cdot; A; \cdot; \cdot; \Omega_N;
\Delta, \Xi_1, \dotsc, \Xi_n \rightarrow \outsem$ (well-formed in
relation to $T$) then:

\begin{itemize}[leftmargin=*]
   \item $n$ comprehensions are derived:\\
   $\done \Gamma; \Delta_N; \Xi_N, \Xi_1, \dotsc, \Xi_n; \Gamma_{N1},
   \Gamma_1, \dotsc, \Gamma_n; \Delta_{N1}, \Delta_1, \dotsc, \Delta_n; \Omega_N \rightarrow \outsem$
   \item $n$ $\mz$propositions for the $n$ comprehension matches:
   \begin{itemize}[leftmargin=\secondm]
      \item $\mz \Gamma; \Xi_1 \rightarrow A$
      \item $\dots$
      \item $\mz \Gamma; \Xi_n \rightarrow A$
   \end{itemize}
   \item $n$ derivation implications for HLD: \\
   $\forall_{\Omega_x}.($ if $\dz \Gamma; \Delta, \Xi_{i+1}, \dotsc, \Xi_n; \Xi_N, \Xi_1,
         \dotsc, \Xi_i; \Gamma_{N1}, \Gamma_1, \dotsc, \Gamma_i; \Delta_{N1},
         \Delta_1, \dotsc, \Delta_i; \Omega_x \rightarrow \outsem$ then $\dz \Gamma; \Delta, \Xi_{i+1}, \dotsc, \Xi_n; \Xi_N, \Xi_1,
         \dotsc,
         \Xi_i; \Gamma_{N1}, \Gamma_1, \dotsc, \Gamma_{i-1}; \Delta_{N1},
         \Delta_1, \dotsc, \Delta_{i-1}; B, \Omega_x \rightarrow \outsem)$
\end{itemize}
\end{lemma}

\begin{proof}
Apply Lemma~\ref{thm:comprehension_body_match} to get two sub-cases:
   
\begin{itemize}[leftmargin=*]
   \item Match fails:
   
   
   $\done \Gamma; \Delta_N; \Xi_N; \Gamma_{N1}; \Delta_{N1}; \Omega_N
   \rightarrow \outsem$\\
   \dots \hfill (1) no comprehension application was possible ($n = 0$)\\
   
   \item Match succeeds:
   
   $\mc{AB} \Gamma; \Xi_2, \dotsc, \Xi_n; \Xi_N; \Gamma_{N1}; \Delta_{N1};
\Xi_1; \cdot; \lstack{C}; \lstack{P}; \Omega_N; \Delta_N \rightarrow \outsem$
   
   \dots \hfill (1) result from Lemma~\ref{thm:comprehension_body_match}
   
   $\mz \Gamma; \Xi_1 \rightarrow A$
   \hfill (2) also from Lemma~\ref{thm:comprehension_body_match}
   
   $\dc{AB} \Gamma; \Delta, \Xi_2, \dotsc, \Xi_n; \Xi_N, \Xi_1; \Gamma_{N1}; \Delta_{N1}; B; \lstack{C'};
\lstack{P'}; \Omega_N \rightarrow \outsem$
   
   \dots \hfill (3) applying Corollary~\ref{thm:match_to_derivation} on (1)
   
   $\dc{AB} \Gamma; \Delta, \Xi_2, \dotsc, \Xi_n; \Xi_N, \Xi_1; \Gamma_{N1}, \Gamma_1; \Delta_{N1}, \Delta_1;
   \cdot; \lstack{C'}; \lstack{P'}; \Omega_N \rightarrow \outsem$
   
   \dots \hfill (4) using Theorem~\ref{thm:comprehension_derivation} on (3)\\
   
   $\forall_{\Omega_x}. ($ if $\dz \Gamma; \Delta, \Xi_2, \dotsc, \Xi_n; \Xi_N,
         \Xi_1; \Gamma_{N1}, \Gamma_1; \Delta_{N1}, \Delta_1; \Omega_x
         \rightarrow \outsem$ then
   
    \hspace*{0.5cm} $\dz \Gamma; \Delta, \Xi_2, \dotsc, \Xi_n; \Xi_N, \Xi_1; \Gamma_{N1};
    \Delta_{N1}; B, \Omega_x \rightarrow \outsem)$ \\ \dots \hfill (5)
   from the theorem applied in (4)\\
   
   $\contc{AB} \Gamma; \Delta, \Xi_2, \dotsc, \Xi_n; \Xi_N, \Xi_1; \Gamma_{N1},
   \Gamma_1; \Delta_{N1}, \Delta_1; \lstack{C'}; \lstack{P'}; \Omega_N \rightarrow \outsem$
   
   \dots \hfill (6) inversion of (5)\\
   
   Invert (6) to get either $n = 1$ application of the comprehension or apply Theorem~\ref{thm:multiple_comprehension_derivation} to the inversion to get the remaining $n-1$. 
\end{itemize}
\end{proof}

With the previous lemma the comprehension is applied for as many times as the
database allows. We now have to map these $n$ applications to HLD by rebuilding
the proof tree for these $n$ matches and derivations and then using
$n$ when "guessing" the number of iterative definitions in HLD.
