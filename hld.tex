The High Level Dynamic~(HLD) semantics formalize the mechanism of matching rules
and deriving new facts. HLD is a simple layer over \fragment that presents a
simplified overview of the semantics of LM.  Starting from \fragment presented
earlier, we consider $\Gamma$ and $\Delta$ the database of our program. $\Gamma$
contains the database of persistent facts while $\Delta$ the database of linear
facts. We assume that the rules of the program are persistent linear
implications of the form $\bang (A \lolli B)$ that can be used several times.
However, we do not put the rules in the $\Gamma$ context but in a separate
context $\Phi$. In HLD, we ignore the right side of the sequent calculus and use
\emph{chaining} and \emph{inversion} on the $\Delta$ and $\Gamma$ contexts so
that we only have atomic facts (e.g., the database of facts). To apply rules we
use \emph{focusing}~\cite{Andreoli92logicprogramming} on one of the derivation
rules in $\Phi$. Note that in the focusing process we assume that all the atoms
(facts) are positive thus the chaining proceeds in a \emph{forward chaining}
fashion.~\footnote{Note that neither HLD or LLD model the use of variable bindings when
matching facts from the database. The formalization of bindings tends to
complicate the formal system and it is not necessary for a good understanding of
the system. Instead, we assume that all facts of type $\emph{fact}(\hat{x})$ do
not have the argument $\hat{x}$.}

An operational step is performed by applying a single inference rule, resulting
in a state transition represented as $\Gamma_1; \Delta_1 \rightarrow \Gamma_2;
\Delta_2$. The entry point of HLD is the judgment $\doz \Gamma; \Delta; \Phi
\rightarrow \Xi'; \Gamma'; \Delta'$.  $\Gamma$, $\Delta$ and $\Phi$ have the
meaning explained before, while $\Xi'$, $\Gamma'$ and $\Delta'$ are output
multi-sets from applying one of the rules in $\Phi$ and are usually written as
$\outsem$. $\Xi'$ is the set of consumed linear resources, $\Gamma'$ is the set
of derived persistent facts and $\Delta'$ is the set of derived linear facts.
Note that for HLD semantics there is no concept of rule priority, therefore a
rule is picked non-deterministically.

In HLD computation proceeds as follows:
\begin{itemize}[leftmargin=*]
   \item The rule is unpacked by "guessing" the correct variable assignments for
   the $\forall$ connective;
   \item Once the linear implication of the $A \lolli B$ is reached, the $\lolli
   L$ rule splits non-deterministically the set of matched facts $\Delta$ into
   $\Delta_a$ and $\Delta_b$, where $\Delta_1$ is the set of linear facts used
   for matching the rule body; Matching uses the judgment 
   $\mz \Gamma; \Delta \rightarrow A$ for a given body $A$;
   \item $\dz \Gamma ; \Delta ; \Xi ; \Gamma_1 ; \Delta_1 ; \Omega \rightarrow
   \outsem$ treats the head of the rule $\Omega$ as an ordered context and
   unpacks and derives facts into $\Gamma_1$ and $\Delta_1$.
\end{itemize}

The HLD semantics get complicated when deriving aggregates. First, the aggregate
is assigned an arbitrary $N$ before unfolding and is then unpacked until the
aggregate implication is reached. Here, the $\Delta$ context is split again and
the derivation proceeds normally. The relevant rules are shown
below:\footnote{The aggregate shown collects all values into a list using
   the cons $::$ operator.}

\vspace{-20pt}
{\stuffsize
\[
\infer[\dz \m{agg}^N]
{\dz \Gamma ; \Delta ; \Xi ; \Gamma_1 ; \Delta_1 ; \aggz{N}{A}{B}{C}{\Sigma}, \Omega
   \rightarrow \outsem}
{\dz \Gamma ; \Delta ; \Xi ; \Gamma_1 ; \Delta_1 ;
   \aggunfold{N-1}{A}{B}{C}{\Sigma},
   \Omega \rightarrow \outsem}
\]

\[
\infer[\dz \m{agg}^0]
{\dz \Gamma ; \Delta ; \Xi ; \Gamma_1 ; \Delta_1 ; \aggz{0}{A}{B}{C}{\Sigma}, \Omega
   \rightarrow \outsem}
{\dz \Gamma ; \Delta ; \Xi ; \Gamma_1 ; \Delta_1 ; \aggunfoldz{C}{\Sigma}, \Omega
   \rightarrow \outsem}
\]

\[
\infer[\dz \lolli]
{\dz \Gamma ; \Delta_a, \Delta_b ; \Xi ; \Gamma_1 ; \Delta_1 ; A \lolli B,
   \Omega \rightarrow \outsem}
{\mz \Gamma ; \Delta_a \rightarrow A & \dz \Gamma ; \Delta_b ; \Xi, \Delta_a ;
   \Gamma_1 ; \Delta_1 ; B, \Omega \rightarrow \outsem}
\]
}
\vspace{-20pt}
