The High Level Dynamic~(HLD) semantics formalize the mechanism of matching rules
and deriving new facts. HLD is a simple layer over the sequent calculus that
presents a simplified overview of the semantics of LM. We consider $\Gamma$ and
$\Delta$ the database of our program. $\Gamma$ contains the database of
persistent facts while $\Delta$ the database of linear facts. We assume that the
rules of the program are persistent linear implications of the form $\bang (A
\lolli B)$ that can be used several times. However, we do not put the rules in
the $\Gamma$ context but in a separate context $\Phi$. The persistent terms
associated with each comprehension and aggregate are put in the $\Pi$ dictionary
that maps symbols to persistent terms.

In HLD, we ignore the right side of the sequent calculus and use \emph{chaining}
and \emph{inversion} on the $\Delta$ and $\Gamma$ contexts so that we only have
atomic facts (e.g., the database of facts). To apply rules we use
\emph{focusing}~\cite{Andreoli92logicprogramming} on one of the derivation rules
in $\Phi$. Note that in the focusing process, we assume that all the atoms
(facts) are positive thus the chaining proceeds in a \emph{forward chaining}
fashion.~\footnote{Note that neither HLD or the low level abstract machine
   explicitly use of variable bindings when matching facts from the database. The
   formalization of bindings tends to complicate the formal system and it is not
   necessary for a good understanding of the system.}

An operational step is performed by applying a single inference rule, resulting
in a state transition represented as $\Gamma_1; \Delta_1 \rightarrow \Gamma_2;
\Delta_2$. The entry point of HLD is the judgment
$\doz{\Gamma}{\Delta}{\Phi}{\Pi}{\Xi'}{\Gamma'}{\Delta'}$. The contexts
$\Gamma$, $\Delta$ and $\Phi$ have the meaning explained before, while $\Xi'$,
$\Gamma'$ and $\Delta'$ are output multi-sets from applying one of the rules in
$\Phi$ and are usually written as $\outsem$. $\Xi'$ is the set of consumed
linear resources, $\Gamma'$ is the set of derived persistent facts and $\Delta'$
is the set of derived linear facts.  Note that for HLD semantics there is no
concept of rule priority, therefore a rule is picked non-deterministically.

In HLD computation proceeds as follows:
\begin{itemize}[leftmargin=*]

   \item The rule is unpacked by "guessing" the correct variable assignments for
   the $\forall$ connective;

   \item The linear implication $A \lolli B$ non-deterministically splits the
   set of matched facts $\Delta$ into $\Delta_a$ and $\Delta_b$, where
   $\Delta_1$ is the set of linear facts used for matching the rule body;
   Matching uses the judgment $\mz{\Gamma}{\Delta_1}{A}$ for a given
   body $A$;

   \item $\dz{\Gamma}{\Pi}{\Delta}{\Xi}{\Gamma_1}{\Delta_1}{\Omega}{\outsem}$
   treats the head of the rule $\Omega$ as an ordered context and unpacks facts
   into $\Gamma_1$ and $\Delta_1$.

\end{itemize}

Aggregates are computed during the derivation the head of a rule.  First, we
look into $\Pi$ for the appropriate persistent term and then immediatelly apply
the top linear implication ($\m{copy}$ rule in the $\Pi$ context followed by the
$\lolli L$ rule of the sequent calculus). The aggregate term is substituted by
either the final term (if the aggregate is complete)\footnote{The aggregate
shown collects all values into a list using the cons $::$ operator.} or by the
recursive case. On the recursive case, the implication is deconstructed using
$\dzname \lolli$ where the matching judgment is used again in order to match the
body of the aggregate and then derive the two heads of the aggregate. The
relevant rules are presented as follows:

{\stuffsize
\[
\infer[\dzname \m{agg}_1]
{\dz{\Gamma}{\Pi}{\Delta}{\Xi}{\Gamma_1}{\Delta_1}{\defstwo{agg}{\widehat{V}}{\Sigma},
   \Omega}{\outsem}}
{\begin{gathered}
   \Pi(\m{agg}) = \forall_{\widehat{v}, \Sigma'}.
   (\defstwo{agg}{\widehat{v}}{\Sigma'} \lolli ((\lambda x. C)\Sigma' \with (\forall_{\widehat{x}, \sigma}.
                                                (A \lolli B \otimes
                                                 \defstwo{agg}{\widehat{v}}{\Sigma'
                                                 + \sigma})))) \\
   \dz{\Gamma}{\Pi}{\Delta}{\Xi}{\Gamma_1}{\Delta_1}{\forall_{\widehat{x},
   \sigma}. (A \lolli B \otimes \defstwo{agg}{\widehat{v}}{\Sigma
    + \sigma})\{\widehat{V}/\widehat{v}\}\{\Sigma/\Sigma'\}, \Omega}{\outsem}
   \end{gathered}
}
\]

\[
\infer[\dzname \m{agg}_2]
{\dz{\Gamma}{\Pi}{\Delta}{\Xi}{\Gamma_1}{\Delta_1}{\defstwo{agg}{\widehat{V}}{\Sigma},
   \Omega}{\outsem}}
{\begin{gathered}
   \Pi(\m{agg}) = \forall_{\widehat{v}, \Sigma'}.
   (\defstwo{agg}{\widehat{v}}{\Sigma'} \lolli ((\lambda x. C)\Sigma' \with (\forall_{\widehat{x}, \sigma}.
                                                (A \lolli B \otimes
                                                 \defstwo{agg}{\widehat{v}}{\Sigma'
                                                 + \sigma})))) \\
   \dz{\Gamma}{\Pi}{\Delta}{\Xi}{\Gamma_1}{\Delta_1}{(\lambda x.
      C\{\widehat{V}/\widehat{v}\}\{\Sigma/\Sigma'\} x) \Sigma, \Omega}{\outsem}
   \end{gathered}
}
\]

\[
\infer[\dzname \lolli]
{\dz{\Gamma}{\Pi}{\Delta_a, \Delta_b}{\Xi}{\Gamma_1}{\Delta_1}{A \lolli B,
   \Omega}{\outsem}}
   {\mz{\Gamma}{\Delta_a}{A} & \dz{\Gamma}{\Pi}{\Delta_b}{\Xi, \Delta_a}
      {\Gamma_1}{\Delta_1}{B, \Omega}{\outsem}}
\]
}
